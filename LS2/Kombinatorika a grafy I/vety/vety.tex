\documentclass[10pt,a4paper]{article}
\usepackage[utf8]{inputenc}
\usepackage{amsmath}
\usepackage[czech]{babel}
\usepackage{amsfonts}
\usepackage{amssymb}
\begin{document}
\title{Věty z předmětu Kombinatorika a grafy I.}
\author{David Nápravník}
\maketitle


\tableofcontents
\section{n! \& polynom n}
$
(\frac{n}{e})^n \geq n! \geq en(\frac{n}{e})^n
$
\subsection*{důkaz}
Nechť $x \geq 0$. Potom $e^x = 1 + \frac{x}{1}+\frac1{x^2}{2!} ... \geq \frac{x^n}{n!}$ ; $n! \geq \frac{x^n}{e^x}$

\section{binomic \& polynom}
Pro $ 1 \leq k \leq n $ platí
$
\binom{n}{0}+\binom{n}{1}+\binom{n}{2}+...+\binom{n}{k} \leq (\frac{en}{k})^k
$

\section{binomic \& odmocnina}
$
\frac{2^{2m}}{2\sqrt{m}} \leq
\binom{2m}{m} \leq1
\frac{2^{2m}}{\sqrt{2m}}
$


\section{lat. čtverce}
Nechť $M^1, M^2, ..., M^t$ jsou latinské čtverce řádu $n$ z nichž 
každé dva jsou navzájem ortogonální. Potom $t \leq n+1$










\end{document}