\documentclass[a4paper]{article}
\usepackage[utf8]{inputenc}
\usepackage{amsmath}
\usepackage{amsfonts}
\usepackage{amssymb}
\setlength{\parindent}{0in}
\usepackage{fancyhdr}

\begin{document}

\pagestyle{fancy}
\rhead{David Nápravník - přezdívka " :) "}

\section{du}
\subsection{}

$$
(a_n) = (1, -1, 2, -1, 3, -3, ...)
$$
$
...
$

$$
(b_n) = (1, -3, 5, -7, 9, -11, ...)
$$
\\
prohodíme $-x$ za $x$, přičteme $(1, 1, 1, ...)$ a vydělíme 2
\\ \\
$
\sum_{n \geq 0}(2n + 1)x^n = 
\frac{1-x}{(1+x)^2}
$


$$
(c_n) = (1, 4, 9, 16, 25, 36, ...)
$$
$
\sum_{n \geq 0} n^2 x^n 
$

\subsection{}
$$
[x^5] : (2x - 1)^-2
$$
$
\frac{1}{(1-2x)^2}=
\sum_{n=0}^\infty x^n 2^n (1+n)
\\ \\
\left[ x^5 \right] : 192
$

$$
[x^5] : (1 + x)^{-1/3}
$$
$
\sum_{n=0}^\infty \binom{-\frac{1}{3}}{n} x^n
\\ \\
\left[ x^5 \right] : -\frac{91}{729}
$


\subsection{}
$$
a_0 = 0, a_1 = 1, a_n = a_{n-1} + a_{n-2} + 2
$$
$
a_n = 2^n -1
$

$$
b_0 = 2, b_1 = 3, b_n = 3_{n-2} - 2b_{n-1}
$$
$
a_n = \frac{9-(-3)^n}{4}
$


\end{document}