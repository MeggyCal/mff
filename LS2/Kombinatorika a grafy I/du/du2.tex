\documentclass[a4paper]{article}
\usepackage[utf8]{inputenc}
\usepackage{amsmath}
\usepackage{amsfonts}
\usepackage{amssymb}
\setlength{\parindent}{0in}
\usepackage{fancyhdr}

\begin{document}

\pagestyle{fancy}
\rhead{David Nápravník - přezdívka ":)"}

\setcounter{section}{1}
\section{du}
\subsection{}
Mejme latinsky ctverec rozmeru $n*n$ s vyplnenymi $m$ radky.\\
Kde $m = n-1$, protoze pro ostatni $m$ to bude pote platit tez.\\
\\
Dokazme sporem, predpokladejme ze existuje $n-1$ korektne vyplennych
radku a na $n$tem radku neexistuje permutace takova, ze odpovida definici
lat. ctvercu.\\
Pak existuje index v permutaci jez nemuze obsahovat zadne z cisel $1$ az $n$.
Takove cislo, ale vzdy musime najit,
nebot prave takove jedno cislo chybi v danem sloupci lat. ctverce\\
A takove cislo nemuze byt ani obsazeno v nasem poslednim radku, nebot na vsech ostatnich pozicich byt nemuze, nebot je obsazeno v kazdem sloupci nad danou pozici.

\subsection{}
\begin{itemize}
\item \\
b
\end{itemize}
 

\end{document}