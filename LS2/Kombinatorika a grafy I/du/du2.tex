\documentclass[a4paper]{article}
\usepackage[utf8]{inputenc}
\usepackage{amsmath}
\usepackage{amsfonts}
\usepackage{amssymb}
\setlength{\parindent}{0in}
\usepackage{fancyhdr}

\begin{document}

\pagestyle{fancy}
\rhead{David Nápravník - přezdívka ":)" - v.3}

\setcounter{section}{1}
\section{du}
\subsection{}
Mejme uplny bipartitni graf velikosti $n,n$ kde na jedne strane jsou indexy policek a
na druhe cisla ktera do nich budeme dosazovat.\\
Dale mejme latinsky obdelnik rozmeru $r*n$ kde $n>r$.\\
V bipartitnim grafu odeberme hrany jenz jiz jsou soucasti latinskeho obdelniku.\\
\textit{Pozorovani:} kazda hrana je stupne $n-r$\\
Tudiz radku $r+1$ pak bude na pozici kazdeho indexu mozno priradit rozdilne cislo.
Dle hran ktere vedou z daneho vrcholu v grafu do vrcholu s cislem. 


\subsection{}
\begin{itemize}
\item plati\\
Dokazme sporem, mejme graf takovy, ze obsahuje dva vrcholi na kruznici a jeden
takovy, ze na kruznici nelezi. Pak tento treti vrchol musi byt list (nebo byt
mostem k listu) a to je ve sporu s predpokladem ze je graf 2-souvisly.

\item plati\\
Takovy vrchol musi existovat, nebot pro "zniceni souvislosti" grafu musime
odebrat 3 vrcholi, takze muzeme odebrat vrchol $z$ a graf musi byt vrcholove
2-souvisly a tudiz obsahuje kruznici. Neboli kruznici na ktere nelezel bod $z$.

\item neplati\\
specificky nebude platit pro graf "motylka"\\
motylek: graf o dvou n-kompletnich grafech spojenych pres jediny vrchol.
Takovy graf bude n-hranove souvisly, ale pouze 1 vrcholove.
Tudiz pro $k>1$ neexistuje dostatecne velke $l$, tudiz tvrzeni neplati.
\end{itemize}


\subsection{}
\textbf{puze pro k=1}, jedine graf kde kazda hrana obsahuje maximalne jeden
vrchol splni Hallovu podminku. Nebot pro $k>1$ existuje hypergraf takovy, ze
mame vice hran nez vrcholu.


\subsection{}
Takovy graf bude mit vrcholi stupne minimalne $k$, nebot kazdy vrchol musi
pobrat $k$ hran potrebnych pro $k$ disjunktnich koster.\\
Tudiz bude \textbf{hranove k-souvisly}.\\
O vrcholove souvislosti nam to nic nerekne, nebot muze nastat graf typu motylek
viz vise. A takovy graf bude tedy pouze \textbf{vrcholove 1-souvisly}.


\end{document}