\documentclass[a4paper]{article}
\usepackage[utf8]{inputenc}
\usepackage{amsmath}
\usepackage{amsfonts}
\usepackage{amssymb}
\setlength{\parindent}{0in}
\usepackage{fancyhdr}

\begin{document}

\pagestyle{fancy}
\rhead{David Nápravník - přezdívka ":)"}

\setcounter{section}{1}
\section{du}
\subsection{}
Mejme latinsky ctverec rozmeru $n*n$ s vyplnenymi $m$ radky.\\
Kde $m = n-1$, protoze pro ostatni $m$ to bude pote platit tez.\\
\\
Dokazme sporem, predpokladejme ze existuje $n-1$ korektne vyplennych
radku a na $n$tem radku neexistuje permutace takova, ze odpovida definici
lat. ctvercu.\\
Pak existuje index v permutaci jez nemuze obsahovat zadne z cisel $1$ az $n$.
Takove cislo, ale vzdy musime najit,
nebot prave takove jedno cislo chybi v danem sloupci lat. ctverce\\
A takove cislo nemuze byt ani obsazeno v nasem poslednim radku, nebot na vsech ostatnich pozicich byt nemuze, nebot je obsazeno v kazdem sloupci nad danou pozici.

\subsection{}
\begin{itemize}
\item plati\\
pro kazdy podraf grafu G musi platit, ze existuji 4 vrcholi jez jsou v cyklu
(stupen dva a neni bipartitni) a po odebrani presne dvou se stanou
zbyle dva vrcholi nesouvisle.
Pokud si za zbyle vrcholi dosadime cele komponenty grafu a hrany mezi
odstranenymi vrcholi definujeme jako rez a celou situaci aplikujeme
na kazdy podgraf jez lezi na rezu, tak mame dokazano ze tvrzeni plati.

\item plati\\
pro kazdy podraf grafu G musi platit, ze existuje 5 vrcholu jez jsou stupne minimalne 3. Predpokladejme ze tvrzeni neplati, pak v grafu existuje pouze jedna kruznice obsahujici vsechny vrcholi, to je ale ve sporu s tvrzenim ze kazdy vrchol ma stupen min 3. Neboli takovy graf musi mit hlavni kruznici a min jedno "ucho".

\item plati\\
mejme graf ktery je hranove l-souvisly pak existuje bijektivni zobrazeni $l$ hran na $l$ vrcholu. Nebot kazdou hranu kterou bychom odebrali muzem odebrat i tak ze ji sebereme jeden z vrcholu.
\end{itemize}

\subsection{}
mnozinovy system ma SRR pro kazde kladne $k$, nebot splineme Hallovu podminku.
tj. libovolnych $n$ hran z hypergrafu $G$ ma dohromady alespon $n$ prvku.

\subsection{}
takove grafy budou k-hranove souvisle, nebot mezi kazdymi dvemi body bude minimalne $k$ hranove disjunktnich cest. A tudiz budou i k-vrcholove souvisle.
 

\end{document}