\documentclass[12pt,a4paper]{article}
\usepackage[utf8]{inputenc}
\usepackage[czech]{babel}
\usepackage[T1]{fontenc}
\usepackage{amsmath}
\usepackage{amsfonts}
\usepackage{amssymb}
\usepackage{ dsfont }

\renewcommand{\labelenumi}{(\alph{enumi})}

\title{Úvod do komplexní algebry na žofínském prostoru}
\date{}

\begin{document}

\maketitle


\textbf{Definice 1}
(Žofínský časoprostor).
Nehť parametr $r \in \mathds{N}$ a zobrazení
$m(r) : \mathds{N} \rightarrow \mathds{N},d(r) : \mathds{N} \rightarrow \mathds{N}$
jsou po řadě rok, měsíc a den konání matfyzákého (filosoficko-matfyzákého)
plesu v roce r, přičemž $m(r)$ a $d(r)$ jsou definovány spolkem Matfyzák vždy v roce $r - 1$.
Nehť $\Delta T$ je časový interval
$$
\Delta T := [d(r). m(r). r 19:30 h; (d(r) + 1). m(r): r 2:00 h].
$$
Označme T jednorozměrný čas a $Ž_3$ třídimensionální prostor paláce Žofín, Slovanský ostrov 226, Praha 1. Potom žofínský časoprostor $Ž$ je čtyřrozměrný prostor definovaný
direktním součtem $Ž = Ž_3 \oplus T$ , který splňuje:\\
\begin{enumerate}
\item žofínský časoprostor je nad Vltavou, přičemž prostor $Ž_3$ je nad Vltavou skoro jistě;
\item v intervalu $\Delta T$ je žofínský časoprostor otevřený, jinak je uzavřený;
\item žofínský časoprostor je normovaný (se standardní normou společenského hování);
\item žofínský časoprostor je dobře a úplně definovaný.
\end{enumerate}


\textbf{Definice 2}
(Abstraktní žofínský prostor, zkr. žofínský prostor).
Symbolem C označujme těleso komplexníh čísel. Potom $Ž_{C}^r$ nazveme
abstraktní žofínský prostor nad tělesem komplexníh čísel
(zkráeně žofínský prostor), který splòuje:
\begin{enumerate}
\item $ Ž_C^r$ je omezený;
\item $Ž_C^r$ je dobře definovaný pro daný rok r.
\end{enumerate}


\textbf{Definice 3}
(Struktura žofínského časoprostoru).
Označme podprostory $Ž$ následovně:
\begin{enumerate}
\item $R, R \subset Ž$, RYTÍŘSKÝ sál paláce Žofín,
\item $M, M \subset Ž$, MALÝ sál paláec Žofín,
\item $W, W \subset Ž$, ZIMNÍ ZAHRADA paláce Žofín,
\item $H, H \subset Ž$, HLAVNÍ sál paláec Žofín,
\item $P, P \subset Ž$, PŘÍSÁLÍ hlavního sálu paláce Žofín,
\item $S, S \subset Ž$, PRIMÁTORSKÝ SALÓNEK paláce Žofín,
\item $G, G \subset Ž$, GALERIE paláce Žofín a konečně
\item $U, U \subset Ž$, MUŠLE paláce Žofín.
\end{enumerate}
Označme dále $\Pi$ libovolně zvolený podprostor $Ž$
z podprostorù definovanýh v (a) až (h).
Nehť $\sum$ je libovolný pevně zvolený stùl žofínského časoprostoru z borelovského systému
stolù $S$ určenýh pro návštěvníky plesu. Potom platí
$$
\forall r \in N \forall \sum \in S \exists ! \Pi \subset Ž : \sum \in \Pi
$$

%1Prosíme, neověřujte uzavřenost na sjednoení!    todo

2
(To znamená, že každý stùl $\sum$ se vždy nahází právě v jednom z podprostorù definovanýh
v seznamu (a){(h).)


\textbf{Značení.}
Nehť $s \in Ž^r_C$. Symbolem $\Im(s)$ rozumíme imaginární část a $\Re(s)$ reálnou část
komplexního čísla $s$.


\textbf{Definice 4}
(Struktura žofínského prostoru).
Nehť $s \in \mathcal{Z}^r_C$. Nehť $\Im(s) > 0$. Potom pro
každý stùl $\sum \in \mathfrak{S}$ platí:
\begin{enumerate}
\item $\Re(s) = 10 \Leftrightarrow \sum \in \mathcal{R}$;
\item $\Re(s) = 0 \Leftrightarrow \sum \in \mathcal{M}$;
\item $\Re(s) = 10 \Leftrightarrow \sum \in \mathcal{P}$;
\item $\Re(s) = 11 \Leftrightarrow \sum \in \mathcal{H}$;
\item $\Re(s) = 12 \Leftrightarrow \sum \in \mathcal{S}$;
\item $\Re(s) = 20 \Leftrightarrow \sum \in \mathcal{G}$;
\item $\Re(s) = 21 \Leftrightarrow \sum \ině \mathcal{U}$.
\end{enumerate}
Označíme-li dále $\pi(\Pi)$ celkový počet stolù v podprostoru $\Pi$, potom platí
$$
\forall s \in Z^r_{\mathds{C}} \forall \sum \in (\mathfrak{S} \cap \Pi) : \Im(s) \in \{1, 2, ..., \pi(\Pi)\}.
$$


\textbf{Definice 5}
(Značení vstupenek na ples)
Označme $\Lambda r$ množinu všech vstupenek na ples
v roce r. Nehť funke $n(\sum)$ : $\mathfrak{S}$ ! N určuje počet židlí u stolu $\sum$. Speciálně pro $\sum$ = ;
určuje počet vstupenek na stání. Potom platí
$
\forall r \in N \forall \sum ïn \mathfrak{S} \exists I_{\sum} = \{1; 2; : : : ; n(\sum)\}
$
tak, že 
$
\forall i \in I \sum \exists! \lambda\sum;i
$
r 2 r
a vstupenka $\lambda\sum;i$
r má všehny náležitosti definované spolkem Matfyzák pro rok r. Naví
je-li $\sum 6= ;,$ pak platí:
\begin{enumerate}
\item na každé vstupene $\lambda\sum;i$
r
je uvedeno číslo $s 2 Z^r_C$
, které závisí na $\sum$, nikoli na i;
\item $\Re(s)$ určuje příslušný podprostor $\Pi$;
\item =(s) určuje návštěvníkem vybrané číslo stolu v podprostoru $\Pi$.
\end{enumerate}


\textbf{Lemma 1}. Existuje prosté zobrazení (abstraktního) žofínského prostoru
$Z^r_{\mathds{C}}$ na žofínský
časoprostor $Z$

\textit{Důkaz.}
Dùkaz si laskavý čtenář provede sám za domácí vičení.


\textbf{Lemma 2}
(Brom-Kavalír).
Nehť $s \in Z^r_{\mathds{C}}$
Označme p 2 f1; 0; 1; 2g patro Žofínského
paláce (v rozumném slova smyslu, kde p = 0 značí přízemní podprostor časoprostoru $Z$).
Potom stùl $\sum$ najdeme v p-tém patře, kde
$p =<(s)10:$
Zde bx značí dolní elou část reálného čísla x.

\textit{Důkaz.}
Dùkaz plyne z definice. (Konkrétně definice 4 a intuitivní definice þpatra.


\textbf{Věta 1}
(O stále pohybujíím matfyzákovi)
Žofínský prostor je dobře definovaný.

\textit{Důkaz.}
Je zřejmý. (Návod: použijte definie a lemma 6!)


\textbf{Věta 2}
(Jirotkova plesová).
Existuje bijeke $\sigma : Z^r_{\mathds{C}}$
$\mathfrak{S}$, s 7! $\sum$. V případě =(s) = 0
nemá návštěvník plesu nárok na sezení u stolu v žádném podprostoru $\Pi$ a $\sigma(0) = ;$. číslo s
nazýváme číslem stolu.

\textit{Důkaz.}
Dùkaz je triviální. (Plyne takřka ihned z definie 5 a lemmatu 6.)%todo obr
Císla vstupenek v Gaussově rovině  Schématické znázornění podprostoru Mušle %todo obr
$igma$
Obr. 1. Bijeke $\sigma$ pro několik hodnot s


\textbf{Věta 3} (Euler). Pro každé $x 2 R a y 2 R$ platí
$$
e x+iy = e x (os y + i sin y) 
$$

\textit{Důkaz.}
Dùkaz je na deštivý víkend, resp. na dva semestry.


\textbf{Definice 6} (Značení stolù). Na každém stolu $\sum \in \mathfrak{S}$ je v souladu s větou 9 uvedeno
číslo stolu s v přesném tvaru s =$\Re(s) + =(s) i$, které je vidět jen z určitého směru a dokud
jej někdo neodstraní. Z jiné strany stolu mùže být uvedeno číslo stolu naví v alternativním
zápisu téhož komplexního čísla s s možnou zaokrouhlovaí hybou.


\textbf{Věta 4} (Hledání stolu). Nehť $(s) > 0$. Potom se matfyzák transportuje do správného
patra p Žofínského paláe pod le lemmatu 7 a do příslušného podprostoru $\Pi$ s použitím
definie 4, kde pod le definie 11 bude hledat stùl číslo s. (U stolu si zvolí žid li pod le
vlastního přání.)

\textit{Důkaz.}
Dùkaz je za þplesové vičení.


\textbf{Věta 5} (O zoufalém tanečníkovi). Zoufalý tanečník nemohouí najít svùj stùl použije
větu 10 či znovu si přečte definii 11 a prozkoumá značení stolù z jiného pohledu.

\textit{Důkaz.}
Dùkaz je zřejmý.


\textbf{Věta 6} (O zoufalém filosofovi). Nehť filosof zná číslo s 2 $Z^r_{\mathds{C}}$
, příp. má vstupenku $\lambda 2r$
 s tímto číslem. Nehť na matfyzákém plese (v časoprostoru $Z$)
 je alespoò jeden matfyzák. Potom zoufalý filosof zvládne nalézti stùl, jemuž přísluší číslo s.
Nyní předvedeme konstruktivní dùkaz věty. Hlavní trik spočívá v tom, že záležitost
elegantně převedeme na snadno řešitelný problém.

\textit{Důkaz.}
Předpokládáme, že v žofínském časoprostoru $Z$
je alespoò jeden matfyzák (což je
zřejmě splněno z předpokladu věty), a současně víme, že počet návštěvníkù v $Z$
je konečný
(pokud to neplyne z definie, plyne to z požárníh předpisù). Proto zoufalý filosof na jde
matfyzáka v konečném čase, bude-li hledat šikovným zpùsobem, tj. každého hosta se zeptá
nejvýše jednou. Následně matfyzáka poprosí o pomo s hledáním svého stolu a ukáže mu
vstupenku, resp. sdělí požadované číslo stolu s. Když matfyzák zná číslo s, vyřeší problém
podle věty 12, popřípadě věty 13. Správné řešení sdělí zoufalému filosovovi (samozřejmě
tak, aby jej pohopil), popř. ho ke stolu s dovede (jsme přee na Žofíně, který je normovaný
podle definie 1 ()!). Tak se zoufalý filosof elý šťastný dostane ke stolu $\sigma(s)$. Q.E.D.2



\end{document}