\subsection{Laby}

Počítačovým laboratořím říká každý správný matfyzák pouze {\it lab\/}
a~na matfyzu jich je naštěstí relativně dost. V~každé budově je
nejméně jeden takový, do něhož mají přístup normální uživatelé
(rozuměj všichni studenti matfyzu). Kromě toho existuje ještě
spousta katedrových počítačů, u~kterých sedávají studenti vyšších
ročníků.

K~přístupu do labu potřebujete zpravidla průkaz studenta,
uživatelské konto a službu, která Vás pustí dovnitř.
Po\-tře\-bu\-jete-li založit konto, procedura bývá taková, že si
člověk zažádá, žádosti je zpravidla vyhověno a často okamžitě je
konto založeno --- výjimkou je progresivní lab na Karlově, kde má přístup k~počítačům automaticky každý student Univerzity. Koho je potřeba požádat, sdělí služba. Většinou to
bývá samotná služba, někdy je však potřeba odchytit správce
a~absolvovat školení o specifikách daného labu.

Obsazenost labů závisí na denní době, fázi semestru a poloze
v~Praze. Nejplnější bývají laby v~zimním semestru v~období psaní
zápočtových programů.  Tradičně nejhůř je na tom kolej (dříve tam
prý někteří dokonce squatovali), nejvíce místa je obvykle na
Karlově, pokud tam není výuka. V~každém labu je k~dispozici
provozní řád, který jsou uživatelé povinni znát a dodržovat
a~který mimo jiné určuje i priority práce na počítači (tzn., že
když chce někdo dělat něco do školy, tak má přednost před tím, kdo
chce hrát hry). Číst provozní řády je velmi
poučné (člověk se dočte leccos, co by jeho samotného ani
nenapadlo), často je to jediná aspoň částečně užitečná věc, kterou
člověk může po příchodu do labu dělat v~případě, že žádný počítač
není volný.  Nezapomeňte se také řídit směrnicí děkana č. 4/2008,
%odkaz vypusten
ze které např. vyplývá, že se máte k~počítačům chovat slušně a že
nesmíte používat nelegální software ani psát vulgární e-maily.

\subsubsection{Malá Strana}

Malostranský lab se nachází v~prostorách budovy zvaných Rotunda
(z~hlavního vchodu pořád rovně, než se po levé straně objeví mohutné
chromované dveře vedoucí do velké místnosti). Lab je doslova
obklopen knihovnou a je opravdu nádherný. Právě v~těchto místech
bývaly kdysi přepážky Národní banky.

V půlce labu jsou Vám k~dispozici počítače s~Windows XP, v~druhé převážně Gentoo Linux (oba systémy sdílejí váš home adresář, takže můžete li\-bo\-vol\-ně pře\-chá\-zet). Vzdá\-le\-ně se můžete přihlásit na Solaris. Autor těchto řádků samozřejmě {\it zásadně} používá Linux a k~Windows by {\it za žádnou cenu nesednul}. Ale názory mezi informatiky, jak si umíte představit, se různí. Domácí stránky labu jsou na webové stránce \url{http://www.ms.mff.cuni.cz/labs}.

Účty se tu zakládají centrálně v~pevné dny a časy. Konzultujte se
službou nebo nástěnkou před labem.

\subsubsection{Karlov}

Na Karlově je problémem vůbec lab najít. Vězte tedy, že se nalézá
v~suterénu děkanátní budovy (tedy Ke~Karlovu~3) v~nejzapadlejším koutě,
který by koho mohl napadnout, za dveřmi číslo 36 (sejdete
k~bufetu, ale zabočíte doleva a na konci chodby to je). 

Na každém počítači si můžete vybrat mezi systémem Windows 7 a Ubuntu. Nemusíte se registrovat, protože přihlašování probíhá přes CAS a stačí tak použít heslo, které používáte třeba do SISu.

Bližší informace (třeba seznam nainstalovaných programů) na stránce \url{http://www.mff.cuni.cz/net/karlov/plk/plk.htm}.

\subsubsection{Karlín}

Lab (K10) se nachází v~přízemí budovy, hned vpravo od vchodových
dveří. V~labu je možno získat účet na linuxovém serveru Artax.
Tento účet Vám umožní vzdálenou práci přes SSH, práci na lokálních
stanicích a také přihlašování k~Windows. V~labu existuje též
server Atrey, kde se účty až na vyjímky nezakládají. Na všech
počítačích je možné pracovat pod MS~Windows i pod Linuxem.
Softwarové vybavení laboratoře pokrývá potřeby matematiků. Další
informace viz \url{http://atrey.karlin.mff.cuni.cz/lab/}.

\subsubsection{Troja}

LabTF se od ostatních laborek na matfyzu liší tím, že je ve dvou od
sebe vzdálených místnostech.  V~přízemí v~místnosti~113 (za
bývalým papírnictvím) sedí služba.

Druhá místnost je v~1. patře a má číslo dveří 235 (vyjdete po
schodech, obejdete dvě kulaté posluchárny a tam už to je).  Na rozdíl od spodní místnosti tady probíhá jenom výuka. Bývala tu kamera, která ukazovala v~televizi ve spodní místnosti, co tam kdo tropil.
Občas se několik lidí dole
poťouchle bavilo počínáním nebohého páru, který si myslel, že je
tam sám\dots\ Na všech počítačích je možné pracovat pod
MS~Windows 7, a když se zrovna najde někdo, kdo to rozběhne, tak
i~pod Linuxem (Ubuntu). Další informace viz \url{http://sheeni.troja.mff.cuni.cz/LabTF/}.


\subsubsection{Troja, budova jazyků}

V~Troji naleznete ještě jeden lab (LabTS), který se nachází
v~prvním poschodí budovy jazyků (po vyjití schodů se vydáte chodbou
doleva, za zatáčku a za skleněnými dveřmi doprava).
V~klimatizované místnosti je 16 počítačů s~Windows a Linuxem.
Podrobné informace na \url{http://labts.troja.mff.cuni.cz}.

\subsubsection{Kolej 17. listopadu}

Lab v~přízemí budovy A koleje 17. listopadu obvykle funguje od 8 do 24 hodin s~tím, že má-li služba zrovna dobrou náladu (a po půlnoci jsou zájemci), zůstává lab otevřený dále do non-stop. V žádném případě na to ale nelze spoléhat.

Počítače s~Windows XP obsluhuje Novellovský server. Podrobnosti jsou na webu \url{http://port.kolej.mff.cuni.cz}.

\subsubsection{Alternativy?}

Situace, kdy má počítačová laboratoř zavřeno a spolubydlící si zabral buňku pro radostné trávení času s~přítelkyní jsou časté a může se stát, že budou kolidovat s~chvílemi, kdy opravdu potřebujete pracovat. Máte-li notebook, můžete využít noční studovnu v~Národní technické knihovně v~Dejvicích.

Je otevřená kdykoliv, kdy není otevřená vlastní knihovna, je v~ní k~dispozici WiFi, elektrický proud, záchod a automaty s~pitím a bagetami. Registrovat se ale je potřeba někdy v~běžných otevíracích dobách --- stojí to padesát korun na rok.