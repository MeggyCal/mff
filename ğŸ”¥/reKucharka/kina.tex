\subsection{Kina}


Nejvíce multikin v~Praze provozuje Cinema City, který v~roce 2011 odkoupil zdejší Palace Cinemas a od té doby se nachází ve stavu pokročilé dezintegrace. Najdete je třeba na Andělu (nejnavštěvovanější kino v~republice, pět zastávek tramvají 12 a 20 od Malostranského náměstí), ve Slovanském domě (tj. v~centru, kousek od Náměstí republiky) či v~OC Letňany (od kolejí 17. listopadu daleko, ale bez přestupu autobusem 186).

Na Andělu najdete ještě CineStar, kteří se mohou puchlubit dobrým zvukem a častým pořádáním zajímavých akcí. Kombinace jejich věrnostního programu a studentské slevy může průměrnou cenu za projekci pro častého návštěvníka srazit na stokorunu.

V Paláci Flóra se nachází IMAX – jde o nepříliš kvalitní digitální projekci na obrovské plátno. Před rokem 2010 se promítalo z~analogových pásů s~mimořádně velkými okénky, takže šlo o zážitek srovnatelný s~pozorováním 70 mm kopie.


Kino Světozor má sály dva a najdete ho na Václavském náměstí
naproti knihkupectví Academia. Má 4K projektor a slavnou historii,
sídlí tam od roku 1918 a v~roce 68 v~něm promítali Kinoautomat
Radúze Činčery. V jeho foyer najdete obchod Terryho ponožky, kde
kupčí s~novými i starými filmovými plakáty. Patří do četné
kategorie pražských artkin.

Kino Aero najdete čtyři zastávky autobusem 133 od Florence směrem
do Žižkova (zastávka Biskupcova). Mimo jednoho velkého sálu s~2K
projektorem nabízí volné WiFi a hezký bar. Společně se Světozorem
nedávno vytvořili vlastní distribuční společnost Aerofilms, která
u nás uvedla třeba Valčík s~Bašírem. Přímo na Florenci je kino
Atlas, jeho nabídka je však spíš multi- než art-.

Bio Oko se nachází v~Holešovicích, kousek od Strossmayerova
náměstí (rozhlížejte se cestou vzhůru po ulici Milady Horákové).
Vynikají alternativními možnostmi sezení (různé pytle, plážová
lehátka\dots) a občasným uváděním Bo\-lly\-woo\-dských trháků.

Ponrepo je kino zřízené Národním filmovým archivem. Zařídíte-li si tam průkazku, budete moct třeba denně chodit na všemožné důležité staré artové filmy za 30 Kč.

Každé úterý probíhá v~modré posluchárně Rektorátu projekce {\it
Filmového klubu UK}, určená studentům a pracovníkům univerzity.
Roční členský příspěvek je 100 korun, vstup na jednotlivá
představení je pro držitele členské karty zdarma. Je však potřeba
předem si na webu rezervovat místo. V~klubu se promítají mnohé
filmy ještě před svou oficiální distribuční premiérou. Bližší
informace najdete na adrese \url{http://certik.ruk.cuni.cz/filmklub}.

Další filmový klub otevřený všemu studentstvu funguje na Fakultě humanitních studií UK --- viz \url{http://fkfhs.blogspot.cz/}.