\subsection{Nejoblíbenější hospody}

Jsou chvíle, kdy se i nejpilnější studenti uchylují k~věcem se
studiem nesouvisejícím. Někdo začíná navštěvovat divadla
a~galerie, někdo pak s~výkřiky \uv{pivo, pivo} rovnou hledá
vhodnou občerstvovnu, jiní razí heslo, že nezapitá zkouška je
skoro jako neudělaná. Pravdou je, že jisté rozdíly tu jsou,
například zapití zkoušky nemá vliv na to, jestli Vám za ni na
studijním dají kredity\dots\ Každopádně Praha je město tisíce
věží, pod každou věží je aspoň pět hospod, každý den jich aspoň
deset zanikne a větší počet vznikne. Následující odstavce jsou
tedy značně nekompletní, protože není v~silách konečného počtu
lidí prolézt týden před uzávěrkou všechny následující nálevny.

Jak jste si již mohli všimnout výše, přímo v~budově na Malé Straně
se nachází {\it Profesní dům}, kde je každý den mezi 11:30 a 14:00
na výběr ze tří hotových jídel a několika jídel na objednávku,
později zařízení funguje jako běžná hospoda.

Z~dalších restaurací v~okolí Malé Strany lze hlavně v~letních
měsících doporučit zahrádku restaurace {\it U~Písecké brány}, pár
kroků od metra Hradčanská směrem k~Písecké bráně.

Případně {\it U~Černýho vola} v~komplexu Pražského hradu. Skvělá
atmosféra, Kozel, též rozumné ceny, co víc si může člověk
(student) přát.

Restaurant {\it Pawlovnia} se nachází kousek od koleje \17l,
u~magistrály asi 100 metrů od zastávky Kuchyňka směrem do kopce.
Jídlo i obsluha jsou výborné, ceny poněkud vyšší (vhodné je
návštěvu naplánovat na happy hours). Pohodu na venkovní zahrádce
ruší jen hluk z~blízké magistrály.

Hospodu {\it U~Špačka\/} najdete, půjdete-li nahoru po magistrále a pak
doleva, v~Kubišově ulici. Cesta po magistrále je relativně dlouhá, proto
k~dopravě doporučujeme použít čehokoliv na Vychovatelnu a odtud sejít
dolů po magistrále až k~oné ulici (v~tom případě je to první nenápadná
ulice doprava). K~dobrému Budvaru se přidává i skvělá obsluha.

U~vchodu do ZOO je hospoda {\it U~Lišků}. Skvělé levné Krušovice.

Výborná kuchyně a přátelská obsluha, zahrádka a trochu vyšší ceny,
to je vinný restaurant {\it V~Holešovičkách}.  Jděte podél levé strany
magistrály nahoru a koukejte se.

Když půjdete ještě dál, narazíte na restauraci {\it U~Hofmanů},
kde se točí Gambrinus.  Kousek odtud si můžete zatancovat polku
{\it Na Vlachovce}, hraje se každou sobotu; vaří tam dobře.

Za řekou v~Holešovicích je hospoda na každém rohu a každý má oblíbenou
jinou.  Kdo se chce dobře najíst, určitě si najde kousek od Nádrholu
restauraci {\it U~Kaprála\/} (nápověda: je třeba obejít okrovou budovu
školy). Když vystoupíte z~tramvaje č.~12 na zastávce Maniny, vrátíte
se zpět na nejbližší křižovatku a dáte se doprava, dostanete se do
{\it Domoviny}. Napít se tu můžete Samsona a občas i Platana, zato místní
vzduch by se dal krájet a vyvážet.

V~Holešovicích nelze přehlédnout hostinec {\it Na Kovárně}. Lidé, kteří
jezdí tramvají na Malou Stranu, se naň mohou dívat při každé cestě, je
totiž právě tam, kde se dělí tramvajové koleje (jdete od metra doprava
na první křižovatku, přejdete ji a stojíte přímo před hospodou). Točí
dobrý Prazdroj $12^\circ$ (za rozumnou cenu) a Krušovickou desítku (desítka
a~dvanáctka dnes již vlastně neexistují, tedy Plzeňský ležák a Krušovické
výčepní). Až budete mít pocit, že máte v~krku záněty, zkuste si tu dát
utopence. Zdejší utopenci vypálí všechno, co se jim postaví do cesty.

Hospoda {\it U~Vystřelenýho voka\/} se nachází na Žižkově.
V~letech '93 a '94 vyhrála cenu Hospoda roku. Dostanete se k~ní
autobusem 133 nebo 207 z~Florence. Vynikající Radegast, rozumné
jídlo, a ta atmosféra, mňam.

Další je oblíbená hospoda Bohumila Hrabala {\it U~Zlatého tygra}.
Prazdroj, dobré jídlo a na to, že je to v~centru (blízko Městské
knihovny směrem od řeky), jsou i rozumné ceny. Nepotkáte tam moc
turistů, ale o to více štamgastů. Jestli se chcete dobře usadit,
musíte přijít tak 15 až 30~minut před otvíračkou (15:00), ale
stojí to za to. Pokud ale nebudete pít dostatečně rychle, tak Vás
tam nebudou mít rádi.

Chcete-li jít někam na večeři a nezruinovat se, příjemná je
restaurace {\it Bruska}, kde občas hraje i živá hudba. Vystoupíte
na Hradčanské směrem na Bubenečskou ulici, za kolejemi zahnete
vlevo po Dejvické. Hostinec najdete po chvíli na pravé straně.

Netradiční zábavu v~podobě malůvek na stěně nabízí {\it Pivrncova
putyka}. Dostanete se k~ní, projdete-li ze Staroměstského náměstí
okolo kostela (je to kostel svatého Mikuláše jako na
Malostranském). Je v~nejbližší postranní uličce. Točí tam Pivrnce!

Jedete-li od Karlína, několik hospod je hned na Karlínském náměstí
(jedna zastávka od fakulty). Ve většině tamních hospod točí
Staropramen.