\subsection{MHD}

Jelikož budovy MFF UK jsou rozmístěny po polovině Prahy, je
zvládnutí MHD výhodou, která umožňuje ušetřit až hodinu denně.
Máme to štěstí, že pražská MHD patří k~nejlepším městským
hromadným dopravám na světě. i přesto, že k~dokonalosti
a~maximální časové efektivitě se dá dopracovat až několikaletou
praxí, nabízíme zde několik tipů, jak se zpočátku neztratit.


Kompletní jízdní řády a další informace o MHD jsou na webu
Dopravního podniku \url{http://www.dpp.cz/}

\subsubsection{Jízdné}

Protože matfyzáci při cestách do školy (a~večer z~jiných důvodů)
obvykle několikrát denně křižují Prahu, je téměř nezbytnou
pomůckou nějaký jízdní kupón, který je výhodnější než jednorázové
jízdenky. Dlouhou dobou jím byla tzv.~lítačka, kterou je stále
možné využívat, ale v~dohledné době ji má nahradit slavná
OpenCard, která je již dnes plnohodnotnou alternativou.

\subsubsubsection{Lítačka --- studentský kupón}

K~lítačce je nezbytně nutný univerzální studentský průkaz
(prvákům rozdávaný na Albeři, ISIC), který opravňuje lítačku
koupit a bez něj je lítačka neplatná. 

Kupón na měsíc stojí 260~Kč, čtvrtletní je za 720~korun.
Prodávají se Na~Bojišti --- centrála DPP a některých stanicích
metra, např. Nádrhol, I.$\,$P.$\,$Pavlova, Florenc, Můstek,
Kačerov nebo Karlovo náměstí. Na některých místech lze zaplatit
i~platební kartou, není to tedy samozřejmostí.

Dříve byly měsíční a čtvrtletní kupóny vázány ke kalendářním
měsícům. Od 13. června 2010 záleží jen na datu pořízení kupónu,
jsou tedy nyní flexibilnější.

Nárok koupit si lítačku se studentskou slevou máte celý rok,
tzn. i o prázdninách.


\subsubsubsection{OpenCard}

Slavná OpenCard měla od 31. prosince 2011 úplně nahradit obyčejné kupony. Nestalo se tak a důvod pořizování OpenCard pro studenty v~podstatě není. Pokud se Vám ale líbí nabíjení lítačky přes internet, budete ji potřebovat.

Pořízení OpenCard je o poznání složitější. Musíte navštívit jedno z~kontaktních míst DPP (Na Bojišti nebo vybrané stanice metra včetně Holešárny), kde odevzdáte vyplněnou žádost o vydání karty, souhlas se zpracováním osobních údajů a jednu průkazovou fotografii. Dále s~sebou mějte doklad totožnosti. Hotová OpenCard Vám poté přijde poštou.

Jedna z~příjemnějších možností jak zažádat o OpenCard je přes internet a poté si ji nechat zdarma poslat do některé z~knihoven (například jednu zastávku od Nádraží Holešovice, poblíž Ortenova náměstí se taková maličká knihovna vyskytuje).

Kompletní návod, jak získat OpenCard včetně potřebných formulářů, seznamu slev a dalších informací, naleznete na portálu \url{opencard.praha.eu}.

První nabití kupónu je nutné provést na kontaktním místě DPP s~doložením studentského průkazu. Další nabíjení je možné jak na kontaktních místech, tak i prostřednictvím internetu na stránkách e-shopu DPP. Ceny jsou stejné jako u lítačky, tedy 260 Kč 30-ti denní a 720 korun 90-ti denní. Používáte-li OpenCard déle než rok, je potřeba znova ukázat studentský průkaz na libovolném kontaktním místě.


\subsubsubsection{Obyčejné (jednorázové jízdenky) jízdenky}

Někdy se kupovat celý kupon nevyplatí nebo se jen zapomene před
Vánoci nakoupit a nebo prostě jen přijedou příbuzní z~Moravy
a~chtějí poradit, jaké lístky si mají koupit. Klíčové jsou 3
základní možnosti (pro dospělé osoby), z~nichž zdaleka nejjednodušší je SMS jízdenka.

\begin{itemize}

\item Přestupní jízdenka (cena 32 Kč) --- platí na všechny dopravní
prostředky MHD po dobu 90 minut.

\item Zlevněná jízdenka (cena 24 Kč) --- nepřestupní. Pro tramvaje
a~autobusy platí po dobu 30 minut, pro metro 30 minut, nepřestupně
a~maximálně 5 stanic.

\item SMS jízdenka (cena 32/24 Kč) --- SMS jízdenku je možné koupit jak na 32, tak na 24 korun. Pokud si nevšimnete letáků vyvěšených na každých dveřích MHD, algoritmus je následující: Na číslo {\bf 90206} pošlete zprávu {\bf DPT$x$}, kde $x\in\left\{24, 32\right\}$. SMS bude stát $x$ a tomu bude odpovídat délka platnosti jízdenky. Jízdenka Vám přijde jako SMS kód, který ukážete revizorovi.

\end{itemize}

\subsubsubsection{Revizoři}

Když Vás chytí revizor bez lístku, je pokuta 1000~Kč; pokud
zaplatíte na místě nebo ve lhůtě stanovené ve smluvních podmínkách
DPP, platí se jenom 800~Kč (patrně proto, aby lidé raději platili
a~nenechali se vymáhat soudně, což ovšem DP také umí). Pokud jste
si lítačku jenom zapomněli a přivezete ji ukázat Na Bojiště,
zaplatíte jenom 50~korun.

\subsubsection{Části MHD}
\subsubsubsection{Metro (krtek)}

Je páteří celého systému, jezdí od pěti ráno do půlnoci,
respektive trochu déle, protože kolem půlnoci vyjíždějí poslední
spoje z~konečných. Spousta času se ztrácí na jezdících schodech,
kde také platí jistá forma cestovatelské etiky --- v~pravé části
schodů se stojí, v~levé chodí. Až budete v~Praze déle, všimnete
si, že se cestování metrem dá optimalizovat --- tedy v~nástupní
stanici, kdy se na vlak čeká, se připravit na místo, kde je
v~cílové stanici výstup z~metra. Někteří matfyzáci systém ladí
k~dokonalosti, kdy se přesouvají v~rámci vlaku i na mezistanicích.


Metro je ideální pro přepravu na delší vzdálenosti. Pokud jedete
jenom kousek a ve stejném směru jede i tramvaj, je lepší jet
tramvají. V~mnoha případech je výhodnější jít i pěšky. Na~rozdíl
od jiných zahraničních měst (Londýn, New York, ...) je v~pražském
metru příjemně i při největších parnech nebo~mrazech.

Více informací o budování metra, předrevolučních názvech stanic,
futuristických vizích metra v~Praze za 100 let a mnoho dalšího
najdete na zajímavém webu \url{http://www.metroweb.cz/}.

\subsubsubsection{Tramvaj}

Byla páteří celého systému. Je ideální pro poskakování po městě.
Rychlost závisí především na dopravní situaci. Ve špičce nemůže
tramvaj v~některých úsecích kvůli autům ani projet
(Malostranská--Anděl). Každá linka má svou stálou trasu, ale skoro
pořád je někde výluka, takže se musí dávat pozor a sledovat
vývěsky, kde je v~naprosté většině včas napsáno, k~jakým čachrům
zase došlo. Výluky najdete na informačních tabulích na
nástupištích metra, na tramvajových a autobusových zastávkách ve
formě žlutých tabulek nad jízdními řády a v~tramvaji si občas
můžete vzít letáčky s~aktuální výlukovou situací (najdete je ve
schránkách náhodně rozmístěných po tramvaji, většinou je jedna
hned za kabinou řidiče). Pozor, někdy dojde k~výluce jen v~jednom
směru, to znamená, že člověk stojí na zastávce a když už
v~protisměru projíždí třetí tramvaj očekávané linky, je dobré se na
tabulku výluk podívat a změnit plán.

Jízdní řády nejsou víc jak pět zastávek od konečné příliš
směrodatné. Během dne je pohyb tramvají po městě zcela chaotický,
večer pak platí, že čím je zastávka dál od začátku, s~tím větším
předstihem na ni tramvaj přijede. Z~těchto důvodů je užitečné
chodit tak o pět minut dřív. Pokud jste od konečné zastávky dost
daleko, můžete mít štěstí, že jízdní řád opět začne platit,
protože předstih tramvaje dosáhne intervalu mezi jednotlivými
spoji.

Čísla tramvají jsou od jedničky do dvaceti šesti.

\subsubsubsection{Autobusy}

Nikdy nebyly páteří systému. Jezdí v~místech, kde nejezdí ani
tramvaj, ani metro, většinou paprskovitě ze stanic (hlavně
konečných) metra. Pokud se zrovna neopravuje silnice, je na jízdní
řád docela spolehnutí. V~autobusech se také vyskytují návaly
(např.~112 v~8.29 na Pelc-Tyrolce). Autobusy mají čísla 100--499
(s~velkými mezerami). Čím větší číslo autobus má, tím dále od
centra systému Vás doveze. Autobusové linky číselných řad 100
a~200 patří pod MHD a není nutné lítačku komukoli ukazovat, ale
i~v~nich se čas od času revizor vyskytne. Příměstské autobusy (linky
číselných řad 300 a 400) mají zabudovány v~pokladně u~řidiče
čtečky karet. Při nástupu je tedy nutné nejen lítačku (OpenCard)
ukázat řidiči, ale v~případě OpenCard navíc je potřeba přiložit
kartu na určené místo na boku pokladny, které je označeno symbolem
čipové karty, a řidič si tak zkontroluje platnost nahraných
kuponů.

\subsubsubsection{Pantografy neboli žabotlamy}

Jsou to povětšinou ještě modrošedá monstra snažící se zajistit
příměstskou vlakovou dopravu v~okolí Prahy. i když jsou v~poslední
době vytlačovány novými soupravami City Elefant, ještě nějakou
dobu je budou České dráhy nasazovat. Jelikož už nějakou dobu
v~Praze funguje tzv.~{\it Integrovaný dopravní systém}, mělo by
být celkem jedno, čím pojedete, zkrátka lítačka (OpenCard) platí
na metro, tramvaje, městské autobusy, osobní a spěšné vlaky
(a~mizivé množství rychlíků), lanové dráhy a i přívozy na území
Prahy.

Pro matfyzáky je zajímavá trať z~Hlavního nádraží do Benešova
(v~katalozích ČD má číslo 221). Lze ji použít při cestování do
Hostivaře. Výhodou je, že zatímco metrem a autobusy se
k~Hostivařskému nádraží mlátíte hodinu, vlak tuto trať urazí za
čtvrt hodiny (pokud se po cestě nerozbije). Nevýhodou je, že
takovéto monstrum vyjíždí z~Hlavního nádraží nejvýše jedno za půl
hodiny (intervaly se mění podle denní doby, v~nočních a poledních
hodinách dosáhnou i devadesáti minut). Na vlacích neplatí
ustanovení pražských dopravních podniků o pokutách, jinými slovy
sleva pro sklerotiky se neposkytuje; zapomenete-li si lítačku, tak
pokud to zpozorujete včas a přiznáte se průvodčímu, zaplatíte
jízdné Českých drah s~příplatkem 30 korun; pokud Vás bude muset
průvodčí nebo revizor odhalit sám, je pokuta 800 korun, platí se
ale jinde (při zaplacení na místě se snižuje na 400 korun).

\subsubsubsection{Lodní doprava}

Kromě nákladních lodí a výletních parníků pro turisty brázdí vody
Vltavy i několik přívozů spadajících pod MHD. (Např. z~Podhoří do
Podbaby, který jde výhodně použít k~dopravě do Suchdolu, kde má
sídlo hnojárna --- Česká zemědělská univerzita.) Přívozy nejsou
v~provozu přes zimu a nevyplují ani v~případě zvýšené povodňové
aktivity. Pro zajímavost, když byla z~důvodů rekonstrukce
vyšehradského tunelu přerušena tramvajová doprava po břehu řeky,
byla zavedena náhradní kyvadlová lodní doprava.

\subsubsubsection{Noční tramvaje a autobusy --- sběrače mrtvol}

ESM --- elektrický sběrač mrtvol --- tramvaj --- jezdí od~23.45
do~5.00 (přibližně) a to ve stejných kolejích, ale po jiných
trasách než normální tramvaje. Všechny ESM se společně sjíždějí na
stanici {\bf Lazarská} (u~Spálené ulice kousek od Národní třídy),
kde je vždy čas na~přestup i pro ty největší mrtvoly. Interval
mezi tramvajemi je 30~minut, takže je třeba dávat pozor, kde Vás
ten sběrač vyklopí. Mnohým mrtvolám se stává, že dojedou na úplně
opačný konec Prahy, kde je ještě spánkem zpitomělé řidič vyhodí
z~tramvaje a nechá mrznout. Síť ESM podporovaná sítí MSM
(motorového sběrače mrtvol --- nočních autobusů), pro něž platí
trošku jiná pravidla (např.  se nepotkávají a některé z~nich jezdí
v~šedesátiminutových intervalech), obstojně kopíruje trasu metra~C
a~jezdí během noční přestávky metra. Vzhledem k~časové návaznosti
a~dobré noční propustnosti ulic lze tomuto způsobu dopravy přijít
na mnoha trasách na chuť. Například noční transfer z~Jižáku na
kolej 17. listopadu je v~noci mnohem rychlejší než přes den.
Vhodnou stanicí, jak se dostat v~noci z~centra, je I. P. Pavlova,
kde staví noční autobusy jedoucí jak na Kuchyňku, tak na Volhu.

Čísla ESM jsou od 50 do 60, MSM přes 500.

\subsubsection{Specifika pro Kolej 17. listopadu}
\medskip
\subsubsubsection{Přístupové cesty na kolej}

Nejdůležitějším bodem, odkud se cestuje na kolej, je stanice metra
Nádraží Holešovice. Odtud se na kolej buď jde pěšky přes most,
nebo jede autobusem. Cesta pěšky trvá dvanáct minut, ale pokud
\emph{opravdu\/} spěcháte, můžete to uběhnout i za pět minut.
Autobusem se jezdí buď linkou~112 na zastávku \emph{Pelc-Tyrolka},
která je koleji nejblíže, nebo na Kuchyňku autobusy s~čísly 102
a~186. Až se budete rozhodovat, jestli jet na Kuchyňku, nebo na
Pelc-Tyrolku, vězte, že průměrnému matfyzákovi trvá cesta z~Kuchyňky
o~3--4~minuty déle než cesta z~Pelc-Tyrolky.

Při cestách od Výstaviště (tedy např. ze Štrossmajeráku nebo
Právnické fakulty), se s~výhodou používá tramvaj~14 nebo~17 do
zastávky Trojská (z~koleje: po silnici k~ZOO, až narazíte na
tramvajové koleje). Cesta na tramvaj trvá asi osm minut nebo se dá
využívat stodvanáctka ve směru k~ZOO na stanici Povltavská. Cestou z~Povltavské na kolej občas potkáte veselé bezdomovce.

Každý matfyzák dříve nebo pozdeji objeví, že tramvajové zastávky s~názvem Nádraží Holešovice jsou dvě. A to tehdy, když zrovna vystoupí na té špatné a zmateně se rozhlíží, kde to je. Do odlehlejší zastávky jezdí tramvaje 14 a 17. Buď můžete tramvajemi pokračovat až do zastávky Trojská a přesedlat na autobus 112, anebo se pořádně rozhlédnout, dokud nespatříte budovu nádraží či červenou ceduli označující metro a projít buď tunelem nebo perónem metra na obvyklou zastávku.

\subsubsection {Stodvanáctka}

Je náš kolejní autobus. Jezdí do ZOO. Většinou na ní nejsou
návaly, ale u~některých spojů se z~ní stává induktivní autobus
(když už se tam vešlo $n$ lidí, vejde se jich tam i $n+1$).
Nevejdou se pouze ti, kteří vyměknou. Problémy jsou hlavně, když
se hodně jezdí do Troje za zvířaty. Velice poučné je v~období
školních výletů sledovat marné snahy učitelek na~Holešárně narvat
do stodvanáctky celou třídu najednou. Posledních pár let zavedla
pražská ZOO v~teplých měsících dopravu z~Holešárny do ZOO za
korunu (dříve zdarma) a vyjíždí tzv. Zoobus, který staví až
v~cílové stanici. Bohužel většina cestujících se stejně natlačí do
stodvanáctky, která jede o minutu dříve, a za ní jede Zoobus
prázdný.

\subsubsection{Ideální algoritmy MHD}

\def\startpath{B 112, M C (Nádrhol \ra }
\def\ra{$\rightarrow$}

V~následujících doporučených cestách pomocí MHD je použito toto
značení:

\smallskip

\noindent\begin{tabularx}{\textwidth}{ l|X }
     M C (X $\rightarrow$ Y) & metro,  trasa C, z~X do Y \\ \hline
     T 8, 24 (X $\rightarrow$ Y)/2 & tramvaje číslo 8 a 24 z~X do Y a jsou to dvě zastávky \\ \hline
      B 112 &  autobus číslo 112 (zbytek jako u~tramvaje), pokud je
      bez dalšího, znamená B 112 (Pelc-Tyrolka $\rightarrow$ Nádrhol)/2, \\ \hline
      V~221 (X $\rightarrow$ Y) & vlak na trati 221 z~X do Y \\     
\end{tabularx}
\subsubsubsection{Cesty od koleje 17. listopadu k~budovám fakulty}

\bigskip
\textbf{Troja (fyzika)} Pěšky po chodníku směrem k~mostu, projít
pod mostem do~nízké modrošedé budovy spojené s~věžákem stejné
barvy --- momentálně rekonstruovaným.

\textbf{Troja (angličtina)} Pěšky po chodníku směrem k~mostu, před
ním odbočit doleva a po asi dvaceti metrech doprava, projít pod
mostem a rovně do~nízké budovy nespojené s~věžákem stejné barvy.

\textbf{Karlín} 
\startpath Florenc), T~8, 24 (Florenc \ra
Křižíkova)/2, přejít ulici, budova naproti nebo i pěšky
z~Florence.

\textbf{Karlov} 
\startpath Pavlák), vylézt směr ulice Na Bojišti,
Karlov, přejít magistrálu, projít kousek po její pravé straně
nahoru směrem k~Vyšehradu, zahnout doprava (ulice Na Bojišti), na
konci doleva, pořád rovně, poslední dvě budovy na pravé straně
jsou Ke~Karlovu~{5~a~3} (v~tomto pořadí).

\textbf{Malá Strana} 
B~112, T~12 (Nádrhol \ra Malostranské
nám.)/8, nahoru přes parkoviště, ta nádherná bílá budova je Matfyz.
Tento algoritmus je optimální s.$\,$v. (skoro vždy), ale má-li
T~12 výluku, platí:

\startpath Muzeum), M~A (Muzeum \ra
Malostranská), T~12, 22, 23 (Malostranská \ra Malostranské
náměstí)/1. Mnohdy je ale doprava kolem Malostranské zkolabovaná,
tudíž je mnohdy lepší jít z~Malostranské (častěji na
Malostranskou) pěšky. Metro se doporučuje také v~zimě, když je
doprava omezená sněhem (pak může cesta tramvají trvat i přes
hodinu).

\textbf {Hostivař} 
\startpath Hlavák), V~221 (Hlavák \ra
Hostivař), pěšky po silnici podél trati směrem na Benešov, dokud
zpoza paneláků nevykoukne SCUK,

Nebo:  \startpath Muzeum), M~A (Muzeum \ra Skalka), B~154, 271
(Skalka \ra
Gercenova), %ono to totiž je: B154(Skalka \ra Gercenova)/8 nebo B271(Skalka \ra Gercenova)/7.
příčnou ulicí (to je Gercenova), vlevo kolem nákupního střediska.

\subsubsubsection{Cesty mezi jednotlivými budovami fakulty}
\bigskip

\textbf {Karlov --- Karlín}
 pěšky na Pavlák, M~C (Pavlák \ra
Florenc), T~8, 24 (Florenc \ra Křižíkova)/2, přejít ulici a vejít
dovnitř.

\textbf{Troja --- kamkoliv}
 jako z~koleje nebo zastávky Kuchyňka
čímkoliv směr Holešovice.

\textbf{Karlov --- Malá Strana}
 pěšky celou ulicí Ke~Karlovu až
dojdete k~tramvajovým kolejím; tam je zastávka Štěpánská, T~22, 23
(Štěpánská \ra Malostranské náměstí)/6. Bojíte-li se jít na
Štěpánskou, tak pěšky na Pavlák a T~22, 23 (Pavlák \ra
Malostranské náměstí)/7. Hlavně na Pavláku nepodlehněte nacvičené
cestě do metra.

\textbf{Karlín --- Malá Strana}
 M~B (Křižíkova \ra Národní třída),
T~22, 23 (Národní třída \ra Malostranské náměstí)/4.