\section{Recepty}
Když už se příručka, kterou právě držíte v rukou nebo čtete její elektronickou verzi, jmenuje Kuchařka, máme zde pro vás několik jednoduchých receptů. Jedná se o jídla, která je dobré si uvařit po příchodu ze školy, když máte hlad. Jsou snadná a rychlá na přípravu.

\subsubsection{Buřtguláš}
Na tři až čtyři porce budeme potřebovat 1 větší cibuli, 200g točeného salámu/2 párky, 6 brambor, trochu oleje, 4 lžíce hladké mouky, 3-4 lžíce červené sladké papriky (koření), kmín, pepř/nové koření, bobkový list (3ks), sůl.

Postup přípravy: Nadrobno si nakrájíme cibuli, vložíme do hrnce na rozpálený olej a zpěníme (chvíli smažíme dozlatova, ne dohněda), poté přisypeme červenou papriku a hladkou mouku, zamícháme a zalijeme studenou! vodou (po horké by zhrudkovatělo). Vložíme oloupané brambory nakrájené na menší kousky, přidáme nové koření, bobkový list, kmín a sůl a vaříme, dokud nejsou brambory uvařené. Poté přidáme nakrájený točený salám, majoránku, dochutíme dle potřeby a vaříme ještě tak 2 minuty. Pokud se nám zdá guláš málo hustý, můžeme vlít mouku rozmíchanou s vodou a provařit. Jíme s chlebem, starší chleba lze nalámat přímo do guláše.


\subsubsection{Česnečka}
Česnečka je jedna z nejméně pracných polévek, které si můžete ukuchtit. Na jeden hrnec polévky budeme potřebovat 5 stroužků česneku, 3 brambory, 2 vajíčka, tvrdý sýr (50 - 100 g) plátky/kousky/strouhaný, 1 kostku masoxu, sůl, kmín (cca půl čajové lžičky). Množství si ale můžete libovolně upravit podle toho, jak hustou a ochucenou polévku chcete mít. Správná česnečka se dělá z vývaru z uzeného masa, masox ale pomůže vytvořit dobrou chuť i bez vývaru.

Postup přípravy: Oloupeme brambory a česnek, brambory nakrájíme na menší kousky, česnek rozmělníme a obojí vložíme do hrnce s vodou (pokud si chceme ušetřit čas, uvaříme si vodu nejprve ve varné konvici). Přisypeme kmín, můžeme přidat trošku soli, ale ne moc, přikryjeme pokličkou a čekáme, až se brambory uvaří (jsou měkké). Poté přidáme kostku masoxu, před vložením ji trochu rozdrolíme, rozklepneme vajíčka a zamícháme. Asi minutu vaříme a dosolíme podle chuti. Přidáme sýr do hrnce nebo do talíře a podáváme. Můžeme si také připravit osmažené kousky chleba na vhození do polévky.

Pozn: Vajíčka je také možno dát naspod talíře (místo do polévky) a zalít horkou polévkou. Při vaření brambor nemusíte stát u plotny, stačí občas pohlídat nebo ztlumit plotnu, aby polévka neutekla.

\subsubsection{Chleba ve vajíčku}
Ideální, rychlá večeře, která navíc pomůže zbavit se snadno a chutně starého pečiva. Na základní variantu budeme potřebovat chleba, dokonce je lepší starý, vajíčka (podle množství chleba), sůl, trocha tuku na smažení (olej/sádlo/...).

Postup přípravy: Do misky nebo hlubokého talíře si rozklepneme vajíčko, přidáme špetku soli a trochu vody/mléka, vše rozmícháme (dobře se rozkvedlává vidličkou). Poté vezmeme plátky chleba a namáčíme je ve vajíčku. Chleba obalený vajíčkovou směsí dáme na rozpálenou pánvičku a smažíme, dokud není vajíčko uvařené. Chleba lze obalovat i smažit z jedné nebo z obou stran, do směsi můžeme přidat plno dalších věcí na ochucení - pepř, česnek a majoránku, strouhaný sýr, plátky šunky... Hotový chleba někdo potírá hořčicí nebo kečupem, sype pórkem nebo cibulí. Možností je plno.

\subsubsection{Kuskus}
Kuskus je jeden z pokrmů, které stačí zalít vařící vodou a jsou téměř hotové. Lze ho ochutit jak na sladko (Granko nebo kompot), tak z něj připravit slané jídlo (zamícháním mražené zeleniny/kousků masa/konzervy tuňáka/sýra/čehokoliv, co nám zbylo). S trochou fantazie se hodí i jako příloha k omáčce.

\subsubsection{Pasta pomodoro}
Tento recept je vylepšením špaget s kečupem, používá minimum ne úplně trvanlivých potravin, takže se hodí i jako záchrana, pokud jste si zapomněli nebo nestihli nakoupit.

Budeme potřebovat: těstoviny, rajčatový protlak, cibule, (česnek), sýr (nejlépe nějaký parmazánovitý - dají se koupit různě malé různě levné pytlíčky), bazalka/oregano a sůl.

Postup přípravy: Uvaříme těstoviny v horké vodě, vedle si dáme osmažit cibulku, poté přidáme česnek, zalijeme protlakem a zasypeme oreganem.

Pozn: Lepší je z čerstvých rajčat, ale je snadné mít na koleji protlaky v plechovce, nezabírají místo v lednici a jsou trvanlivé.

\subsubsection{Rýže s ...}
Toto je klasika mezi kolejními jídly. Uvaříme rýži a k ní si dáme, co najdeme. Osvědčené je osmažit si cibulku a k ní přidat mraženou zeleninu. Ochutíme solí, pepřem a majoránkou, vmícháme do rýže a můžeme večeřet. Místo mražené zeleniny můžeme použít i čerstvou, případně kombinovat. Osvědčené přísady jsou také lilek a paprika. Pro pestrost můžeme zkoušet různá koření a jejich kombinace, například: chilli, tymián, curry, bazalka, hořčice. Do zeleniny můžeme přidat ještě nakrájený párek či jiné maso a na hotový pokrm nastrouhat sýr. Všeobecně ale platí, že do rýže lze zamíchat všemožné zbytky a vytvořit dobré jídlo.

\subsubsection{Tipy a rady}
Následující recepty a návody jsou spíše pro krizové situace anebo pro opravdové gurmány. Máte špagety a nemáte hrnec? Máte vajíčka a nemáte hrnec?

\subsubsection{Vajíčko natvrdo v mikrovlnce}
Mysleli jste si, že vaření vajíčka v mikrovlnce je jen vtip? Znáte někoho, komu vajíčko celou mikrovlnku vymalovalo? Chcete být lepší než oni? Pojďme se společně podívat na jednoduchý recept.

Do hrnečku (nejlépe menšího, protože vajíčka nejsou většinou příliš velká) přidáme pár kapek oleje (pár jsou dvě $\pm$ pár), potom do něj rozklepneme vajíčko, můžeme lehce osolit a vložíme do mikrovlnky na 45 s $\pm$ 10 s podle výkonu, požadované konzistence a aktuální nálady strávníka. Jíme lžičkou přímo z hrnečku.

\subsubsection{Špagety v konvici}
Špagety se dají uvařit i ve varné konvici, chce to jenom trochu trpělivosti. Napustíme konvici vodou a vnoříme špagety tak, aby v konvici zbývala ještě alespoň polovina místa. Zapneme konvici a jak se špagety začínají změkčovat, potápíme je a do vody nacpeme i zatím nezměkčené konce, které doposud vykukovaly ven. Poté vaříme, dokud nejsou špagety měkké.

\subsubsection{Kulinářské rady}
smažení - pokud cokoli smažíme, ať na oleji nebo sádle, musí být tuk rozehřátý, pak ho smažené věci nenasají tolik
masová konzerva - hodí se jako rychlovka do rizota, těstovin...; dobrou masovou konzervu za rozumnou cenu mají v Lidlu
sladké pokrmy (buchty, palačinky...) - do téměř všech sladkých jídel se hodí pro zvýraznění chuti přidat špetku soli
rajská omáčka - nevěřili byste, jak výbornou chuť dodá trocha skořice
1 lžíce mouky/cukru je cca 25 g
masový vývar - maso vkládáme do studené vody, pustí více šťávy
loupání rajčat - rajčata jdou snáze oloupat, přelijeme-li je horkou vodou

\subsubsection{Menza}
Když všechny jiné postupy selžou, můžete se jít o všedních dnech naobědvat do Menzy :)

\subsubsection{Návod na vyčištění varné konvice}
Tohoto receptu se sice nenajíte, ale jelikož je naše kolejní voda vydatná na minerály, určitě jej doceníte. Zhruba po měsíci používání varné konvice zjistíte, že se vám na dně usadila vrstva železa a po půl roce už bude železa v konvici tolik, že už s ní budete opravdu muset něco udělat. Pokud ji neodevzdáte do starého železa a za utržené peníze si nekoupíte konvici novou, budete ji muset vymýt.

Konvici vyčistíme jednoduše. Nalijeme do ní vodu a ocet v poměru 1:1, dáme vařit. Potom necháme ocet asi hodinu působit a následně konvici vymyjeme čistou vodou. Může se stát, že postup bude potřeba opakovat. Pokud chceme něco silnějšího než ocet, nasypeme do teplé vody kyselinu citrónovou a necháme působit.