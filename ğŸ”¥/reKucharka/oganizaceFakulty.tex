\subsection{Organizace fakulty}
Fakulta je dost složitý organismus, takže její strukturu popíšeme pouze stručně.

\subsubsection{Univerzita}
Celá fakulta pak patří pod Univerzitu Karlovu, o které jste už asi někdy slyšeli. Kromě nás tam patří filosofové, právníci, biologové a další nematfyzáci. Vedení univerzity se říká {\it rektorát}, v~jeho čele je rektor (teď prof. Václav Hampl z~Pří\-ro\-do\-vě\-de\-cké fakulty). V podstatě s~rektorátem nikdy nepřijdete přímo do styku, kromě velmi, velmi vyjímečných událostí.

\subsubsection{Vedení}
V~čele fakulty stojí {\it děkan}, který se snaží celou fakultu ukočírovat. Samozřejmě, že to nemůže zvládnout sám, a proto má každou oblast života fakulty na starosti příslušný {\it proděkan} (celkem jich je sedm). Děkan, proděkani a tajemník {\it fakulty} (ten má na starosti hospodaření) tvoří {\it vedení fakulty}.

Děkan se letos měnil; do minulého školního roku jím byl prof. Zdeněk Němeček, ve značně napínavé volbě byl zvolen děkan nový --- takže od září 2012 do roku 2016 je děkanem prof. Jan Kratochvíl.

{\it Vědecká rada}, složená z~největších fakultních a externích odborníků schvaluje studijní plány, a mimo jiné projednává nových docentů a profesorů.

\subsubsection{Senát}
Dalším klíčovým orgánem fakulty je {\it akademický senát (AS)}, jež má 24 členů --- z~toho 16 členů tvoří {\it zaměstnaneckou komoru (ZKAS)} a 8 členů {\it studentskou komoru (SKAS)}, o ní víc dále. V čele stojí předsednictvo tvořené předsedou, dvěma místopředsedy a jednatelem. Senát má značný vliv na většinu podstatných fakultních záležitostí --- volí a odvolává děkana, přijímá vnitřní předpisy a jeho souhlas je potřeba při sestavování fakultního rozpočtu, zřizování a rušení kateder a jmenování členů vědecké rady.

Studenti Matfyzu dále každé tři roky volí své dva zástupce do {\it Akademického senátu Univerzity Karlovy (AS UK)}. Tento takzvaný velký senát je víceméně obdobou fakultního senátu, ale s~působností v~rámci celé univerzity (tedy např. volí rektora nebo přijímá vnitřní předpisy UK, které platí na všech fakultách).

\subsubsection{Sekce}


MFF UK se dělí na tři {\it sekce}, a to na {\it Matematickou}, {\it Fyzikální} a {\it Informatickou}, každá sekce má svého proděkana.  Sekce se skládají z~jednotlivých {\it kateder}. Mimo tyto sekce stojí tzv. Ostatní: {\it Kabinet jazykové přípravy} a {\it Katedra tělesné výchovy} (která je od jisté doby mezi Matfyzáky pře\-kva\-pi\-vě populární).  Dále jsou na fakultě samostatná {\it výzkumná centra}, {\it účelová zařízení} (např. reprografické středisko v~Karlíně), {\it knihovna fakulty} a pochopitelně je tu {\it děkanát}, ve fakultní hantýrce nazývaný {\it uzel D}, který se skládá ze spousty oddělení, o kterých běžný Matfyzák vůbec neví a ani vědět nepotřebuje. Čestnou výjimkou je pochopitelně oddělení studijní, v~jehož čele stojí {\it proděkan pro studijní záležitosti} (prof. František Chmelík).
