\subsection{Slovníček pro Nepražáky}
\begin{itemize}
\item \textbf{Arnošt} - menza Arnošta z Pardubic
\item \textbf{Blanka} - tunel se zběsile měnícími se semafory, který od září 2015 pravidelně ucpává prostor okolo Koleje 17. listopadu
\item \textbf{Hlavák} - Hlavní nádraží (C); též Wilsoňák
\item \textbf{Karlák} - Karlovo náměstí (B), !!!TOTO NENÍ KARLŮV MOST!!!
\item \textbf{Kulaťák} - Vítězné náměstí, Dejvická (A)
\item \textbf{Lítačka} - tramvajenka
\item \textbf{Masaryčka} - Masarykovo nádraží; některými zvrhlíky zváno Masna, pamětníky zváno Střed
\item \textbf{Máj} - OD MY (Tesco) na Národní třídě
\item \textbf{Mírák} - Náměstí Míru (A)
\item \textbf{MS} - Malá Strana
\item \textbf{Nádrhol} - Nádraží Holešovice (C); někdy Holešárna, Holešky, nebo jen Holešovice
\item \textbf{Národka} - Národní třída (B)
\item \textbf{Nároďák} - Národní divadlo nebo národní tým
\item \textbf{Opletalka} - menza Jednota
\item \textbf{Palačák} - Palackého náměstí, vedou sem jihozápadní výstupy z metra Karlovo náměstí (B)
\item \textbf{Pavlák, Ípák, Slinták} - I. P. Pavlova (C), informatici občas vyslovují [aj pí]
\item \textbf{pod ocasem, pod koněm} - u sochy sv. Václava na Václaváku, klasicky používaná fráze: "Sejdeme se pod ocasem."
\item \textbf{Smícháč} - Smíchovské nádraží (B)
\item \textbf{Staromák} - Staroměstské náměstí - to s orlojem
\item \textbf{Šervůd} - park před Hlavákem obydlený bezdomovci
\item \textbf{Štrosmajerák}, Štros - Strossmayerovo náměstí
\item \textbf{Václavák} - Václavské náměstí - to s koněm
\end{itemize}
