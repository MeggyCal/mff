\subsection{Jízdné}

Protože \mfz{}áci při cestách do školy (a~večer z~jiných důvodů)
obvykle několikrát denně křižují Prahu, je téměř nezbytnou
pomůckou nějaký jízdní kupón, který je výhodnější než jednorázové
jízdenky. Dlouhou dobou jím byla tzv.~lítačka, kterou je stále
možné využívat, ale v~dohledné době ji má nahradit slavná
OpenCard, která je již dnes plnohodnotnou alternativou.

\subsubsection{Lítačka --- studentský kupón}

K~lítačce je nezbytně nutný univerzální studentský průkaz
(prvákům rozdávaný na Albeři, ISIC), který opravňuje lítačku
koupit a bez něj je lítačka neplatná. 

Kupón na měsíc stojí 260~Kč, čtvrtletní je za 720~korun.
Prodávají se Na~Bojišti --- centrála DPP a některých stanicích
metra, např. Nádrhol, I.$\,$P.$\,$Pavlova, Florenc, Můstek,
Kačerov nebo Karlovo náměstí. Na některých místech lze zaplatit
i~platební kartou, není to tedy samozřejmostí.

Dříve byly měsíční a čtvrtletní kupóny vázány ke kalendářním
měsícům. Od 13. června 2010 záleží jen na datu pořízení kupónu,
jsou tedy nyní flexibilnější.

Nárok koupit si lítačku se studentskou slevou máte celý rok,
tzn. i o prázdninách.


\subsubsection{OpenCard}

Slavná OpenCard měla od 31. prosince 2011 úplně nahradit obyčejné kupony. Nestalo se tak a důvod pořizování OpenCard pro studenty v~podstatě není. Pokud se Vám ale líbí nabíjení lítačky přes internet, budete ji potřebovat.

Pořízení OpenCard je o poznání složitější. Musíte navštívit jedno z~kontaktních míst DPP (Na Bojišti nebo vybrané stanice metra včetně Holešárny), kde odevzdáte vyplněnou žádost o vydání karty, souhlas se zpracováním osobních údajů a jednu průkazovou fotografii. Dále s~sebou mějte doklad totožnosti. Hotová OpenCard Vám poté přijde poštou.

Jedna z~příjemnějších možností jak zažádat o OpenCard je přes internet a poté si ji nechat zdarma poslat do některé z~knihoven (například jednu zastávku od Nádraží Holešovice, poblíž Ortenova náměstí se taková maličká knihovna vyskytuje).

Kompletní návod, jak získat OpenCard včetně potřebných formulářů, seznamu slev a dalších informací, naleznete na portálu \url{opencard.praha.eu}.

První nabití kupónu je nutné provést na kontaktním místě DPP s~doložením studentského průkazu. Další nabíjení je možné jak na kontaktních místech, tak i prostřednictvím internetu na stránkách e-shopu DPP. Ceny jsou stejné jako u lítačky, tedy 260 Kč 30-ti denní a 720 korun 90-ti denní. Používáte-li OpenCard déle než rok, je potřeba znova ukázat studentský průkaz na libovolném kontaktním místě.


\subsubsection{Obyčejné (jednorázové jízdenky) jízdenky}

Někdy se kupovat celý kupon nevyplatí nebo se jen zapomene před
Vánoci nakoupit a nebo prostě jen přijedou příbuzní z~Moravy
a~chtějí poradit, jaké lístky si mají koupit. Klíčové jsou 3
základní možnosti (pro dospělé osoby), z~nichž zdaleka nejjednodušší je SMS jízdenka.

\begin{itemize}

\item Přestupní jízdenka (cena 32 Kč) --- platí na všechny dopravní
prostředky MHD po dobu 90 minut.

\item Zlevněná jízdenka (cena 24 Kč) --- nepřestupní. Pro tramvaje
a~autobusy platí po dobu 30 minut, pro metro 30 minut, nepřestupně
a~maximálně 5 stanic.

\item SMS jízdenka (cena 32/24 Kč) --- SMS jízdenku je možné koupit jak na 32, tak na 24 korun. Pokud si nevšimnete letáků vyvěšených na každých dveřích MHD, algoritmus je následující: Na číslo {\bf 90206} pošlete zprávu {\bf DPT$x$}, kde $x\in\left\{24, 32\right\}$. SMS bude stát $x$ a tomu bude odpovídat délka platnosti jízdenky. Jízdenka Vám přijde jako SMS kód, který ukážete revizorovi.

\end{itemize}

\subsubsection{Revizoři}

Když Vás chytí revizor bez lístku, je pokuta 1000~Kč; pokud
zaplatíte na místě nebo ve lhůtě stanovené ve smluvních podmínkách
DPP, platí se jenom 800~Kč (patrně proto, aby lidé raději platili
a~nenechali se vymáhat soudně, což ovšem DP také umí). Pokud jste
si lítačku jenom zapomněli a přivezete ji ukázat Na Bojiště,
zaplatíte jenom 50~korun.