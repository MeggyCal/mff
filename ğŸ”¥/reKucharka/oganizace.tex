\subsection{Organizace}
Fakulta je dost složitý organismus,
takže její strukturu popíšeme pouze stručně.


\subsubsection{Univerzita Karlova}
I když si to mnoho lidí neuvědomuje, Matfyz patří pod Univerzitu Karlovu, o
které jste asi už někdy slyšeli. Kromě nás tam patří filosofové, právníci,
biologové a další nematfyzáci. Vedení univerzity se říká rektorát, v jeho čele
je rektor (od února 2014 prof. Tomáš Zima z 1. lékářské fakulty) a jeho tým
prorektorů. S rektorátem prakticky nepřijdete přímo do styku, kromě velmi,
velmi výjimečných událostí.


\subsubsection{Matematicko-fyzikální fakulta}
V čele fakulty stojí děkan, který se ji snaží ukočírovat. Samozřejmě to nemůže zvládnout sám, a proto má k dispozici kolegium děkana - do něj patří například osm proděkanů (každý má na starosti některou oblast života fakulty) a tajemník fakulty, který se stará o hospodaření. Děkan a kolegium tvoří vedení fakulty. Od září 2012 do roku 2020 je děkanem prof. Jan Kratochvíl.

Vědecká rada, složená z největších fakultních a externích odborníků, mimo jiné schvaluje studijní plány a projednává jmenování nových docentů a profesorů.


\subsubsection{Senát}
Dalším klíčovým orgánem fakulty je Akademický senát MFF UK (AS), jenž má 25 členů – z toho 16 členů tvoří zaměstnaneckou komoru (ZKAS) a 9 členů studentskou komoru (SKAS). V čele stojí předsednictvo tvořené předsedou, dvěma místopředsedy a jednatelem. Senát má značný vliv na většinu podstatných fakultních záležitostí – volí a odvolává děkana, přijímá vnitřní předpisy a jeho souhlas je potřeba při sestavování fakultního rozpočtu, zřizování a rušení kateder a jmenování členů vědecké rady. Na konci každého akademického roku studenti volí do SKASu tři zástupce na tříleté funkční období. Jednání senátu i zápisy z jednání jsou veřejně přístupné.

Studenti Matfyzu dále každé tři roky volí své dva zástupce do Akademického senátu Univerzity Karlovy (AS UK). Tento takzvaný velký senát je víceméně obdobou fakultního senátu, ale s působností v rámci celé univerzity (tedy např. volí rektora nebo přijímá vnitřní předpisy UK, které platí pro všechny fakulty).


\subsubsection{Sekce}
MFF UK se dělí na tři sekce, a to na matematickou, fyzikální a informatickou. Sekce se skládají z jednotlivých kateder nebo ústavů a každá má vlastního proděkana. Mimo tyto sekce stojí Katedra jazykové přípravy a Katedra tělesné výchovy (která je od jisté doby mezi matfyzáky překvapivě populární). Dále jsou na fakultě tzv. účelová zařízení (např. knihovna fakulty nebo reprografické středisko v Karlíně) a pochopitelně děkanát, který se skládá ze spousty oddělení, o kterých běžný matfyzák vůbec neví a ani vědět nepotřebuje; až na čestnou výjimkou, kterou je oddělení studijní, v jehož čele stojí proděkan pro studijní záležitosti (doc. František Chmelík), na kterého se nebojte případně obrátit.