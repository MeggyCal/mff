\subsection{Volný čas}
Praha nabízí plno možností, co dělat ve volném čase. Najdete tu rozmanité obchody, všemožná sportoviště, divadla od amatérských až po Národní, kina, hospody, cukrárny a téměř vše, na co si vzpomenete. Nejčastěji vám pomůže najít, co hledáte, strýček Google, naším cílem proto není vyjmenovat vše od A do Z, ale podat pár zažitých a ozkoušených tipů, často z okolí Koleje 17. listopadu.

\subsubsection{Kam vyrazit v okolí koleje}
Chcete si odpočinout nebo zajít na procházku? Občas se najde chvilka, kdy máte chuť vyrazit do přírody, posedět v parku, sednout si s učením ven nebo si udělat kratší procházku. V okolí kolejí se nachází pár míst, která stojí za zmínění.

Stromovka – rozsáhlý park, který sahá od Výstaviště až dolů k řece a do šířky k Bubenči. Najdete zde chládek stromů i několik rybníčků.
Zoo Praha – vzdálená 3,5 km, k hlavnímu vstupu se dá dostat jak pěšky podél řeky, tak autobusem 112. Najdete zde mj. několik nových pěkně upravených výběhů. Prohlédnout si celou zoo zabere téměř celý den, počítejte také s větším počtem návštěvníků. Velká sleva pro studenty bývá v pondělí.
Botanická zahrada Praha – nachází se kousek nad zoo, studentské denní vstupné vyjde na 45 Kč (s Lítačkou 43 Kč), roční permanentka pak 300 Kč, v zimě je vstup zadarmo. V létě je zde příjemný chládek a dobře se učí.
Jen tak se projít můžete podél pravého (bližšího) břehu řeky na obě strany.


\subsubsection{Sportování v okolí koleje}
Přímo na kolejích se v suterénu nachází aerobiková místnost a místnost s pinpongovým stolem. V čase vzniku tohoto textu byly tyto místnosti uzavřeny kvůli probíhající stavbě na budově C, ale očekává se jejich opětovné otevření. Zapůjčení (až bude umožněno) je zdarma oproti kolejence. Také se v suterénu nachází posilovna, kam se platí symbolické roční vstupné. Na východ od kolejí je travnatá plocha, na které se dá např. hrát frisbee, ale místo které má vzniknout minipark a víceúčelové hřiště. Na jižní straně jsou před kolejí umístěna bradla a hrazda, které časem taky zmizí a nahradí je propojení mezi ulicemi Povltavská a Pátkova. Některé sportovní vybavení lze půjčit na vrátnici B. Novinky ohledně sportovního vybavení na koleji pak píše kolejní rada na svých stránkách.

Dále za zmínku stojí velmi pěkná cyklostezka podél řeky. Začíná přímo před kolejemi směrem na západ a vede přes zoo až do Klecánek (dále se dá pokračovat třeba do Kralup nad Vltavou, ale kvalita je už horší). Zde se dá dobře jezdit na inline bruslích, projet se na kole nebo dojet k nedalekému sportovnímu kanálu, kde můžete vidět trénovat i vrcholové sportovce či zhlédnout Světový pohár ve vodním slalomu. Směrem na východ od kolejí vede cyklostezka podél řeky do Libně, zpočátku ale chvíli podél silnice. Po ní se dá dostat téměř až ke Kauflandu, k budově Matfyzu v Karlíně a nebo i na Malou Stranu, pokud vám nevadí, že jste si trošku zajeli.

Jestliže se chystáte jezdit na kole, na vrátnicích obou budov kolejí jsou k zapůjčení velké cyklistické pumpičky s univerzální koncovkou pro autoventilky i galusky.

Pokud dáte přednost vnitřním aktivitám nebo si chcete zasportovat v zimě, nabízí se několik možností:

Lezecká stěna – Plno matfyzáků chodí někam lézt. Nejblíž koleje je Mamut, pokud dáte přednost nižšímu lezení bez jištění (boulder), tak BoulderBar. Obě stěny jsou blízko sebe, nedaleko Výstaviště na druhé straně silnice.
Fitness – Přímo za mostem (Argentinská 38) se nachází rozsáhlé studio FitInn s otevírací dobou denně od 6:00 do 24:00. FitInn nabízí několik možností cvičení, od moderních a kvalitních kardio a posilovacích strojů, 15min kruhový trénink, funkční zónu až po oddělený tréninkový prostor pro dámy. Měsíční permanentka vyjde na 499 Kč. Dále je zde možno využít i solária za 9 Kč/min.
Plavání – Nejbližší plavecký bazén je na Výstavišti, hodina plavání vyjde studenta na 50 Kč. Trochu dál, v Kobylisích, se nachází Aquacentrum Šutka, které nabízí 50m plavecký bazén (jeden ze dvou v Praze - druhý je v Podolí, ale ten je jen 49,98m), aquapark a rozumné ceny, obzvlášť s přednabitou "permanentkou".
Squash – V Praze jsou vesměs všude relativně mizerné kurty (tzn. povětšinou jsou všechny stěny, až na zadní, dřevěné). Jedny takové kurty se nachází i v Holešovicích Squash-Holešovice. Jak už bývá obvyklé, studentské ceny se pohybují od 90 Kč do 290 Kč v závislosti na čase.


\subsubsection{Obchody}
Tady uvádíme seznam obchodů, hlavně potravin, díky kterým budete přežívat.

\subsubsubsection{Karlín}
Billa u zastávky tramvaje Křižíkova směr Florenc. Příjemně blízko škole.
Billa mezi východem metra C a zastávkou tramvají Florenc.
Albert přes ulici naproti metru za zastávkou tramvají Florenc.
Lidl v Bílé Labuti, jedna zastávka od Florence na opačnou stranu než budovy MFF UK v Karlíně.
Stánek s pečivem u metra Křižíkova. Fajn místo, kam si zajít pro svačinu, cca. od 15:00 mívají padesátiprocentní slevu na většinu sortimentu.

\subsubsubsection{Malá Strana}
Bio-Vacek Mostecká 3 blízko Karlova mostu, na pravé straně silnice (směrem ze školy). Vhodný obchod, kam si zajít pro svačinu, musíte se dívat na sortiment, něco předražené, něco za normální ceny.
Žabka Karmelitská 24 asi 100 m od Malostranského náměstí po kolejích směrem k Andělu. Také vhodný obchod pro svačinu.
Billa Letenské náměstí - není přímo na Malé Straně, ale u zastávky tramvají Letenské náměstí. Informatici do ní rádi chodí, protože je po cestě tramvají na kolej.

\subsubsubsection{Karlov}
Fyzici a matematici z Karlova mají nejlepší nakupovat na Florenci, ale něco se dá sehnat i na Pavláku.

Žabka Sokolská 31, blízko Karlova, kousíček od metra ve správném směru.

\subsubsubsection{Okolí Koleje 17. listopadu}
Kaufland v Libni mezi zastávkami Palmovka a Libeňský most. Lze se sem dostat tramvajemi 3, 8 a 24 z Karlína, 6 z Holešovic nebo na kole z většiny po cyklostezce podél řeky.
Tesco Letňany je umístěno u zastávky busu 201 a 911 a má otevřeno nonstop.
Kolejní obchůdek v přízemí budovy B, hned za vstupními dveřmi.

\subsubsubsection{Jiné obchody, které se můžou hodit}
Prodejna knih MatfyzPressu v budově MFF na Karlíně.
Pražská tržnice v Holešovicích, jedna zastávka tramvají z metra C Vltavská. Nachází se zde několik hal, mj. hala s několika stánky se zeleninou, potraviny Norma a Penny Market, plno stánků od Vietnamců, pošta, Alza, kuchyňské potřeby, vše za 15, atd.
Knihkupectví NeoLuxor na Václaváku. Obsáhlé knihkupectví, plno druhů map, cizojazyčné knihy, možnost objednání knih.
IKEA blízko zastávek metra Černý most nebo Zličín. Vhodná, pokud si chcete levně pořídit doplňky na kolej - závěsná látková skřínka, zrcadlo, kuchyňské nádobí, lampička...
Decathlon kousek od Tesca v Letňanech (na druhém konci Obchodnícho centra Letňany), u metra Černý most (v Centru Černý most), u metra Zličín (Nákupní centrum Metropole Zličín) nebo zastávku od metra Chodov (sem se bez auta hůře dostává). Sportovní velkoobchod, kde seženete v zimě plavky a v létě čepice. Velký výběr sportovních potřeb a oblečení. Pokud nepotřebujete nejvyšší kvalitu (k tomu slouží specializované obchody), seženete většinu zde, levně a poměrně kvalitní.




\subsubsection{Hospody}
Hospody, restaurace, výčepy, stánky a další zařízení, kam se chodí (nebo jezdí, letí, plazí, plave, skáče nebo cokoliv jiného) uhasit žízeň, jsou jednou z nejdůležitějších částí studentského života. Je jedno, jestli zažíváte horký den, schováváte se před bouřkou, přednáškou, přítelem/-kyní nebo vám strčil spolubydlící do ruky stovku s tím, že potřebuje volný pokoj. Jestli nevíte, co s dlouhým zimním večerem, chcete slavit nebo prostě máte chuť na pití, použijete nějakou hospodu. V Praze jich je nespočetné množství, nějakou najdete takřka kdekoliv. V každé ulici, na každém náměstí. Proto je tenhle text prakticky zbytečný a slouží pouze pro inspiraci. Záleží jen na vás, jestli chcete být štamgast v jednom podniku, nebo budete raději objevovat, poznávat a porovnávat. V dalším textu dáme tipy na hospody okolo (pro matfyzáky) důležitých míst.

\subsubsubsection{Kolej 17.listopadu}
Kolej 17. listopadu je pro většinu studentů místo, kde prožijí několik let. Bohužel má dost nešťastné umístění, za hospodou se musí cestovat.

Asi nejblíže je letní zahrádka přímo na tramvajové zastávce Výstaviště Holešovice, kde točí pivo z rakovnického pivovaru Ferdinand. Z Nádraží Holešovice je to jedna zastávka tramvaje. Pivo je dobré a za rozumnou cenu.
Pivovar Marina – vaří si vlastní dobré pivo. Dostaneme se tam pěšky nebo např. busem 156, vystoupíme na zastávce Přístav Holešovice, vrátíme se pár metrů na kruhový objezd a na něm odbočíme směrem k Vltavě. Po levé straně stojí vytoužený cíl. Terasa v létě poteší.
U Houbaře - jedeme tramvají na zastávku Veletržní palác, potom pěšky kousek ve směru tramvaje (pokud byla předchozí zastávka Výstaviště Holešovice) a po levé straně na rohu budovy je náš podnik. Čepují dobrého Pilsnera a dá se tu dobře najíst. V okolí Veletržního paláce je hospod obecně hodně, nemusíte zrovna k Houbaři. Najdete tu třeba restauraci Valcha (čepují Pilsner, podnik je nepřehlédnutelný na rohu mezi Výstavištěm a Veletržním palácem), nonstop se Staropramenem (mezi Houbařem a Valchou) nebo třeba levný výčep, kde mají Gambrinus (pod Houbařem).
Slovensko-česká restaurace | U Skřetů je vhodná nejen pro Slováky, co touží po kousku domoviny v podobě slovenského piva Zlatý Bažant anebo tradičních slovenských jídel. Podnik najdeme mezi zastávkami Kamenická a Letenské náměstí. Hledáme ceduli Zlatý Bažant po pravé straně ve směru na Letenské náměstí.
Nebo se můžete vydat nahoru do Kobylis. Například u metra Ládví je pivovar Cobolis, kde občas pořádají prohlídky i s degustací. Nemůžete si ho nevšimnout, když vyjdete z metra směrem na knihovnu.

\subsubsubsection{Karlov}
Tady nic není. Teda je, ale musí se více hledat. Nejblíže je Švejk, naproti centrále Dopravního podniku v ulici Na Bojišti. Pokud se vypravíte na Pavlák po magistrále (u Urologické kliniky doprava, ale místo do podchodu to vzít doleva), projdete kolem Kulového blesku, kde mají široký výběr rodinných pivovarů (např. Matuška). Oba podniky jsou ale pro křehký studentský rozpočet trochu drahé, musíme proto využít jiných možností. Můžeme například jít podchodem, který začíná u Urologické kliniky, na jeho konci najdeme proti sobě dva podniky. Nebo se vyprdnout na složité hledání a prostě si vybrat něco kolem metra I.P.P, je tam toho dost. Je tu ještě jedna záludná možnost. A to využít bar Mrtvá ryba. Navštěvují ho posluchači Přírodovědecké fakulty, je tu tedy prostor pro seznámení se. Obvykle vám dají pivo Černá Hora v dobré ceně. Lokalizace je patrně složitější a je dál od Matfyzu - dejte si do Googlu adresu Benátská 4. Vyzkoušenou máme také hospodu Merlin Irish Pub a Galerka nedaleko od budečské koleje.

\subsubsubsection{Karlín}
V Karlíně není problém najít dobrý podnik, kde uhasit žízeň, zde jenom doporučíme několik oblíbených míst. Nejsnáze k nalezení je hostinec U Tunelu, projdete kolem metra Křižíkova a pokračujete rovně, podnik se nachází před vchodem do Žižkovského tunelu po pravé straně. Pokud rádi poznáváte nová piva, doporučujeme vám pivní bar Diego, kde mívají na čepu až deset piv z různých pivovarů. Najdete ho, když se dáte od školy směrem na Invalidovnu (tedy opačně než na Florenc), bar se nachází za semafory po pravé straně. Po pracovní době tam ale bývá plno lidí z blízkých kanceláří. Na nábřeží poblíž zastávky přívozu Rohanský ostrov se v létě nachází příjemný bar se zahrádkou, kde čepují pivo Přístav. Pokud máte štěstí a není tam moc lidí, je to milé místo, kam si sednout a v klidu pracovat.

\subsubsubsection{Malá Strana}
Malá Strana je turisty vyhledávaná lokalita, a tak se musíte připravit na vyšší cenovou hladinu. Vyzkoušet můžete například blízkou restauraci | U Glaubiců, kterou najdete takto: vyjděte z Matfyzu a vydejte se doprava, podél kolejí k první silnici (po pravé straně), tu přejděte na druhou stranu a vydejte se směrem nahoru. Kousíček po levé straně objevíte restauraci.

\subsubsubsection{SCUK Hostivař}
Kolem SCUKu máme dvě základní možnosti. Pokud jedete vlakem (často nejrychlejší spojení), nepřehlédnete výčep Hostivařská nádražka. Levný Staropramen potěší obzvlášť při čekání na vlak. Druhá možnost je také nepřehlédnutelná, Radegast vám natočí při cestě ze SCUKu na zastávku Gercenova (nebo při cestě do SCUKu, ale pozor, abyste to nepřehnali cestou na tělocvik).


\subsubsection{Čajovny}
Občas se najde chvíle, kdy si chcete s kamarády posedět nad šálkem dobrého čaje, vyřešit úkoly, nebo si naopak od všeho odpočinout. Přinášíme vám pár tipů na ozkoušené blízké čajovny.

Kousek od ulice Ke Karlovu se nachází čajovna Na Cestě, na ISIC tam dávají drobnou slevu (asi 5 %). Matematici z Karlína pak mají blízko jednu z nejlepších čajoven, čajovnu Dharmasala, kterou naleznete na Karlínském náměstí. Mají zde výborné čaje a klid. Po cestě z Malé Strany na kolej se vyskytuje čajovna U Kostela na Štrosmajeráku, ale ve špičce bývá dost zakouřená od vodnic. Poblíž Národního divadla je velmi příjemná Květinová čajovna. Poslední, trochu netradiční, čajovna, která stojí za zmínění, je čajovna U Cesty (neplést s Na Cestě), která se pohybuje v Botanické zahradě.

Plno matfyzáků má rádo dobrý čaj, a tak nikoho nepřekvapí, že i na koleji pár takových bydlí. Ať už někoho takového znáte nebo jste ho jen zahlédli s gaiwanem na přednášce a máte dost odvahy, můžete také zkusit možnost "Zaklep na dveře správnému člověku a popros ho o čaj".


\subsubsection{Ostatní}
Kulturního vyžití je v Praze opravdu dostatek. Po celém městě najdete desítky divadel a kin. Jejich objevování necháme na vás, ale vězte, že mají opravdu co nabídnout. Většina divadel nabízí hodně výrazné studentské slevy, doporučujeme se po nich porozhlédnout. Speciální slevy do některých divadel uvádíme na stránkách Spolku Matfyzák.

Pokud se chystáte vyrazit do kina, kromě více než 5 multikin můžete navštívit některá menší, která do programu zasazují méně známé filmy z mnoha různých žánrů. Nejblíže kolejím je holešovické Bio OKO.

A v neposlední řadě můžete trávit čas ve společnosti ostatních matfyzáků. Můžete zpívat ve sboru Sebranka nebo se účastnit matfyzáckých tanečních večerů. Pokud věci raději organizujete, zapojte se třeba do organizace některého korespondenčního semináře nebo se přidejte k nám do Spolku. I na koleji se děje spousta zajímavých akcí. Pokud nějakou zahlédnete na facebookové kolejní skupině nebo třeba jen zaslechnete na schodech kytaru, nebojte se přidat.