\subsection{Další koleje}
\budovaAdresa{
    Kolej Na Větrníku, Na Větrníku 1932, 162 00 Praha 6
    Kolej Kajetánka, Radimova 12, 169 00 Praha 6
}
Zatímco 17. listopad má pěkné pokoje, ale nevábné okolí,
kolej \textit{Na Větrníku} (lidově Větrák) to má naopak ---
hezké okolí a park přímo u koleje, ale prťavé a ošklivé pokoje.
Na druhou stranu je celkem levná.
Asi 20 minut chůze od ní je kolej \textit{Kajetánka} (lidově Kajka),
kde potkáte hlavně mediky.

\budovaAdresa{
    Kolej Areál Hostivař, Weilova 2, 100 00 Praha 10
}
Pokud jezdíte na tělocvik do \textit{Hostivaře}, budete souhlasit,
že to snad ani není Praha. Přesto zde má UK kolej.
Z~nějakého důvodu zde často umísťují zahraniční studenty.

\budovaAdresa{
    Kolej Vltava, Chemická 953, 148 28 Praha 4 - Jižní Město
    Kolej Otava, Chemická 954, 148 28 Praha 4 - Jižní Město
}
O kolejích na \textit{Jižním Městě} někdo tvrdí,
že mají dobrý poměr kvality k~ceně, ale někomu zase přijdou trochu z~ruky.

\budovaAdresa{
    Kolej Budeč, Wenzigova 20, 120 00 Praha 2
}
Kolej \textit{Budeč} má docela malé pokoje, ale opravdu velké schodiště.
Kolej Budeč je celkem blízko ke Karlovu.
Malá kolej v centru Prahy (7 minut od I.P. Pavlova) má kolem 225 obyvatel a
její předností je tedy komorní atmosféra, kde se skoro všichni znají.
Obyvatelé koleje často sami organizují malé párty, 4x ročně i velké.
Pokoje jsou trochu menší a hlavně nevyvážené, některé dvoulůžáky jsou
velké jako trojlůžák. Od roku 2017 jsou opravená první tři patra.
Nevýhodou by snad mohla být pouze jediná kuchyň na patře,
dvoje společné záchody a sprchy. Upozornění předem: nejsou
rozdělené podle pohlaví. Uklízečka ale uklízí denně, čistota budiž zajištěna.
Velkou předností koleje je její menza, jedna z nejlepších na celé Karlovce.
Každé patro má svoji studovnu a často zde najdete mediky, právníky a filosofy.
Jak je to s cestou po budovách MFF? Nad Albertov to máte pěšky 6 minut, na
Malostranské náměstí tramvají č. 22 z IP za $\pm$ 17 minut
(pro spěchající metrem i pod 14 minut). Na Tróju je to dobrých 20 minut metrem
a autobusem na Kuchyňku. Celkově stačí vycházet vždy s půlhodinovým náskokem.
Hostivař je i odsud dost daleko, ale z IP se to vyplatí vzít 22.
Zpátky bývá nejrychlejší jet vlakem (jede jeden za cca půl hodiny) na
Hlavní nádraží, odsud jsou to dvě zastávky metrem.
Nejjistější je si vzít dobrou knihu, nastoupit na 22 a po ránu se nikam nehnat.

\budovaAdresa{
    Kolej Švehlova, Slavíkova 22, 130 00 Praha 3
}
V hezké historické budově sídlí i \textit{Kolej Švehlova} (neboli Švehlovka),
jen je oproti Budči podstatně větší a místo menzy má skvělé divadlo.
Nachází se u náměstí Jiřího z Poděbrad, a tak to z ní nemáte do žádné budovy
MFF vyloženě blízko, ale zároveň ani do žádné daleko (včetně Hostivaře!)