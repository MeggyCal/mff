Vítejte, ať už čtete tento text z obrazovek vašich počítačů,
z displejů vašich chytrých telefonů nebo držíte v ruce tištěnou verzi této ne
jen tak obyčejné kuchařky. Vězte, že vám může být v lecčem užitečná.


V této kuchařce nenaleznete recept na jehněčí ragů, protože je příliš složitý.
Najdete v ní ale recepty k přežití na Matfyzu.
Přestože bychom vás jen neradi okrádali o požitek z objevování krás
fakultních budov, tunelů metra nebo věží matičky stověžaté, věříme,
že malá pomoc na začátek se vám může hodit.
Ušetříte si tak možná nejeden trapas nebo nedorozumění a zcela určitě
alespoň část stresu spojeného s nástupem na novou školu.
Tato kuchařka vznikla jako průvodce po budovách MFF,
životem na kolejích (převážně Koleji 17. listopadu) a cestováním pražskou MHD.


První Studentská kuchařka vznikla v roce 1994 zásluhou Martina Krynického
(verze z roku 1998 je dostupná zde).
Od té doby se noví a noví autoři, členové Spolku Matfyzák,
snaží kuchařku vylepšovat opravováním starých chyb a přidáváním chyb nových.
Od roku 2011 je volně přístupná na webu jako wiki.


Popíšeme a vysvětlíme tu téměř všechno, jen jedno zachytit neumíme:
atmosféru přednášek, nenapsané zápočty, referáty odměněné potleskem
či provázené smíchem nebo hlasitým chrápáním, probdělé noci,
týdny a semestry strávené měřením fyzikálních praktik a následným
smolením protokolů, první známku v SISu, zkoušky úspěšně složené
nad hrnkem čaje profesorem ochotně nabídnutým či zkoušky končící
zápisem neprospěl.