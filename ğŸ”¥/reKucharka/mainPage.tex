\textit{V této kuchařce nenaleznete recept na jehněčí ragů, protože je příliš
složitý.  Najdete tu ale recepty k~přežití na matfyzu. Neradi bychom Vás
okradli o kouzlo objevování toho, co znamená být studentem MFF. Na druhou
stranu, místo ú\-spě\-chu Vám to může přinést jen spoustu trapasů a zbytečné
časové ztráty. Tento průvodce \MFF{}ou, kolejí \17l, pražskou MHD a Prahou
vůbec tedy vznikl pro ty z~Vás, kteří se nechtějí se vším mořit sami.}

\textit{První Studentská kuchařka vznikla v~roce 1994 zásluhou Martina
Krynického. Od té doby se noví a noví autoři snaží kuchařku
vylepšovat opravováním starých chyb a přidáváním chyb nových. Od roku 2012 je volně dostupná na webu jako wiki na adrese \url{http://kucharka.matfyzak.cz}.
Uzávěrka tohoto vydání byla v~červnu 2012.}

\textit{Popíšeme a vysvětlíme tu téměř všechno, jen jedno zachytit
neumíme: Atmosféru přednášek, nenapsané zápočty, referáty odměněné
potleskem či provázené smíchem nebo hlasitým chrápáním, probdělé
noci, týdny a semestry strávené mě\-ře\-ním fyzikálních praktik
a~následným smolením protokolů, první známku v~indexu, zkou\-šky
úspěšně složené nad hrnkem čaje profesorem ochotně nabídnutým, či
zkoušky končící zápisem neprospěl\dots}