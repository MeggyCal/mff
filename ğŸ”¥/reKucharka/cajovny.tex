\subsection{Čajovny}

V~Praze je taktéž spousta čajoven. Útulná je {\it Čajovna nad Vokem},
bystrému čtenáři k~navigaci postačí, že se nachází nad hospodou {\it
U~Vystřelenýho voka}.

Velice pěkná je {\it Čajovna ve věži}. Pojedete ze Strossmayerova
náměstí tramvají na Letnou, vystoupíte na zastávce Sparta, půjdete
zpátky podél kolejí směrem k~Letenskému náměstí, u~Ministerstva
vnitra (dvě velké ošklivé budovy) odbočíte doleva Korunovační
ulicí. Kde máte odbočit, poznáte podle toho, že tam začíná velká
a~nepřehledná křižovatka. Čajovna je opravdu ve věži, přehlédnout
ji nelze.

Do třetice ještě uvedeme {\it Růžovou čajovnu\/} v~Růžové ulici (mezi
Opletalovou a Jindřišskou). Je poměrně známá, příjemná a pořádají se zde
různé besedy, zejména s~cestovateli.

Blízko ulice Ke Karlovu se nachází čajovna {\it Shangri-La}, kde se často potkáte se studenty matematiky. Informatici mají zase blízko trochu neobvykle zařízenou, ale výbornou čajovnu {\it Bílý jeřáb}.
