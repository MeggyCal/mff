\subsection{Zdroje informací}
Ať už hledáte příklady z Děmidoviče (Sbírka úloh a cvičení z matematické analýzy), kupce na hromadu krabic od pizzy nebo
příležitost na balení holek, existuje mnoho úložišť a informačních kanálů, na které se obrátit. Ty z nich, které jsou
specializované na matfyzácké potřeby, jsou uvedené v následujícím textu.

\subsubsection{Elektronické zdroje}
\subsubsubsection{Wiki}
Aby to nebylo jednoduché, existuje několik matfyzáckých wiki

Na \url{kucharka.matfyzak.cz} je kopie této kuchařky. Jednou za rok jí vezmeme, vytiskneme a dáme prvákům jako jste Vy.

Na adrese \url{wiki.matfyz.cz} je sbírka nejrůznějších informací od matfyzáků a pro matfyzáky týkajících se studia
(zápisky,
poznámky, zadání dřívějších písemek, projekty, témata diplomek atd.), zkušeností s vyučujícími a předměty, studia v
zahraničí, způsobu trávení volného času v Praze a dalších věcí. Na této wiki je spousta zastaralých stránek. Užitečné
podstránky ale jsou \textit{Předměty} se všemi předměty a \textit{Vyučující} s vyučujícími.

Další, kdysi funkční, wiki je \url{mff.lokiware.info}. Teď už není aktulizovaná, ale dají se tam najít informace
užitečné
především pro studenty informatiky.


\subsubsubsection{Fórum}
Na \url{forum.matfyz.info} je (opět neoficiální!) fórum fakulty. Neocenitelný zdroj informací hlavně pro informatiky (ti
ho založili), ale také pro matematiky i fyziky. Doména \url{www.matfyz.info} poskytuje užitečný rozcestník.

\subsubsubsection{SKAS}
Studentská komora Akademického senátu vyvěšuje své stránky informace o dění na fakultě a v~akademickém senátu na svém
webu \url{skas.mff.cuni.cz}. Kromě pravidelných zpráviček (SKASky) tam ale najdete i aktuální informace o změnách
v~předpisech, různé pozvánky na přednášky, návody nebo výsledky studentské ankety za poslední roky.

\subsubsubsection{Spolek Matfyzák}
Na webu \url{spolek.matfyzak.cz} se objevují informace o chystaných
akcích nebo si tu můžete objednat matfyzácké tričko, polštářek či
hrníček.

\subsubsubsection{Facebook}
Pokud nejste odpůrcem tohoto informačního kanálu, doporučujeme sledovat stránky školy, fakulty, knihovny, dále Spolek
Matfyzák a SKAS MFF. Bydlíte-li na Koleji 17. listopadu, může pro vás být užitečná ještě skupina Koleje 17.listopadu,
případně Koleje 17. listopadu – Společenské Deskové Hry a KolejBĚH (Kolej 17. listopadu).

Nastupující prváci každého oboru mají zpravidla založenou skupinu, kde si sdílejí informace o zkouškách, domácí úkoly a
další užitečné věci.

\subsubsubsection{Propagační akce Matfyzu}
Informace o Matfyzem pořádaných akcích (obvykle pro středoškoláky a základoškoláky, ale nějaké i pro širokou veřejnost),
můžete najít na další wiki, a to ovvp.mff.cuni.cz. Je tam i kontakt, pokud byste se chtěli zapojit do propagačních akcí
jako organizátoři. Propagační články se pak objevují na matfyz.cz.

\subsubsubsection{Studnice vědomostí}
Pokud jste informatici a máte přístup k Linuxové laboratoři, na /afs/ms/doc/vyuka je tzv. studnice vědomostí – různá PDF
a další zdroje k různým předmětům.

\subsubsubsection{Google}
A samozřejmě platí --- pokud něco nevím nebo neumím, nemusím se to učit, stačí mi Google. \url{www.google.com}

\subsubsubsection{Matfyzácké konference}
Jsou mailingovým listem, ve správě odpovídajícího oddělení. Stud-l: zprostředkovává informace od studijního oddělení a
poradenských pracoviště směrem ke studentům. Budete tak dostávat mailem zprávy o vyhlášení různých grantů, soutěží,
úředních hodinách na SO, ale rovněž o plánovaných seminářích a akcích pro studenty mimo rámec výuky. Registrovat se však
musíte sami prostřednictvím https://lists.karlov.mff.cuni.cz/mailman/listinfo/stud-l Podobný mailing list má i oddělení
zahraničních vztahů, které informuje jeho prostřednictvím o stipendijních programech, cenách a konferencích. Případně i
knihovny apod. Informace o těchto rozesílacích seznamech naleznete na psik.mff.cuni.cz.

\subsubsubsection{Matfyz FAQ}
Máte pocit, že podobný dotaz ze života na Matfyzu už třeba někdo položil? Pak možná visí i s odpovědí na Matfyz FAQ!


\subsubsection{Nástěnky}
Důležitý zdroj informací pro každého studenta. Na nástěnkách SKAS
(všude), kolejní rady (na koleji), oddělení propagace a vnějších
vztahů (hlavně nabídky prací a brigád) či studijního oddělení (na
Karlově) bývají důležité (a~zřídka i naprosto nedůležité)
informace týkající se oné oblasti. Na nástěnkách jednotlivých
kateder bývají termíny zkoušek (to už nyní není tak časté, většina
učitelů se spoléhá na SIS, takže pokud Vám to neřeknou jinak,
spoléhejte se také a termíny tam najdete) či úmluvy na výběrové
přednášky. Na ostatních nástěnkách se dozvíte, kdo prodá skripta,
kdo koho doučí, co promítají pražská kina, jaká zajímavá
zaměstnání pro studenty nabízí různé firmy nebo od koho levně
koupíte kolečkové brusle.


\subsubsection{Knihovna MFF}
Knihovna MFF je rozčleněna na 3 oddělení (matematiké, fyzikální
a~informatické). Oddělení fyzikální se nachází v~budově Ke Karlovu 3
(v~první patře), oddělení matematické v~budově Sokolovská 83.
Nejmladší je oddělení informatické --- sídlí v~budově na
Malostranském náměstí 25. Zde také najdete fond knihovny
lingvistiky. Další, pro studenty nepostradatelnou knihovnou je
Půjčovna skript a učebnic --- tu naleznete Tróji, v~přízemí budovy
V~Holešovičkách 2.

V~knihovně si lze půjčit skripta a učebnice (jen na jeden semestr,
tj. 6 měsíců) i knížky (tj. \uv {neskripta}, ty se půjčují na
jeden měsíc). Přehled vašich výpůjček najdete online na Centrálním
systému univerzity \url{ckis.cuni.cz/}. Diplomové práce
a~časopisy se domů nepůjčují, ale můžete si z~nich okopírovat, co
potřebujete. V~každé z~knihoven funguje velká kopírka na
kopírovací karty --- můžete si je koupit v~knihovně.

Knížky na regále jsou označeny různými barevnými identifikačními
štítky. Ty bílé s~písmeny, co jsou nalepeny na hřbetu knihy, jsou
signatury (tj. adresy na regále) --- podle těch knížky hledáme
a~v~podstatě znamenají to, že si tyto knihy lze vypůjčit na 1 měsíc.
Zelené štítky značí, že si můžete knížku půjčit na celý semestr
a~oranžové jsou na knihách, které si odnést nemůžete, a můžete ji
prostudovat jen v~prostorách knihovny.

Každý student si smí půjčit nejvýše 20 dokumentů. Výpůjčky si
můžete samozřejmě prodlužovat a také je umožněno si rezervovat
právě nedostupné dokumenty. Upozornění o tom, že si už knížku
můžete vyzvednout, chodí na e-mailovou adresu (k~tomu je
pochopitelně nutné zadat do SISu správný e-mail).

Také si můžete zřídit tzv. elektronické konto --- můžete si přes
něj kontrolovat výpůjční lhůty vašich dokumentů nebo je
prodlužovat.

Dalším důležitým zdrojem informací pro studium jsou nejrůznější
databáze --- jejich přehled najdete na stránkách knihovny \url{www.mff.cuni.cz/lib/}. Matfyz na jejich předplácení
vynakládá nemalé prostředky (řádově v~milionech korun), tak si
jich patřičně važte a hojně je využívejte.

Na výše uvedené webové stránce najdete i elektronický katalog
knihovny (přes příslušnou ikonku můžete hledat také v~katalozích
ostatních fakult UK). Jinak lze v~katalogu vyhledávat také přímo
v~jednotlivých odděleních knihoven v~tzv. OPACu (Online Public Acess
Catalogue). Při hledání dokumentu přímo na regále Vám ráda pomůže
služba u~výpůjčky.

Registrace do knihovny Vás nebude stát ani korunu! Potřebujete
pouze zelenou studentskou kartu. a jak se jednou v~některém
oddělení knihovny zaregistrujete, můžete navštěvovat i všechna
ostatní oddělení a dílčí knihovny (meteorologie, astronomie,
geofyziky či Knihovnu dějin přírodních věd).

Co je však pro úspěšné fungování v~knihovně nutné vědět, je ten
fakt, že při překročení stanovené výpůjční lhůty Vám knihovna může
nasolit pěknou pokutu. Knihovna totiž kasíruje studenty ještě
přísněji než dopravní policie řidiče a za každý 1 den a 1 dokument
budete platit 1 Kč. a to bez varování v~podobě papírové či jiné
upomínky. Pokuty se nepromíjejí, kromě tzv. {\it odpustkového
týdne}, který připadá na Týden knihoven, což je někdy v~polovině
října. a ani tehdy se Vám nepromine jen tak, ale máte šanci si ji
v~knihovně odpracovat.

\subsubsubsection{Další zdroje literatury}
Občas je výhodné obrátit se i na mimofakultní zdroje, třeba na
{\it Městskou knihovnu}~---~ta je na Mariánském náměstí poblíž
stanice metra Staroměstská~---~nebo na {\it Národní technickou
knihovnu}, která se nachází v~krásné budově v~Dejvicích. {Národní
knihovna} je umístěna v~Klementinu a pokud jste vytrvalí, dá se
tam sehnat téměř vše (ostatně vedou CEZL, tj. Celopražskou
evidenci zahraniční literatury, takže mají přehled i o fondu
menších knihoven a jsou schopni poradit i telefonicky).Matematici
mohou mít se speciálnějšími požadavky úspěch v~knihovně
Matematického ústavu Akademie věd v~Žitné.