\subsection{Studium}
Studium na nejlepší fakultě na světě se může zdát složité, leč při držení se doporučených postupů může být i radostné. Vše o povinných předmětech, doporučených průbězích studia a dalších povinnostech se dozvíte na webu fakulty nebo v Oranžové Karolínce. První dva roky slouží jako síto, které způsobí, že nějakou část vašich spolužáků už neuvidíte. Překonání tohoto začátku vám však dává dobrou šanci Matfyz úspěšně dokončit. Existuje spousta velice užitečných rad, které vám budou starší kolegové a vyučující dávat, ale vy se jimi řídit nebudete a noc před zkouškou si budete říkat: "Proč já se neučil už v průběhu semestru?"


\subsubsection{Boloňský systém}
Na naší fakultě se studuje podle Boloňského systému, pojmenovaného podle boloňských špaget. Nejprve se vaří těstoviny (bakalářské studium), poté omáčka (navazující magisterské studium), což většině studentů stačí, ale někteří připravují další zdobení, sýry a bylinky (doktorské studium). Naše kuchařka se omezuje pouze na základy vaření těstovin, případně omáček.

\subsubsubsection{Bakalářské studium}
Suché těstoviny, tedy bakalářské studium (Bc.), je to první, co začnete na Matfyzu vařit. Můžete si vybrat ze tří studijních programů: špagety (informatika), makaróny (matematika), nebo tagliatelle (fyzika). Postup přípravy je u všech stejný, jen jsou k němu potřeba jiné předměty. Vaří se zpravidla 3 roky. Někteří je vaří roky čtyři, což nemusí být na škodu, ale pokud budete vařit ještě déle, pak si za každý další rok už budete muset zaplatit.

Vaření končí státní bakalářskou zkouškou, ke které jste připuštěni, jestliže splníte všechny povinné předměty, nasbíráte alespoň 180 kreditů a napíšete bakalářskou práci. Po jejím splnění se z vás stanou bakaláři.

\subsubsubsection{Navazující magisterské studium}
Po úspěšném uvaření těstovin většina studentů vaří i omáčku, tedy navazující magisterské studium (NMgr.). Vaření omáčky bývá kreativnější a volnější než v případě těstovin. Po úspěšném uvaření dostanete titul magistra (Mgr.). I běžně dvouleté studium NMgr. si můžete bezplatně o jeden rok prodloužit, ale raději si to ověřte, neboť předpisy se občas mění.


\subsubsection{Studijní programy}
Na Matfyzu se studuje ve studijních programech (týká se i těch, kteří mají k programování daleko). Lze si vybrat matematiku, fyziku, nebo informatiku. Názvy programů asi z 74,314 % souhlasí s tím, co se v nich opravdu studuje. Každý program obsahuje několik oborů, které blíže specifikují zaměření vašeho studia.

Speciální obory mají budoucí učitelé. Ty mají podle zaměření prazvláštní zkratky FMUZV, MDUZV, MIUZV, ZMIJOZEL, MRZIMOR a další a formálně se řadí pod různé programy.



\subsubsection{Předměty}
Předmětů se na Matfyzu vyučuje hodně (všechny najdete v SISu nebo v Bílé Karolínce).


\subsubsubsection{Kredity}
Kredity jsou bohatství, které získáváte za splněné předměty. Jejich počet rozhoduje o vašem postupu do dalšího úseku studia. Neukládají se vám na kreditní kartu, nýbrž na vaše konto v SISu.

Každý předmět má pevně daný počet kreditů, ale nejsou kredity jako kredity. Podle konkrétního studijního oboru jsou nějaké předměty povinné, nějaké povinně volitelné a ostatní volitelné.

Pro postoupení do dalšího ročníku je vždy nutné mít alespoň 45 kreditů z ročníku v průměru (tedy pokud v prvním ročníku nasbíráte přes 90 kreditů, nemusíte ve druhém hnout prstem), normální počet je však 60 za ročník, čemuž odpovídají doporučené studijní plány. Povinné a povinně volitelné kredity musí mít převahu, pro 1. - 3. ročník Bc. studia se z volitelných kreditů pro účely kontroly uznává max. 15 % z normálního počtu kreditů. Všechny potřebné hranice naleznete v Pravidech pro organizaci studia na MFF UK a ve Studijním a zkušebním řádě UK, oba dokumenty vč. jiných užitečných naleznete na fakultním webu. Kromě ročníkové kontroly studia je v prvním ročníku zavedená průběžná kontrola již po zimním semestru, kdy potřebujete nasbírat alespoň 15 kreditů. Nemyslete si ale, povinné předměty splnit musíte, ať máte kreditů, kolik chcete.

Na rozumné dokončení bakalářského studia potřebujete získat celkem 180 kreditů (pokud studujete více než 5 let, počet se zvyšuje kvůli průměru 45 kreditů na rok).

Pokud si nebudete s něčím jisti, neváhejte se obrátit s dotazem na SKAS nebo na Spolek Matfyzák, využít můžete například e-mailovou adresu sos (na) matfyzak.cz.



\subsubsubsection{Prerekvizity a jiné}
Předměty mohou mít prerekvizity, korekvizity, neslučitelnosti a záměnnosti s jinými předměty.

Předmět v prerekvizitě musíte mít před zápisem nového předmětu splněný; předmět v korekvizitě si musíte zapsat nejpozději společně s novým předmětem; nový předmět nelze zapsat, máte-li zapsaný či splněný neslučitelný předmět, a je-li předmět záměnný s jiným, splněním jednoho z nich splníte oba (ale kredity dostanete za ten, který jste reálně absolvovali).




\subsubsubsection{Zkoušky}
Každý předmět si můžete zapsat dvakrát ve studiu a pokaždé jít na zkoušku třikrát, pokud stihnete vypsané termíny.

Ze zkoušky nemůžete dostat horší známku než 4, což není způsobené tím, že by zkoušející byli tak hodní a nedávali 5, ale tím, že horší známka než 4 není. Známky 1 - 2 mají oproti 3 výhodu v tom, že při opakovaném zápisu na školu nemusíte tyto předměty opakovat (to by vám v ideálním případě nemuselo vadit).

Jak taková zkouška probíhá? V prváku, kdy je ještě studentů dostatek, bývají zkoušky převážně písemné s případnou ústní částí. Později už bývají zkoušky převážně ústní. Ústní zkouška probíhá například tak, že vám zkoušející vybere nebo vás nechá si vylosovat papírek se zadáním a vy máte spoustu času na rozmyšlenou a sepsání řešení na papír. Někteří vyučující to s vámi vydrží třeba i 6 hodin. Pak to se zkoušejícím proberete a případně vám opět nechá čas na opravu chyb.

Vyučující vypisují zkoušky v SISu, je tedy užitečné se nechat upozorňovat na nově vypsané termíny e-mailem. Pokud vám nevyhovuje žádný z vypsaných termínů, zkoušku jste napoprvé nedali nebo ji nestihli, nebojte se vyučujícímu napsat a poprosit ho o další termín. Většina vyučujících je vstřícná a zkoušku vám vypíše třeba i koncem července.

\subsubsubsection{Zápočty a jiné}
Ne každý předmět musí nutně končit zkouškou, ačkoli většina těch povinných ji má. Abyste mohli ke zkoušce, obvykle potřebujete zápočet, který vám po splnění předem daných podmínek udělí obvykle cvičící. Výjimku tvoří programování, kde je podmínkou zápočtu zápočtový program, který píšete, kdy chcete, a pak zvláštní případy. Předměty neukončené zkouškou bývají končené zápočtem, někdy i klasifikovaným (například na praktikách, zvláštním pekle pro fyziky), nebo kolokviem, o kterém nikdo neví, co to vlastně je, ale zvláště matematici se v něm vyžívají.

\subsubsubsection{Tělocvik a jazyky}
Matfyzák musí umět dobře anglicky (třeba proto, že literatury v angličtině je mnohem víc). Angličtina se vyučuje 4 semestry. Její absolvování není povinné, ale zajistí vám nezanedbatelné množství přilepšujících bodů ke zkoušce z angličtiny, která povinná je (ale dá se z části prominout, máte-li nějakou z mezinárodních zkoušek).

Čtyři semestry z tělocviku povinné jsou, jeden z nich si však můžete nahradit sportovním kurzem. Matfyz naštěstí poskytuje širokou škálu sportovního vyžití (od plavání přes softbal až třeba po kanoistiku). Říká se, že tradičním matfyzáckým sportem je volejbal, v němž se konají pravidelně turnaje studentů i absolventů.

Každý matfyzák však musí umět plavat, což se prokazuje uplaváním čtyř bazénů na začátku studia. Neplavci budou o tělocviku naučeni plavat (ale mohou po dohodě s vyučujícím navíc chodit i na jiný sport, ovšem bez kreditů a zápočtu). Nutnost umět plavat dokládají i ne tak dávné povodně - při těch v roce 2002 dokonce kolej evakuovali a místo matfyzáků rejdily na zastávce 112 kačenky.


\subsubsection{Doporučený průběh studia}
Vedení fakulty vás nenutí dodržovat žádný pevně stanovený řád studia. Abyste postupovali do dalších ročníků, stačí mít dostatek kreditů a na konci studia splněné všechny povinné předměty. Nicméně abyste se neztratili v síti vztahů mezi předměty a aby jejich posloupnost dávala smysl, fakulta pro vás připravila doporučený průběh studia, kterého se většina studentů drží. Jde v podstatě jen o seřazení povinných předmětů a doporučení několika předmětů volitelných. Každý obor má svůj doporučený průběh a naleznete ho v Oranžové Karolínce nebo v SISu. Chcete-li se se pokochat, jaké všechny předměty na univerzitě najdete, nahlédněte do Bílé Karolínky.

Obzvláště v prvním ročníku je nejlepší se jednoduše držet studijního plánu. Pokud si věříte, můžete si navíc zapsat ještě nějaké volitelné předměty z vyšších ročníků, z kterých můžete přinejhorším snadno vycouvat (za neudělané volitelné předměty vám nic nehrozí, až na podmínky stipendia, a pokud chcete a již jeho nemáte zapsané podruhé, můžete si je další rok zopakovat).


\subsubsection{Studijní oddělení}
Studijní oddělení je něco jako úřad studia. Vyřizují se zde všechny vaše žádosti - ať už radostné (například ukončení studia) nebo smutné (například ukončení studia). Potřebu jeho osobní návštěvy čím dál více oddaluje SIS. Studenty každého programu má od druhého ročníku na starosti jedna referentka, prváci mají svou vlastní (tak ji prosím nezlobte).

Fronty se na studijním zpravidla netvoří, nezbývá-li náhodou několik minut do důležitého deadline. Kdybyste někdy měli náladu na studijní neslušně ošklivě myslet, vězte, že to na Matfyzu je jedno z nejpříjemnějších (nejen v porovnání s jinými fakultami).


\subsubsection{SIS}
Studijní informační systém (SIS) je po většinu času fungující systém, který mimo jiné čím dál více snižuje potřebu osobního kontaktu studenta se studijním oddělením. Používá ho celá univerzita, čili až se na něj budete zlobit, představte si, jak s ním musí zápasit třeba takoví filosofové, a hned vám bude líp. SIS naleznete na adrese https://sis.cuni.cz/.

SIS vás provází celým studiem. Pravděpodobně jste se s ním setkali již při podávání přihlášky. Dále si v něm budete hledat a zapisovat předměty, načež se budete moci kochat vaším rozvrhem a ignorovat hlášení, že se vám na čtyřech místech překrývá výuka. O zkouškovém se v SISu budete hlásit na zkoušky a poté smutně anebo radostně koukat na výsledky a aktuální stav kreditů. Také tam naleznete informace o vašich vyučujících nebo studentskou anketu, do které můžete o zkouškovém psát slovní hodnocení vašich vyučujících, a naopak díky které si můžete prohlédnout hodnocení vyučujících a cvičících dříve, než si zapíšete jejich předmět.

\subsubsection{Stipendia}
Při studiu můžete získat 3 základní typy stipendií a to prospěchové, ubytovací a sociální. Dalšími druhy jsou ještě účelové stipendium a pak různé granty a vědecká stipendia. Za účelem hladkého přidělení (a především vyplacení) daného stipendia je vhodné hned na začátku studia fakultě sdělit číslo účtu.

\subsubsubsection{Prospěchové stipendium}
Prospěchové stipendium je přidělováno na konci akademického roku na celý rok příští (akademický); v prvním ročníku se vyplácí už v letním semestru podle výsledků zkoušek ze zimního semestru - pokud se budete pilně učit od prvního dne školy, můžete dostat příspěvek na prázdniny. Jeho výši určuje každý rok děkan a je vypláceno ve dvou mírách, podle ročníku a známkového průměru. Typicky na něj mají nárok studenti, kteří v daném akademickém roce získali alespoň 60 kreditů (pro zimní semestr prváku 30), nepřekročili standardní dobu studia (3 roky na Bc.) a jejich vážený průměr známek ze všech zkoušek, které skládali, nepřekročí danou hranici. Podrobná pravidla pro přiznání stipendia naleznete v předpisech fakulty.

\subsubsubsection{Ubytovací stipendium}
Ubytovací stipendium existuje v rámci celé Univerzity Karlovy. Více v sekci Přidělování koleje.

\subsubsubsection{Sociální stipendium}
Pokud je vaše rodina v tíživé sociální situaci, můžete požádat rektora UK o tzv. sociální stipendium. Pro jeho přiznání se posuzuje zejména výše příjmů rodiny a počet vyživovaných osob. Podmínky přiznání nelze úplně jednoduše formulovat, pokud však máte podezření, že byste na sociální stipendium mohli dosáhnout, neváhejte kontaktovat například své studentské zástupce ze SKASu, kteří vás určitě dokáží nasměrovat na osobu schopnou vám kvalifikovaně poradit.

\subsubsubsection{Účelové stipendium}
Účelové stipendium je jednorázové stipendium, na které máte nárok nejčastěji, když dosáhnete úspěchu ve sportovní či odborné soutěži, při službě v labu nebo když pomáháte s propagací fakulty.

\subsubsubsection{Granty}
A konečně jsou tu stipendia na SFG, což je zkratka pro studentské fakultní granty. Grant si můžete vymyslet sami (třeba sepsání poznámek z celého semestru, pokud ještě nejsou skripta), ale většinou se využívá nabídky kateder (nejaktivnější jsou v tomto směru fyzici). Přihlásit se musíte pro podzimní kolo do 15. listopadu, a pro jarní kolo do 15. května, váš projekt pak posoudí komise pro studentské granty.

\subsubsubsection{Vědecké a doktorské stipendium}
V magisterském a doktorském studiu pak budete mít ještě víc možností - různá vědecká stipendia za pomoc při výzkumu nebo výuce, univerzitní nebo dokonce státní granty... A samozřejmě, doktorandi dostávají svůj "plat" formou stipendia doktorského.



