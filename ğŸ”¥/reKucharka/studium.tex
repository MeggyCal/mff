\subsection{Studium}

Studium na naší fakultě je záležitostí poměrně komplikovanou, ale
pro většinu účastníků radostnou (alespoň se to říká). 

\subsubsection{Boloňský systém}

Studium se dělí se na
bakalářské, navazující magisterské a doktorandské (poslední se
ale, až na trapné výjimky, prváků netýká, a proto se o něm
nebudeme dále zmiňovat). Toto dělení se také nazývá \uv{Boloňský systém} podle boloňských špaget.

Bakalářské studium je zpravidla tříleté
a~po jeho úspěšném ukončení se (pře\-kva\-pi\-vě) stáváte bakaláři
(Bc.). Navazující magisterské navazuje na bakalářské (je
\uv{navazující}) a je také v~těch úspěšnějších případech zakončeno
magisterským titulem. Oboje se dá stihnout za pět let, tři roky bakaláře a dva roky magistra. Oba stupně se každý dají v~případě nouze navýšit beztrestně ještě o rok, za další roky studia platíte jak penězi, tak společenským statutem věčného studenta. Radši se tomu vyhněte.


\subsubsection{Studijní programy}

Na \mfz{}u se studuje ve studijních programech (týká se i těch,
kteří mají k~programování daleko). Lze studovat matematiku (M),
fyziku (F) a informatiku (I); programy se ještě dělí na obory, kterých je moc a jejichž seznam je na webových stránkách školy.

Budoucí učitelé se dělí podle podivných zkratek FMUZV, FMUSSS, FMU2ZV, FMU2SZS, MDUZV, MDUSSS, MIUZV, MIUSSS, UMF, UMI, podle nichž bývají rozděleni mezi programy matematika a fyzika.

Pokud chcete zjistit, který program Matfyzák studuje, zeptejte se ho, který je nejlepší.

\subsubsection{Předměty}
Celé studium má příchuť jakési bojové hry, chybí už
jen černé a červené puntíky (ty si vychutnají pouze informatici
při studiu červeno-černých stromů).

\subsubsubsection{Předměty}

Každý předmět, ať má libovolný status a ať se studentům libovolně líbí nebo nelíbí, má pevně daný počet kreditů. Pokud předmět úspěšně absolvujete (uděláte zkoušku, napíšete zápočtovou písemku), kredity za daný předmět jsou Vaše.

Předměty se dělí na {\bf povinné} --- ty k~dodělání matfyzu {\it potřebujete}; dále {\bf povinně volitelné} --- z~nich potřebujete mít určitý počet kreditů {\it dohromady}, a nakonec předměty úplně {\bf volitelné}. Nikde není pevně stanoveno, který z~povinných předmětů si musíte v~tom kterém roce zapsat, a tak je rozložení předmětů na Vás.

K~tomu, abyste se s~fakultou předčasně nerozloučili, potřebujete
nasbírat minimálně 45 kreditů (a to převážně z~povinných a povinně
volitelných předmětů), přičemž očekávaný (normální) počet kreditů
je 60. 
Nezískáte-li potřebný počet kreditů, jste ze hry diskvalifikováni.

\subsubsubsection{Zkoušky}
Každý předmět je možno si zapsat dvakrát ve studiu a pokaždé jít na zkoušku třikrát; je to sice dostatek, ale někteří nešťastníci opravdu skládají zkoušky až napošesté. Dokážete si představit profesorovu radost, když k~němu student přichází pošesté po pěti neúspěšných pokusech; radost je oboustranná.

Ze zkoušky můžete dostat známky 1 až 4; ne, že bychom byli milosrdnější, než střední škola, ale prostě je 4 nejhorší stupeň. Známky 1 až 2 jsou dobré; 3 má nevýhodu, že při opakovaném zápisu na školu se neuznává (v ideálním případě by Vám to vadit nemělo, ale ideální případy neexistují); pokud dostanete 4, zkoušku si zopakujete ještě jednou.

Kromě získávání kreditů je občas
třeba splnit nějakou netradiční povinnost, jako třeba zkoušku
z angličtiny nebo obhajobu diplomové práce.


\subsubsubsection{Tělocvik a jazyky}

Tělocvik a jazyky mají ve studijních plánech zvláštní úlohu.

Matfyzák musí umět dobře anglicky (například proto, že literatury v~angličtině je neporovnatelně víc). Angličtina sice není povinná, jak bývala, ale vaší povinností je pouze
složit zkoušku. 

Pokud jste dobří (a máte na to nějaké ty papíry), dá tato zkouška z~části prominout, ale ne zcela. V opačném případě ji musíte udělat, v~čemž Vám pomohou i zápočty z~výuky angličtiny, díky nimž získáte k~oné zkoušce přilepšovací body.

Tělocvik je naopak stále povinný. Musíte absolvovat čtyři semestry
tělocviku, ale místo jednoho z~nich můžete jet na týdenní
sportovní kurz. Matfyz naštěstí poskytuje nebývale širokou škálu
sportovního vyžití (od fotbalu a plavání přes softbal až třeba po
kanoistiku či ping-pong), každý si vybere něco, co ho baví. 

Každý matfyzák či matfyzačka však musí umět plavat, a pokud neumí, tak
ho/ji plavat o tělocviku naučí (ale berte to pozitivně, pokud
budete mít o nějaký sport zájem, můžete se domluvit s~vedoucím
kurzu a chodit k~němu navíc, ovšem bez kreditů a zápočtu). A také
nezapomínejme na důležité přísloví (co se v~mládí naučíte...)
a na ne zas tak dávné povodně, při těch v~roce 2002 dokonce kolej
evakuovali a místo Matfyzáků rejdily na zastávce 112tky
spokojené kačenky.

\subsubsubsection{Státnice}
Bakalářský i magisterský pogram jsou zakončeny státní závěrečnou zkouškou, což je něco mezi noční můrou a expedicí na Mount Everest; je to poslední meta, kterou je třeba před získáním titulu zvládnout. Přesné požadavky se liší podle oboru, ale nikde nejsou nízké; po úspěšném absolvování si ale před jméno můžete psát zvláštní zkratky.

\subsubsubsection{Doporučený průběh}

Pokud Vám výše uvedené přijde příliš komplikované, existuje doporučený způsob studia. Ten pro Váš program si najdete v~Karolínce (to je ta bílá tlustá kniha) nebo na webu fakulty. Oproti starším kolegům máte štěstí (nebo smůlu --- záleží na úhlu pohledu), neboť si můžete mixovat dle libosti svoje předměty už od prváku a na doporučený průběh vůbec nehledět, minulé ročníky měly první ročník pevně nalinkovaný. 

Ale pozor, přílišné mixování nemá vždy zcela pozitivní účinky (ať už jde o předměty nebo o alkohol).

\subsubsubsection{Studujme!}
První dva ročníky jsou
pro drtivou většinu Matfyzáků nejnáročnější a slouží jako síto; kdo
je přežije, ten s~vysokou pravděpodobností dostuduje úspěšně
(pozor, podmínka nutná, nikoliv dostačující). Algoritmus pro
přežití si každý konstruuje sám, poznamenejme snad jen, že často
je důležitější se průběžně věnovat škole, než si nechat látku
přerůst přes hlavu a spoléhat na kdovíjaký supertalent.

Komu připadá zaměření jeho fakulty oproti jeho zájmům příliš úzké,
toho může potěšit fakt, že všechny přednášky na UK, jakož i na
většině pražských škol, jsou veřejně přístupné. Bohužel se to
zpravidla netýká pitev a dalších zábavných forem výuky s~omezenou
kapacitou. Přednášky na libovolné fakultě UK si dokonce můžete
regulérně zapsat a dostat za ně kredity (ale asi jenom velmi
zřídka se budou moci započítat mezi povinné nebo povinně volitelné
předměty).

Komu naopak připadá zaměření jeho fakulty oproti jeho zájmům
příliš mělké (podmínka postačující, nikoliv nutná), ten by se měl
již po pohodovém dokončení prváku začít zajímat o další po všech
stránkách nejlepší školy, aby věděl, kam půjde, až bude bakalář,
a stihl si tudíž včas zařídit všelijaké požadované zkoušky
z anglosaské inteligence a financování školného a pobytu.

\subsubsection{Studijní oddělení}

Místo studenty hojně navštěvované. Sem se chodí uzavírat ročníky,
podávat žádosti, loučit se se studiem a mnoho dalších věcí.
Studenty každého oboru má od druhého ročníku na starosti jedna
referentka, prváci mají svoji vlastní (tak ji, prosím, nezlobte).


Fronty, které se na studijním vyskytují, bývají způsobeny tím, že
99,9~\% studentů chodí poslední možný den nejlépe minutu před
koncem pracovní doby. Čekací doby se naštěstí rok od roku zmenšují
díky vylepšování studijního informačního systému (už několik let
funguje elektronický zápis předmětů, počínaje rokem 2009/2010 už
dokonce elektronický zápis plus vlastnoruční zápis do indexu
stačí). Přesto je výhodné vědět, že drtivou většinu věcí (včetně
kontroly studijních povinností) lze vyřídit vložením příslušných
podkladů do zvláštní schránky pro tento účel umístěné v~budově
děkanátu nebo do obálky umístěné ve vrátnicích všech ostatních
fakultních budov (v~tomto případě ale počítejte s~tím, že chvíli
potrvá, než se materiál teleportuje na studijní).

Může se stát, že budete v~nepříjemné situaci a budete na studijní oddělení {\it neslušně ošklivě myslet}. Mějte v~tu chvíli na paměti, že studijní oddělení na matfyzu je jedno z~nejpříjemnějších ze všech fakult v~Česku.

\subsubsection{Karolinky}

To jsou ty dvě knížky, 
bez kterých se na \Mfz{}u neobejdete. 

Existuje karolinka {\it
bílá\/} a {\it oranžová}. Oranžová obsahuje spoustu užitečných
informací o studiu, které by letos měly z~93,512~\% souhlasit se
skutečným stavem (informace je ověřena experimentálně, tedy
koukáním). Bílá je přehledem všech vyučovaných i nevyučovaných
předmětů a vztahů mezi nimi. Většina \mfz{}áků (nejrychlejší
v~tomto smyslu bývají informatici) časem zjistí, že není vůbec
nutné tyto knížky kupovat každý rok, protože vše, co je v~nich, je
také na webu fakulty a ve {\it studijním informačním systému}
(SISu).


\subsubsection{SIS}

Jedná se o webové stránky na adrese \url{https://is.cuni.cz/studium/}, které snad v~budoucnosti nahradí
všechny (nebo skoro všechny) úřední úkony, kvůli kterým je
potřeba, aby student chodil na studijní oddělení. SIS sdílíme se zbytkem univerzity (ačkoliv takoví filosofové ho mají rádi mnohem méně, než my matfyzáci).

Dnes se lze přes SIS pohodlně zapsat na většinu zkoušek, zobrazit
si svůj rozvrh a zapsané předměty, dozvědět se veškeré informace
o~vyučovaných i nevyučovaných předmětech (čímž efektivně odstraňuje
nutnost vlastnit novou Bílou Karolínku), zobrazit své dosud
získané kredity, podle různých kritérii (jméno, ročník, kruh,
zapsaný předmět atd.) najít a zobrazit informace o studentech
a~zjistit celou řadu dalších zajímavých (a~občas i užitečných) věcí.

Občas se kolo zadře, přehodí se špatná výhybka, SIS nefunguje, jak má, a Vy na něj budete jenom {\it neslušně ošklivě myslet}. Chce to se zhluboka nadechnout, on se zase rozjede.

V SISu lze jednou za semestr vyplnit tzv. Studentskou anketu, více se o ní dozvíte v~kapitole o SKASu, který ji pořádá.

\subsubsection{Stipendia}
\Mfz{} je fakulta neobyčejně štědrá, pokud jde o stipendia.

\subsubsubsection{Studijní stipendium}
Nejméně
10 \% z~každého oboru podle prospěchového průměru za předchozí rok
se může těšit z~nemalé \uv{prémie} (loni bylo vypláceno
prospěchové stipendium ve výši 1800~Kč měsíčně). 

K~získání tohoto
stipendia je třeba splnit některé omezující podmínky (studujete na
\mfz{}u poprvé, máte alespoň 60 kreditů). Pozitivní zprávou ale je,
že pokud získáte z~předmětů zakončených zkouškou aspoň 50 kreditů,
jedno vaše zakolísání (dvojka) se Vám smaže.

\subsubsubsection{Účelové stipendium}
Kromě prospěchových stipendií můžete získat též stipendia účelová
jednorázová (za výsledky odborné i sportovní, na podporu výjezdu
do zahraničí) a stipendia účelová pravidelná (např. přihlásíte-li
se jako služba do labu, což Vám ale v~prvním ročníku zcela vážně
nedoporučujeme). 

\subsubsubsection{Ubytovací stipendium}
Dalšími prostředky jsou stipendia ubytovací, které dostávají všichni, kteří se o ně přihlásí (a to i ti, kteří bydlí pěkně doma u maminky). Měsíčně to bývá okolo 600 Kč a posílá se to vždy po 3 měsících, takže první splátka přijde v~prosinci a člověk tak má za co nakoupit pod stromeček.


\subsubsubsection{Granty}
A konečně jsou tu stipendia na SFG, což je
zkratka pro studentské fakultní granty. Grant si můžete vymyslet
sami, ale většinou se využívá nabídky kateder (nejaktivnější jsou
v~tomto směru fyzici). Přihlásit se musíte do 15. listopadu, váš
projekt pak posoudí stipendijní rada. Za splnění projektu můžete
dostat nejenom peníze, ale také kredity.

\subsubsubsection{Další}

Podrobné informace o přidělování stipendií naleznete ve Pravidlech
pro přiznávání stipendií na MFF (stejně jako ostatní vnitřní
předpisy je zveřejněn na internetových stránkách fakulty \url{http://www.mff.cuni.cz/fakulta/predpisy/}).

V~některých souvislostech (např. plánovaná zahraniční stáž) je
rozumné hledat i mimofakultní zdroje podpory: státní či
hostitelské peníze jsou často nabízeny přímo v~podmínkách
výměnného programu, a pokud nejsou, kromě fakultního účelového
stipendia lze žádat též o příspěvek z~univerzitního fondu mobility
či u~několika hodných nadací. Při vyřizování zahraničního pobytu
však budete muset většinu podstatných informací shánět sami, neboť
servis, který fakulta v~této oblasti svým studentům poskytuje, je
zatím celkem malý. Vyplatí se zeptat starších spolužáků, kteří už
do zahraničí vyjeli.

