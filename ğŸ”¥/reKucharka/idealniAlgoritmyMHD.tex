\subsection{Ideální algoritmy MHD}

\def\startpath{B 112, M C (Nádrhol \ra }
\def\ra{$\rightarrow$}

V~následujících doporučených cestách pomocí MHD je použito toto
značení:

\smallskip

\noindent\begin{tabularx}{\textwidth}{ l|X }
     M C (X $\rightarrow$ Y) & metro,  trasa C, z~X do Y \\ \hline
     T 8, 24 (X $\rightarrow$ Y)/2 & tramvaje číslo 8 a 24 z~X do Y a jsou to dvě zastávky \\ \hline
      B 112 &  autobus číslo 112 (zbytek jako u~tramvaje), pokud je
      bez dalšího, znamená B 112 (Pelc-Tyrolka $\rightarrow$ Nádrhol)/2, \\ \hline
      V~221 (X $\rightarrow$ Y) & vlak na trati 221 z~X do Y \\     
\end{tabularx}
\subsubsection{Cesty od koleje 17. listopadu k~budovám fakulty}

\bigskip
\subsubsubsection{Troja (fyzika)} Pěšky po chodníku směrem k~mostu, projít
pod mostem do~nízké modrošedé budovy spojené s~věžákem stejné
barvy --- momentálně rekonstruovaným.

\subsubsubsection{Troja (angličtina)} Pěšky po chodníku směrem k~mostu, před
ním odbočit doleva a po asi dvaceti metrech doprava, projít pod
mostem a rovně do~nízké budovy nespojené s~věžákem stejné barvy.

\subsubsubsection{Karlín} 
\startpath Florenc), T~8, 24 (Florenc \ra
Křižíkova)/2, přejít ulici, budova naproti nebo i pěšky
z~Florence.

\subsubsubsection{Karlov} 
\startpath Pavlák), vylézt směr ulice Na Bojišti,
Karlov, přejít magistrálu, projít kousek po její pravé straně
nahoru směrem k~Vyšehradu, zahnout doprava (ulice Na Bojišti), na
konci doleva, pořád rovně, poslední dvě budovy na pravé straně
jsou \KK{5~a~3} (v~tomto pořadí).

\subsubsubsection{Malá Strana} 
B~112, T~12 (Nádrhol \ra Malostranské
nám.)/8, nahoru přes parkoviště, ta nádherná bílá budova je \Mfz{}.
Tento algoritmus je optimální s.$\,$v. (skoro vždy), ale má-li
T~12 výluku, platí:

\startpath Muzeum), M~A (Muzeum \ra
Malostranská), T~12, 22, 23 (Malostranská \ra Malostranské
náměstí)/1. Mnohdy je ale doprava kolem Malostranské zkolabovaná,
tudíž je mnohdy lepší jít z~Malostranské (častěji na
Malostranskou) pěšky. Metro se doporučuje také v~zimě, když je
doprava omezená sněhem (pak může cesta tramvají trvat i přes
hodinu).

\subsubsubsection {Hostivař} 
\startpath Hlavák), V~221 (Hlavák \ra
Hostivař), pěšky po silnici podél trati směrem na Benešov, dokud
zpoza paneláků nevykoukne SCUK,

Nebo:  \startpath Muzeum), M~A (Muzeum \ra Skalka), B~154, 271
(Skalka \ra
Gercenova), %ono to totiž je: B154(Skalka \ra Gercenova)/8 nebo B271(Skalka \ra Gercenova)/7.
příčnou ulicí (to je Gercenova), vlevo kolem nákupního střediska.

\subsubsection{Cesty mezi jednotlivými budovami fakulty}
\bigskip

\subsubsubsection {Karlov --- Karlín}
 pěšky na Pavlák, M~C (Pavlák \ra
Florenc), T~8, 24 (Florenc \ra Křižíkova)/2, přejít ulici a vejít
dovnitř.

\subsubsubsection{Troja --- kamkoliv}
 jako z~koleje nebo zastávky Kuchyňka
čímkoliv směr Holešovice.

\subsubsubsection{Karlov --- Malá Strana}
 pěšky celou ulicí Ke~Karlovu až
dojdete k~tramvajovým kolejím; tam je zastávka Štěpánská, T~22, 23
(Štěpánská \ra Malostranské náměstí)/6. Bojíte-li se jít na
Štěpánskou, tak pěšky na Pavlák a T~22, 23 (Pavlák \ra
Malostranské náměstí)/7. Hlavně na Pavláku nepodlehněte nacvičené
cestě do metra.

\subsubsubsection{Karlín --- Malá Strana}
 M~B (Křižíkova \ra Národní třída),
T~22, 23 (Národní třída \ra Malostranské náměstí)/4.
