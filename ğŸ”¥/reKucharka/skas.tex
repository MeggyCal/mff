\subsection[SKAS (Studentská komora akademického senátu)]{Studentská komora AS}
Studentská komora akademického senátu (SKAS) zastupuje studenty a jejich zájmy ve vedení fakulty. Má 8 členů, které volíte \emph{Vy, studenti} z~řad studentstva vždy v~květnu pro následující akademický rok. Jejím úkolem je spolurozhodovat o chodu fakulty (má celou třetinu v~akademickém senátě --- viz. kapitola Organizace) a hájit tam především zájmy studentů.

Kromě snahy zvyšovat stipendia a dohlížení nad tím, aby se dalo (po)řádně studovat, pomáhají členové SKASu řešit problémy jednotlivých studentů (zejména ve studijních záležitostech – sporné a zvláštní situace, výklady studijních předpisů, atd.). 

Bližší informace o činnosti a složení SKAS najdete na nástěnkách, které jsou v~každé budově, a nebo na webové stránce \url{skas.mff.cuni.cz}. Pakliže je chcete přímo kontaktovat, stačí použít mail \url{skas@mff.cuni.cz} nebo přes web.

\subsubsection{Studentská anketa}
Důležitou prací je také patronát nad studentskou anketou. Ta slouží k~hodnocení těch, kteří Vás (m)učili poslední půlrok (provádí se vždy ve zkouškovém období). Je dobré ji vyplňovat, neboť tím pomůžete vyučujícím i ostatním
studentům; málokdy dělají vaši cvičící a přednášející chyby
úmyslně. Nebojte se psát k~předmětům slovní připomínky, ať už
pochvalné nebo kritické; věřte, že si je přečtou jak Vaši
vyučující, tak vedení fakulty.

Anketa se dá vyplnit v~SISu (viz. kapitola SIS).
