\subsection{Kolej \17l}


Budovy koleje a menzy \17l jsou jediné budovy z~grandiciózního
re\-ál\-ně-so\-ci\-a\-lis\-tic\-kého plánu na výstavbu
univerzitního městečka, které byly skutečně postaveny. Bývala to
kolej matfyzácká, ale dnes už slouží všem. Pro dobrou dostupnost
ji oceňují hlavně právníci a medici.

Celý komplex je tvořen dvěma bloky koleje (budovy~A a B) a budovou
dnes již bývalé menzy (budova~C). Pod všemi budovami se nachází
rozsáhlý suterén, kde je umístěna nová menza, technické zázemí
koleje a spousta nevyužitých prostor. (I proto sem byl roku 1999
z~Malé Strany přestěhován soudní archív, aby byl vzápětí v~roce
2002 zničen povodní.)

Budovy~A a B jsou v~suterénu spojeny průchozí cestou. Její
absolvování v~podzemním labyrintu bylo podmíněno silným
orientačním smyslem.  Když tam tehdejší vedoucí Ing. Zmrzlík
zabloudil už podruhé, přikázal vyznačit optimální trasu názorným
modrým pruhem. Ten je dnes již značně setřený a odplavený povodní,
ale dá se nalézt.


\subsubsection{Budova A}

Větší z~obou bloků (20 pater + přízemí). Doprava je zajišťována
čtyřmi výtahy (většinou jezdí jenom tři, dvakrát do roka se stává,
že nejede ani jeden) a dvěma požárními schodišti potaženými vysoce
hořlavým linem. Bůh sám ví, co by se dělo, kdyby tu začalo hořet.
Dva malé výtahy mají jednotný přivolávací systém a jejich provoz
je optimalizován, čímž se výrazně snížily čekací doby. Do třetího
a~z~pátého patra se vyplatí jít pěšky (občas se vyplatí jít pěšky
i~z~osmnáctky).

Na Áčku najdete kanceláře (přízemí) a hotelové hosty (1.~patro).
Dříve tu sídlila i vedoucí koleje, ale byla povýšena
a~přestěhována. V~ostatních patrech zabírají něco málo přes dvě
třetiny \mfz{}áci, zbytek jsou studenti jiných fakult~UK.

V přízemí za schodištěm do suterénu je lab, který nahradil původní studovnu. O labu už jste se víc dočetli v~kapitole o labech.

Jedna smutnější informace zde musí také zaznít, ač se tónem nehodí do veselého zbytku kuchařky. V roce 2008 ukončil svůj život student skokem z~19. patra budovy A. Nelze k~tomu dodat nic jiného než --- nedělejte to.

\subsubsection{Budova B}

Menší z~budov (16 pater + přízemí). Kdysi bývalo synonymem vyšší životní úrovně, ale to už nějakou dobu neplatí.

V~podzemí je místnost Kolejní
rady s~klavírem, modlitební místnost (používaná hlavně studenty z~blízkého východu) a cestou k~Áčku hudební
zkušebna. V~přízemí pak sídlí kolejní obhůdek, studovna, televizní místnost a jeden
ze sedmi vedoucích správy KaM (podorgán ředitele KaM).
V~současnosti tuto funkci zastává paní Věra Kronusová.

Doprava je zajištěna dvěma malými výtahy a výtahem požárním. Pravidelně jeden až dva z~nich nejezdí; nejméně poruchový je ten nejpomalejší. Ještě že je tu požární schodiště, pro velký úspěch opět potaženo linem. 


\subsubsection{Bývalá menza (budova C)}

Shora vypadá jako protiatomový kryt a údajně jím také je. Má dvě
podlaží nad zemí a dvě pod ní. Nyní již studentům neslouží, ale po
několika letech nejistého osudu se zdá, že zase sloužit bude. Měla
by se sem (po rozsáhlé přestavbě budovy) přesunou Fakulta
humanitních studií, která dosud vlastní budovu nemá.

\subsubsection{Magistrála}

Vede z~Jižního Města na Prosek. Je opravdovým požehnáním naší
koleje. Ničí naše uši a otravuje naše plíce. Je použitelná na
mnoho způsobů. V~zimních měsících nám poskytuje rozptýlení, když
sledujeme boj silničářů o udržení sjízdnosti této důležité
komunikace. Ve zkouškovém období a při výuce cizích jazyků se
bavíme počítáním projíždějících aut. Navíc poskytuje blahodárný
nízkofrekvenční zvuk přispívající ke klidnému usínání. Její
osvětlení v~noci připomíná řecké písmeno epsilon.

Skoro každé ráno na magistrále vzniká fronta spěchajících řidičů,
pohybující se rychlostí pomalého chodce.  Na psychiku některých
pak působí příznivě, mohou-li zdeptané řidiče pěšky předcházet.

Staveniště v~blízkosti je vyústění tunelu Blanka. Jak zasáhne do
dopravní situace okolo koleje, si netroufají předpovědět ani
finanční matematici.

\subsubsection{Milada}
U koleje býval squat Milada, vhodný pro svobodomyslnější matfyzáky. Poté, co policisté vítězoslavně Miladu dobyli, ji na další 3 roky nechali být a rozpadat se. Stále ale můžete obdivovat umělecká díla na jejích zdech.

\subsubsection{Technické speciality}

Zásadní význam pro dnešní stav koleje mělo rozhodnutí nedokončit
vnější obložení budovy sklem. Takto byly sice zachráněny životy
ubytovaných studentů, kteří by se při spolehlivosti ventilace
jednou určitě udusili, ale nesvědčilo to panelům bez vnější úpravy
a~silně to prodražovalo provoz, protože v~zimě budova bez tepelné
izolace vytápěla i slušnou část okolí. Tato vlastnost sice byla
odstraněna rekonstrukcí před několika lety, jiná specifika však
zůstala.

Také původní okna nebyla ledajaká. Otevírat šlo pouze jedno (kdysi
bylo možné otevírat obě, ale bohužel rámy měly tendenci zcela
svévolně vypadávat, čímž o\-hro\-žo\-va\-ly ubytované studenty jednak při
procházce kolem budov, a pak i nebezpečím vyloučení z~koleje,
kterým se trestá vyhazování věcí z~oken). Okna však nebylo třeba
příliš otevírat, protože většinou větrala sama. Když do několika
pokojů zavřenými okny přes noc nasněžilo, bylo rozhodnuto
o~kompletní výměně oken na obou budovách, a to okamžitě, tedy za
prosincových třeskutých mrazů. Pro ilustraci, práce probíhaly
následovně: každé ráno dělníci vybourali okna v~jednom patře, celý
den běhali po pokojích a kutali do toho, co zbylo ze stěn, a večer
vsadili okna nová.  Patrně se jednalo o novou zdravotní akci
mající za účel donutit studenty k~otužování.  Někteří kolejní
domorodci tvrdí, že takovéto jevy jsou spíše pravidlem než
výjimkou, a na jízlivou otázku, zda se příští rok nebude celou
zimu opravovat topení, hrozí pěstmi a vykřikují: \uv{Radši
neprorokuj! Nebylo by to poprvé.}

Rozsáhlá rekonstrukce v~létě roku 2005 měla zase za následek úplné
vystěhování všech ubytovaných, tedy jev, který někteří studenti
ubytovaní na některých kon\-kré\-tních patrech budovy a dosud nikdy
nezažili.

Zlí jazykové tvrdí, že otevře-li se jistý nadkritický počet oken
nad sebou, kolej spadne. Zatím to však nebylo experimentálně
ověřeno. Sama budova má podobné vlastnosti jako komín. Zatímco
v~nejnižších patrech bývala zima, v~osmnáctém patře nebylo dokonce
i~v~neizolované budově za největších mrazů vůbec potřeba otevírat
topení (problémy byly pochopitelně opačné, topení obvykle nešlo
zavřít). Tento jev byl zachován i u~koleje s~vnější izolací, ale
byl výrazně oslaben.

Meteorologové Vám jistě rádi vysvětlí, proč celá kolej funguje
jako větrná hůrka. Zatímco v~celé Praze fouká jen mírný větřík,
kolem Áčka stojí matfyzáci, kteří se snaží v~silných poryvech
větru alespoň udržet se na nohou. Ti šťastnější občas udělají
i~malý krok dopředu. Celá cesta kolem (kratší!) stěny Áčka trvá až
několik minut, vydrží jen ti nejodolnější.

%V předchozích letech jste však mohli spatřit někoho, jak marně
%zápasí s~dveřmi od menzy, aniž by panoval silný vítr. Obvykle byly
%dveře pouze zamčené. Menza měla totiž čtvery dveře vedle sebe
%a~jejich odemykání a zamykání bylo řízeno velmi důmyslným systémem
%vedoucím k~stejnému opotřebování všech dveří. Aby bylo studentům
%pomoženo, byly všude nalepené šipky s~nápisem {\it vchod}, ale
%protože byly skutečně všude, moc to nepomáhalo.

Zatím poslední rekonstrukce proběhla přes letní období roku 2009
a~týkala se vestibulu obou budov. Týkala se nevyhovujících
požárních předpisů a možnosti unikání požáru z~budovy. Prastaré
památeční dřevěné (a~krásné) vrátnice byly odstraněny, zmizely
i~sedačky, stolky a květiny. S~přestavbou vestibulu byla spojena
hlavně nutnost procházet podzemím do druhé budovy a teprve tamtudy
ven (a~obráceně). Zvláště z~vynášení odpadků se tehdy stalo
dobrodružství. Budova a získala nový vchod a ze starého se stal
evakuační východ (přecejen už skrz něj prošla řádka lidí, ať si
chvíli odpočine). Na autobus je to teď opticky blíž. Novinkou jsou
automatické dveře, oceníte hlavně při stěhování.

Stavební práce na (a~kolem) koleje všeobecně začínají se
zkouškovým obdobím, těžko říci zda náhodou či úmyslem. Ať už to
bylo budování protipovodňových opatření v~zimě 2009, nebo zmíněná
přestavba vestibulů v~létě 2009. O další zdroj hluku se stará výše
zmíněné budování tunelu Blanka.

%\place{Výtahy}
%
%Jsou tu pro Vás. Nadávat na ně lze téměř vždy. Záludností provozu
%požárních výtahů (plechové, nalevo od dvojice malých) je to, že se
%vystupuje na druhou stranu. V~případě zaseknutí je lepší pojistit
%si záchranu i voláním.  Pokud se tlačítko po zmáčknutí nerozsvítí,
%neznamená to nutně, že funguje tlačítko, ale většinou pouze, že
%nefunguje žárovka v~něm. Když výtah nejezdí, nemělo by se tlačítko
%na patře rozsvítit, bohužel to nefunguje vždy, takže nejlepší
%indikací je stoupnout si k~šachtě a poslechnout si, jestli se tam
%něco pohybuje. Je dobrým zvykem slušného kolejáka dodržovat
%několik pravidel, jako třeba nejezdit z/do 1.--5. patra, nemačkat
%svoje patro když je zmáčknuté vedlejší nebo nejezdit z~-1. patra.
%Také je považováno za opravdu neslušné přivolávat všechny výtahy
%(nebojte, on si Vás časem ten jeden vyzvedne a ani Vás nebude
%bavit zastavovat v~patrech naprázdno). Celou výtahovou etiketu
%naleznete na kolejnetu.

\subsubsection{Internet}

Díky neutuchající aktivitě někdejších členů SRK (správní rada
koleje, dnes už jen KR, kolejní rada) a za vydatné podpory SKAS
byl počátkem roku~1996 realizován jeden \mfz{}ácký sen. \Mfz{} se
tak stal třetí fakultou v~republice, která pro své studenty
zajistila na kolejích plné připojení na Internet. V současnosti je ubytovaným k~dispozici gigabitové připojení s~veřejnou IP adresou za 100 korun měsíčně; říká se mu také KolejNet.

Co je třeba k~připojení na Internet udělat?
Zajdete do kanceláře na konci chodby u~ubytovaček, společně se svým OP, studentskou průkazkou, emailem, a MAC adresou síťovky. Nahlásíte, kde bydlíte a která zásuvka bude vaše, a dostanete účet. 
Nevýhodou systému je,
že se už nějak na Internet musíte dostat, protože Vám po něm
příjde heslo a na \url{http://is.ms.mff.cuni.cz} si generujete
platby. Pak stačí nakonfigurovat počítač podle návodu na Kolejnetu
\url{http://www.kolej.mff.cuni.cz} a počkat, až dorazí vaše
platba.


Součástí pravidel Kolejnetu je i (zkratkovitě řečeno) zákaz dělat server zbytku internetu. V překladu to znamená, že nesmíte používat P2P sítě, jako je BitTorrent nebo eMule (tj. to, co činí internet zábavným), ale ani používat Skype (protože ten z~vašeho počítače také udělá server). Technicky zdatnější si nesmí např. nechávat otevřené SSH pro vzdálenou práci na kolejním počítači.

Dále platí 5 GB limit odchozích dat.

Kromě toho platí jakási magická pravidla o odchozích SMTP serverech jako obrana proti spamu. Zeptejte se Dana Lukeše, on Vám je jistě ochotně vysvětlí.

Formálně je síť určena pro studijní účely a nesmíte na ní řešit osobní, natožpak komerční korespondenci, ani navštěvovat webové servery nesouvisející se vzdě\-lá\-vá\-ním. Toto pravidlo však není vynucováno.


\subsubsection{Opravy}

Opravy na naší koleji, kupodivu, probíhají docela rychle. Nejdůležitější
je zapsat poruchu na vrátnici do knihy závad (pozor, pro různé typy závad
existují různé knihy). Opravná akce probíhá asi takto: ráno zapíšete
do sešitu tekoucí sifon u~umyvadla. Kolem poledne se na pokoji objeví
zámečník a ujistí Vás, že na tohle je potřeba instalatér. Areál koleje je
velký, takže než je instalatér nalezen, má po směně. Druhý den ráno se
dostaví instalatér.  Obecně platí, že máte-li jakoukoli závadu (včetně
maličkostí typu ucpaný odpad, nefunkční klíč od skříně či zavzdušněné
topení), napište to do závad. Většinou opravář skutečně přijde.


{\bf Hlavně nic neopravujte sami!\/}

\subsubsection{Kolejní rada}

Kolejní radu, svůj samosprávný orgán, který zastupuje jejich zájmy
proti zájmům vedení koleje, si volí na koleji ubytovaní studenti.
Úroveň webových stránek rady je zřejmě na maximální možné úrovni
kvality a najdete tam i to, co byste nikdy nehledali --- formuláře,
řády a také výtahovou etiketu. Adresa: \url{http://www.kolej.mff.cuni.cz/~kr}.

 \subsubsection{Společné ubytování kluka a holky}

Je možné a dokonce legálně. Odpadají tím v~minulosti běžné problémy
se zamlouváním pokojů, přestěhování \uv{načerno} a následný chaos
v~ubytovacích štaflích (štafle jsou papíry, na kterých je napsáno,
kde kdo bydlí). Po revoluci o tuto výsadu studenti svedli několik
bitev a vyhráli. Někteří mají ještě v~živé paměti, jak museli mít
potvrzení od rodičů, že ve svých pětadvaceti letech mohou bydlet se svou
přítelkyní/přítelem (v~extrémních případech manželkou či manželem). Celé
to spo\-čí\-va\-lo v~jakési vyhlášce ministerstva zdravotnictví ze sedmdesátých
let, která společné ubytování považovala za odporující socialistické
morálce a doporučovala pro každé pohlaví zvláštní budovu; není-li to
z~technických důvodů možné, tak alespoň zvláštní patro.

\subsubsection{Zamlouvání pokojů}

Celá akce probíhá přes Internet někdy koncem srpna, takže se Vás
bude týkat až další rok. Nejdříve dostanou šanci ti, kteří jsou
spokojeni se svým příbytkem a nechtějí se v~příštím roce stěhovat
jinam. Poté si ti, kteří se stěhovat chtějí, vyberou z~toho, co
zbylo, systémem kdo dřív přijde, ten si lépe vybere.

\subsubsection{Co si lze půjčit}

U~vrátného si můžete zapůjčit klíče od prádelny, sušičky či
sušárny (praček je řádově méně, než sušáren), stejně jako vysavač,
žehličku, žehlicí prkno a fén. Zatímco fény si můžete půjčit
najednou klidně čtyři, na prádelny a sušárny bývá fronta. Vyplatí
se mít dobré kontakty s~vrátnými. Získáte je nejlépe tak, že jim
budete dělat společnost při jejich nočních službách.

Pro uschování jízdních kol slouží tzv.
kolárny. U~vybraných jedinců si je rovněž možno zdarma půjčit
i~následující věci: stolní tenis (síťka, pálka, míček), míče
(basketbal, volejbal, fotbal). Podrobnější informace najdete na
nástěnkách či webových stránkách~KR.

\subsubsection{Svoboda na naší koleji}

Je značná, téměř už taková, jakou bychom chtěli. Smí se (přesněji,
je tolerováno) libovolně si přestavět pokoj a vyzdobit si ho dle
libosti. Na základě vyhlášky magistrátu o čistotě hlavního města
Prahy je však zakázána výzdoba oken. Rovněž na chodby a do výtahů
je zakázáno cokoliv vylepovat. Taktéž je zakázáno lepit cokoliv,
nebo, nedej bože, zatloukat hřebíčky do stěn či dveří; ale koho by
ty holé zdi bavily\dots

Za měsíční poplatek můžete mít na pokoji el. spotřebič ---  počínaje
počítačem, televizí a konče třeba rychlovarnou konvicí. Jen na
lednici potřebujete potvrzení lékaře.

Chov zvířat je střídavě zakazován a povolován hygieniky. Většinou
je třeba souhlas alespoň spolubydlícího, občas celé buňky. a pak
samozřejmě vedení koleje. Informujte se v~ubytovací kanceláři.

\subsubsection{Sportování}

V~areálu koleje se nacházejí dvě písková, dnes už převážně
zatravněná, hřiště v~de\-zo\-lát\-ním stavu, dvě hřiště na volejbal nebo
nohejbal, dva basketbalové koše mezi Béčkem a menzou a ještě
tenisové kurty u~řeky, kde ale platíte stejně jako ostatní
(s~kolejí mají společnou jenom polohu), jakési pseudohřiště na
fotbal. Letité pokusy o změnu situace příliš ovoce nepřinesly, což
je způsobeno zejména tím, že v~okolí koleje není zcela jasné, co
komu patří.

Suverénně nejoblíbenějším sportem je volejbal (od května se hraje
skoro každý večer). Každoročně se také pořádá patrový volejbalový
turnaj. Na asfaltě se hraje nohejbal a občas i tenis.

\subsubsection{Kuchařka}
Kolej 17. listopadu má svoji vlastní, podrobnou kuchařku na adrese \url{http://listopad.koleje.cuni.cz/kucharka}.