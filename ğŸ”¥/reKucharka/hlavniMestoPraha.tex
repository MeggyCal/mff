\section{Hlavní město Praha}

Praha je veliká. Nejhorší situace ale nenastávají, když se ztratíte (nejlepší
je nechat se něčím odvézt na metro, pak už trefíte), ale když chcete něco
koupit. Je tady totiž strašně moc obchodů, ale nikdy takový, který zrovna
potřebujete.

Není v~našich silách vypsat všechny dobré hospody a potraviny v~Praze; na to jsou jiné, lepší brožurky a stránky. Berte tuto kapitolu jako spíše doporučení --- a Matičku Stověžatou poznávejte a procházejte hlavně sami (nebo raději ve dvou).

\subsubsection{Slovníček pro nepražáky}

\noindent\begin{tabularx}{\textwidth}{ l X }
   Hlavák &  Hlavní nádraží (též Wilsoňák) \\
   Karlák &  Karlovo náměstí \\
   Kulaťák &  Vítězné náměstí, Dejvická \\
   Masaryčka &  Masarykovo nádraží (některými zvrhlíky zváno Masna) \\
   Mírák &  Náměstí Míru \\
   MS &  Malá Strana \\
   Nádrhol &  Nádraží Holešovice (někdy Holešárna, nebo jen Holešovice) \\
   Opletalka &  menza Jednota \\
   Pavlák, Ípák &  I. P. Pavlova, informatici občas vyslovují [aj pí] \\
   Štrosmajerák &  Strossmayerovo náměstí \\
\end{tabularx}
