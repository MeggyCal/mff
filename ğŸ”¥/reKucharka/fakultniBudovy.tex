\subsection{Fakultní budovy}
\smallskip %smallskip kvůli tomu, že zběsile odečítám místo z headerů a tady se to bije
\subsubsection{Karlov}

\BUAaP{\KK3, \A2; \KK5, \A2}{V~budově \KK3 se označují
písmenem \uv{M} (M1--M3). Budova \KK5 se honosí písmenem \uv{F}
(F1, F2 a KFK).}{Matematici, fyzici, děkanát, studijní.}


Budova \KK3 je centrální nervovou soustavou fakulty. Sídlí zde
děkan, schází se tu vedení, akademický senát i vědecká rada; časté
cesty všech studentů končí hned v~přízemí na studijním oddělení.
Na přednášky sem v~poslední době chodí hlavně matematici
(M1 je největší posluchárnou \mfz{}u, takže se do ní vejde celá
paralelka) a také tu probíhají základní fyzikální praktika.
Dobře zásobený, i když trochu drahý, bufet se vyskytuje
v~suterénu. Na chodbě v~prvním patře také naleznete Galerii
vědeckého obrazu.

V~budově \KK5 sídlí Fyzikální ústav UK. V~posluchárně F1 probíhá
v~zimním semestru \uv{exhibiční přednáška} Fyzika v~experimentech.
Vřele do\-po\-ru\-ču\-je\-me navštívit (a~zatleskejte jim, budou mít
radost). Samotná F1 je krásná posluchárna --- je plná nejrůznějších
fyzikálních hraček. Budete-li v~ní mít přednášku, jistě ji
oceníte.

\subsubsection{Karlín}

\BUAaP{Sokolovská 83, \A8} {Jsou označeny \uv{K}
(K1--K12) + seminární místnosti.} {Matematici}

Výsadní právo na tuto čtyřpatrovou budovu mají matematici, sídlí
tu matematické katedry. V~přízemí je lab a prodejna Matfyzpressu.
Naproti přes ulici je prodejna potravin a pekárna. Napravo od něj
je také čínské bistro, kde je po domluvě možné si jednu přestávku
jídlo objednat a další vyzvednout (hodí se, když máte vyučování
bez pauzy na oběd).

\subsubsection{Malá Strana}

\BUAaP {Malostranské náměstí 25, \A1} {Označují se
písmenem \uv{S} (S1, S3,~\dots, S11), čtyři počítačové učebny (SW1, SW2, SU1, SU2), rotunda, spousta seminárních
místností. Celá budova je často označována jako MS.} {Informatici}


Místnosti se označují písmenem
 \uv{S} (S1, S3,~\dots, S11). Dále je tu rotunda, čtyři počítačové učebny a spousta seminárních
 místností. Celá budova je často označována jako MS.

Informatická budova přilepená k~chrámu sv. Mikuláše, která prošla
náročnou, ale mimořádně zdařilou rekonstrukcí (už se nestává, že
nesvítí světla, protože vypadl jistič o dvě patra výše). Došlo
i~k~rekonstrukci některý maleb na zdech, najdete-li si chvíli,
doporučujeme exkurzi na zadní schodiště.

Malostranská budova je opředena mnoha pověstmi. Například pod
celou budovou se nacházejí opuštěné trezorové místnosti Národní
banky a byl zde docela nedávno nalezen obrovský transformátor, po
kterém rozvodné závody vyhlásily už kdysi dávno celo-malostranské
pátrání. Záhada dodávky proudu do okolních malostranských paláců
obsazených Parlamentem ČR a spol. tak byla konečně vyřešena.

Pověsti, které se o budově mezi lidmi proslýchají, se mnohdy
vzájemně vylučují. Např. ta od RNDr. Kryla, která tvrdí, že \mfz{}
má s~blízkým kostelem sv. Mikuláše společnou jen jednu tlustou
neproniknutelnou zeď, se vylučuje s~historkou, podle níž jednou
v~průběhu oprav chrámu vylezl v~průběhu přednášky z~jedné skříně
v~tehdejší posluchárně S6, zvané saloon, zedník, který zabloudil
ve svatomikulášských tajných chodbách. Pikantní na tom je, že se
tak stalo právě během Krylovy přednášky. Zedník se zadíval na
tabuli, pak polohlasně zahuhlal \uv{A sakra!} a zalezl zpátky do
skříně.

Přepadne-li Vás během dne na Malé Straně hlad, jistě oceníte
stravovací zařízení \uv{Profesní dům} v~suterénu. Vaří tam dobře,
ale \mfz{}áci tam nemají žádnou slevu, takže jídlo stojí kolem
80~korun. Pokud chcete ušetřit, je lepší nosit si svačinu
s~sebou, nebo využít bagetomat. Můžete také zajít do Biomarketu
u~Karlova mostu, navštívit čínské bistro asi 100 metrů po kolejích
ve směru tramvaje 12 nebo okusit potraviny o 20 metrů blíže ve
stejném směru. Pokud máte chuť na kus umělé hmoty, je k~dispozici i McDonald's u Karlova mostu.

Na Malostranském náměstí sídlí také HAMU. Máte-li rádi vážnou hudbu, můžete tak skoro každý večer po škole zadarmo či skorozadarmo na koncert --- stačí být nevandrácky oblečen (oblek ale určitě potřeba není).

\subsubsection{Trója}
\BUAaP{V~Holešovičkách 2, \A8} {Označeny písmenem \uv{T},
(T1--T12, TX, T258 + seminární místnosti na jednotlivých
katedrách, \uv{podivná} trojciferná čísla).}{Fyzici}

Budov je vlastně pět ---  vysoká a čtyři placaté. Placaté se zovou:
\emph{hlavní}, \emph{těžké laboratoře\/}, ~\emph{jazyky a nízké
teploty} a \emph{kryopavilon}, vysoká pak \emph{katedrový objekt}.
Hlavní vchod vede zcela neočekávaně do hlavní budovy. Zde jsou
velké posluchárny T1 a T2 a v~prvním patře malé T3--T11. V~přízemí
je bufet, který si lidé pochvalují, sympatická knihovna
a~samozřejmě neopomenutelný lab. V~zaskleném prostoru bývala
\SKP, další osud tohoto místa je prozatím neznámý. 

Průchodem se
dostanete do vysoké budovy. Ta patří výlučně fyzikům (a~nejen
našim, sídlí zde i FJFI ČVUT). V~letech 2009 a 2010 došlo k~výměně
vnějšího obložení budovy, takže se nemusíte bát, že by na Vás
spadla. K~tomu se také váže historka o tom, že původní obložení bylo otočeno naruby. Sice stavební firma prohlásila, že šlo o náhodu, nicméně členové katedry statistiky spočetli, že náhoda byla spíše to, že to teď dali správně.

Za hlavní budovou jsou těžké laboratoře, v~nichž se nachází, mimo
jiné, \mfz{}ácký {\it Van der Waalsův urychlovač\/} a jaderňácký
reaktor VR1, krycím jménem {\it Vrabec\/} (podbízí se přirozená
otázka, co ve skutečnosti znamená ono rčení \uv{jít s~kanonem na
vrabce}). V~budově položené nejblíže magistrále je Kabinet
jazykové přípravy, učebny T263, T212, T203, TX, T240, fyzici
nízkých teplot, astronomové a další lab.  Celý komplex lze projít
\uv{suchou nohou}, díky důmyslnému systému katakomb, který se pod
ním nachází.
