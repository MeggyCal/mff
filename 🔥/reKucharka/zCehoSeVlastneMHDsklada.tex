\subsection{Z čeho se vlastně MHD skládá}
\smallskip
\subsubsection{Metro (krtek)}

Je páteří celého systému, jezdí od pěti ráno do půlnoci,
respektive trochu déle, protože kolem půlnoci vyjíždějí poslední
spoje z~konečných. Spousta času se ztrácí na jezdících schodech,
kde také platí jistá forma cestovatelské etiky --- v~pravé části
schodů se stojí, v~levé chodí. Až budete v~Praze déle, všimnete
si, že se cestování metrem dá optimalizovat --- tedy v~nástupní
stanici, kdy se na vlak čeká, se připravit na místo, kde je
v~cílové stanici výstup z~metra. Někteří matfyzáci systém ladí
k~dokonalosti, kdy se přesouvají v~rámci vlaku i na mezistanicích.


Metro je ideální pro přepravu na delší vzdálenosti. Pokud jedete
jenom kousek a ve stejném směru jede i tramvaj, je lepší jet
tramvají. V~mnoha případech je výhodnější jít i pěšky. Na~rozdíl
od jiných zahraničních měst (Londýn, New York, ...) je v~pražském
metru příjemně i při největších parnech nebo~mrazech.

Více informací o budování metra, předrevolučních názvech stanic,
futuristických vizích metra v~Praze za 100 let a mnoho dalšího
najdete na zajímavém webu \url{http://www.metroweb.cz/}.

\subsubsection{Tramvaj}

Byla páteří celého systému. Je ideální pro poskakování po městě.
Rychlost závisí především na dopravní situaci. Ve špičce nemůže
tramvaj v~některých úsecích kvůli autům ani projet
(Malostranská--Anděl). Každá linka má svou stálou trasu, ale skoro
pořád je někde výluka, takže se musí dávat pozor a sledovat
vývěsky, kde je v~naprosté většině včas napsáno, k~jakým čachrům
zase došlo. Výluky najdete na informačních tabulích na
nástupištích metra, na tramvajových a autobusových zastávkách ve
formě žlutých tabulek nad jízdními řády a v~tramvaji si občas
můžete vzít letáčky s~aktuální výlukovou situací (najdete je ve
schránkách náhodně rozmístěných po tramvaji, většinou je jedna
hned za kabinou řidiče). Pozor, někdy dojde k~výluce jen v~jednom
směru, to znamená, že člověk stojí na zastávce a když už
v~protisměru projíždí třetí tramvaj očekávané linky, je dobré se na
tabulku výluk podívat a změnit plán.

Jízdní řády nejsou víc jak pět zastávek od konečné příliš
směrodatné. Během dne je pohyb tramvají po městě zcela chaotický,
večer pak platí, že čím je zastávka dál od začátku, s~tím větším
předstihem na ni tramvaj přijede. Z~těchto důvodů je užitečné
chodit tak o pět minut dřív. Pokud jste od konečné zastávky dost
daleko, můžete mít štěstí, že jízdní řád opět začne platit,
protože předstih tramvaje dosáhne intervalu mezi jednotlivými
spoji.

Čísla tramvají jsou od jedničky do dvaceti šesti.

\subsubsection{Autobusy}

Nikdy nebyly páteří systému. Jezdí v~místech, kde nejezdí ani
tramvaj, ani metro, většinou paprskovitě ze stanic (hlavně
konečných) metra. Pokud se zrovna neopravuje silnice, je na jízdní
řád docela spolehnutí. V~autobusech se také vyskytují návaly
(např.~112 v~8.29 na Pelc-Tyrolce). Autobusy mají čísla 100--499
(s~velkými mezerami). Čím větší číslo autobus má, tím dále od
centra systému Vás doveze. Autobusové linky číselných řad 100
a~200 patří pod MHD a není nutné lítačku komukoli ukazovat, ale
i~v~nich se čas od času revizor vyskytne. Příměstské autobusy (linky
číselných řad 300 a 400) mají zabudovány v~pokladně u~řidiče
čtečky karet. Při nástupu je tedy nutné nejen lítačku (OpenCard)
ukázat řidiči, ale v~případě OpenCard navíc je potřeba přiložit
kartu na určené místo na boku pokladny, které je označeno symbolem
čipové karty, a řidič si tak zkontroluje platnost nahraných
kuponů.

\subsubsection{Pantografy neboli žabotlamy}

Jsou to povětšinou ještě modrošedá monstra snažící se zajistit
příměstskou vlakovou dopravu v~okolí Prahy. i když jsou v~poslední
době vytlačovány novými soupravami City Elefant, ještě nějakou
dobu je budou České dráhy nasazovat. Jelikož už nějakou dobu
v~Praze funguje tzv.~{\it Integrovaný dopravní systém}, mělo by
být celkem jedno, čím pojedete, zkrátka lítačka (OpenCard) platí
na metro, tramvaje, městské autobusy, osobní a spěšné vlaky
(a~mizivé množství rychlíků), lanové dráhy a i přívozy na území
Prahy.

Pro \mfk{}y je zajímavá trať z~Hlavního nádraží do Benešova
(v~katalozích ČD má číslo 221). Lze ji použít při cestování do
Hostivaře. Výhodou je, že zatímco metrem a autobusy se
k~Hostivařskému nádraží mlátíte hodinu, vlak tuto trať urazí za
čtvrt hodiny (pokud se po cestě nerozbije). Nevýhodou je, že
takovéto monstrum vyjíždí z~Hlavního nádraží nejvýše jedno za půl
hodiny (intervaly se mění podle denní doby, v~nočních a poledních
hodinách dosáhnou i devadesáti minut). Na vlacích neplatí
ustanovení pražských dopravních podniků o pokutách, jinými slovy
sleva pro sklerotiky se neposkytuje; zapomenete-li si lítačku, tak
pokud to zpozorujete včas a přiznáte se průvodčímu, zaplatíte
jízdné Českých drah s~příplatkem 30 korun; pokud Vás bude muset
průvodčí nebo revizor odhalit sám, je pokuta 800 korun, platí se
ale jinde (při zaplacení na místě se snižuje na 400 korun).

\subsubsection{Lodní doprava}

Kromě nákladních lodí a výletních parníků pro turisty brázdí vody
Vltavy i několik přívozů spadajících pod MHD. (Např. z~Podhoří do
Podbaby, který jde výhodně použít k~dopravě do Suchdolu, kde má
sídlo hnojárna --- Česká zemědělská univerzita.) Přívozy nejsou
v~provozu přes zimu a nevyplují ani v~případě zvýšené povodňové
aktivity. Pro zajímavost, když byla z~důvodů rekonstrukce
vyšehradského tunelu přerušena tramvajová doprava po břehu řeky,
byla zavedena náhradní kyvadlová lodní doprava.

\subsubsection{Noční tramvaje a autobusy --- sběrače mrtvol}

ESM --- elektrický sběrač mrtvol --- tramvaj --- jezdí od~23.45
do~5.00 (přibližně) a to ve stejných kolejích, ale po jiných
trasách než normální tramvaje. Všechny ESM se společně sjíždějí na
stanici {\bf Lazarská} (u~Spálené ulice kousek od Národní třídy),
kde je vždy čas na~přestup i pro ty největší mrtvoly. Interval
mezi tramvajemi je 30~minut, takže je třeba dávat pozor, kde Vás
ten sběrač vyklopí. Mnohým mrtvolám se stává, že dojedou na úplně
opačný konec Prahy, kde je ještě spánkem zpitomělé řidič vyhodí
z~tramvaje a nechá mrznout. Síť ESM podporovaná sítí MSM
(motorového sběrače mrtvol --- nočních autobusů), pro něž platí
trošku jiná pravidla (např.  se nepotkávají a některé z~nich jezdí
v~šedesátiminutových intervalech), obstojně kopíruje trasu metra~C
a~jezdí během noční přestávky metra. Vzhledem k~časové návaznosti
a~dobré noční propustnosti ulic lze tomuto způsobu dopravy přijít
na mnoha trasách na chuť. Například noční transfer z~Jižáku na
kolej 17. listopadu je v~noci mnohem rychlejší než přes den.
Vhodnou stanicí, jak se dostat v~noci z~centra, je I. P. Pavlova,
kde staví noční autobusy jedoucí jak na Kuchyňku, tak na Volhu.

Čísla ESM jsou od 50 do 60, MSM přes 500.