\subsection{Specifika pro Kolej 17. listopadu}
\medskip
\subsubsection{Přístupové cesty na kolej}

Nejdůležitějším bodem, odkud se cestuje na kolej, je stanice metra
Nádraží Holešovice. Odtud se na kolej buď jde pěšky přes most,
nebo jede autobusem. Cesta pěšky trvá dvanáct minut, ale pokud
\emph{opravdu\/} spěcháte, můžete to uběhnout i za pět minut.
Autobusem se jezdí buď linkou~112 na zastávku \emph{Pelc-Tyrolka},
která je koleji nejblíže, nebo na Kuchyňku autobusy s~čísly 102
a~186. Až se budete rozhodovat, jestli jet na Kuchyňku, nebo na
Pelc-Tyrolku, vězte, že průměrnému \mfk{}ovi trvá cesta z~Kuchyňky
o~3--4~minuty déle než cesta z~Pelc-Tyrolky.

Při cestách od Výstaviště (tedy např. ze Štrossmajeráku nebo
Právnické fakulty), se s~výhodou používá tramvaj~14 nebo~17 do
zastávky Trojská (z~koleje: po silnici k~ZOO, až narazíte na
tramvajové koleje). Cesta na tramvaj trvá asi osm minut nebo se dá
využívat stodvanáctka ve směru k~ZOO na stanici Povltavská. Cestou z~Povltavské na kolej občas potkáte veselé bezdomovce.

Každý matfyzák dříve nebo pozdeji objeví, že tramvajové zastávky s~názvem Nádraží Holešovice jsou dvě. A to tehdy, když zrovna vystoupí na té špatné a zmateně se rozhlíží, kde to je. Do odlehlejší zastávky jezdí tramvaje 14 a 17. Buď můžete tramvajemi pokračovat až do zastávky Trojská a přesedlat na autobus 112, anebo se pořádně rozhlédnout, dokud nespatříte budovu nádraží či červenou ceduli označující metro a projít buď tunelem nebo perónem metra na obvyklou zastávku.

\subsubsection {Stodvanáctka}

Je náš kolejní autobus. Jezdí do ZOO. Většinou na ní nejsou
návaly, ale u~některých spojů se z~ní stává induktivní autobus
(když už se tam vešlo $n$ lidí, vejde se jich tam i $n+1$).
Nevejdou se pouze ti, kteří vyměknou. Problémy jsou hlavně, když
se hodně jezdí do Troje za zvířaty. Velice poučné je v~období
školních výletů sledovat marné snahy učitelek na~Holešárně narvat
do stodvanáctky celou třídu najednou. Posledních pár let zavedla
pražská ZOO v~teplých měsících dopravu z~Holešárny do ZOO za
korunu (dříve zdarma) a vyjíždí tzv. Zoobus, který staví až
v~cílové stanici. Bohužel většina cestujících se stejně natlačí do
stodvanáctky, která jede o minutu dříve, a za ní jede Zoobus
prázdný.