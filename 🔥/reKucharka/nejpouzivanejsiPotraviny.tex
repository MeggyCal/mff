\subsection{Nejpoužívanější potraviny}
\begin{itemize}

\item {\it Kolejní obchůdek} v~přízemí budovy B, otevřený denně.

\item Samoobsluha {\it Billa Florenc}, naproti zastávce tramvají
v~budově metra nebo {\it Albert} přes ulici.

\item Samoobsluha {\it Billa Argentinská}, po magistrále směrem od
koleje do centra, asi 500$\,$m za mostem barikádníků.


\item {\it Bílá labuť\/}: jedna je na Florenci (jde se k~ní podél
tramvaje opačným směrem než k~MFF), druhá má vchod přímo z~metra
z~Muzea, má asi do deseti hodin.  Jsou to obecně jedny
z~nejlevnějších potravin v~centru.

\item Samoobsluha obchodního domu {\it Tesco}, hned u~stanice
metra Národní třída (pojmenovaná {\it My}, čtěte to jak chcete)


\item {\it Pražská tržnice\/}: kromě spousty stánků se zeleninou jsou tam
i~samoobsluhy {\it Diskont Plus\/} a {\it Norma\/} (z~metra C Vltavská
jedna zastávka tramvají na zastávku Pražská tržnice)

\item {\it Tesco Letňany\/}: občas některé \mfk{}y popadne
nákupní horečka, vezmou velký batoh a vydávají se na noční
nakupování do Tesca v~Letňanech nonstop otevřeného (bus 186, noční
511 z~Kuchyňky na zastávku Tupolevova nebo metrem na konečnou a
přestoupit).


\end{itemize} 


Na {\it Nádrholu}:
\begin{itemize}
\item Potraviny na {\it Nádrholu}, kousek podél tramvají směrem
k~Výstavišti --- po levé straně. 

\item U~metra v~teplém období funguje také
výborný stánek se zeleninou.

\item Stánky v~místech, kde se vystupuje z~busů MHD --- 
nejsou příliš drahé, a když je hlad, tak se hodí. Smažák mají
výborný.


\end{itemize}
