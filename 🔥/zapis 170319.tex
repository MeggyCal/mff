\documentclass[10pt,a4paper]{article}
\usepackage[utf8]{inputenc}
\usepackage[czech]{babel}
\usepackage[T1]{fontenc}
\usepackage{amsmath}
\usepackage{amsfonts}
\usepackage{amssymb}
\usepackage{multicol}
\usepackage{etoolbox, refcount}
\author{David Nápravník}
\begin{document}

\title{ zapis ze schuze}

\maketitle

\section*{účast}
\begin{multicols}{3}
\begin{itemize}
\item Nápravník
\item Vašek
\item Zuzka
\item Lenka
\item Meggy
\item Vojta
\item Anička
\item Míra
\item Kačka Ž.
\item Martin
\end{itemize}
\end{multicols}

\section{Střídavá péče}
chameleon se bude střídat mezi schůzkami
max 2 měsíce (na náklady dalšího)

\textbf{pořadník}
\begin{multicols}{3}
\begin{itemize}
\item Anička
\item Meggy
\item Míra
\item Marta
\item Kačka
\item Gábi
\item Zuzka
\end{itemize}
\end{multicols}


\section{Další akce}
\begin{itemize}
\item Bojovka 2.4. Nápravník
\item Kolej Cup Kvě-Červ Kačka
\item Běh do schodů říjen Meggy
\item Tour de Pub Lenka
\item Naše akce Anička
\end{itemize}

\section{Žofína}
fronta u vstupu byla velmi dlouhá, až k mostu.\\
zprcat lidi co házeli ksichty na lidi\\
zjistit možnost dvou počítačů\\
nebo nenechat lidi pípat si lístky sami
(neboť lidé netuší kam lístek strkat)\\
nebo mít čtečku v ruce\\
\\
budeme otevírat v sedm a nějak vyřešit
rozložení příchodu návštěvníků\\
\\
rozločit využití čteček spolu s tombolou\\
\\
ubrat počet prodejních míst papírových lístků\\
ubrat počet papírových lístků,
směřovat lístky do el. verze\\
\\
měli jsme 1400 lístků na tombolu, v budoucnu budeme chtít víc,
ale ne zase moc, kvůli předminulému pleu, takže příště
kolem 1500 lístků.\\
\\
chceme 2x více jídla, stejně pití.\\
příště vzít kelímky na pití\\
zařídit chlazení baget před plesem\\
bigband chtěl bagety atd. takže na to příště myslet\\
\\
sehnat mff pivo a víno do tomboli - Vlach\\
\\
nechceme zimní zahradu\\
chceme otevřít další kavárnu?
(Mírovi se to líbí, ale jen pokud zdarma)\\
\\
chceme zrušit stůl pro spolek?
$\leftarrow$ zrušit jeden ze dvou stolů\\
přidat 2 židle k tombole\\
\\
vzít výkonnější promítačku\\
plamínek musí říct kde se dá vyzvednout tombola\\
sehnat krabice pro tombolu\\
\\
osoba hlídající VIP je stále užitečná\\
\\
Žofín nově chce 10.000,- za ubrusy a personál\\
\\
příště vyfotit tabuli s úkoly a dát do konfery/na fb\\
\\
Karolíně budeme dávat nadepsané lístkya budeme to hlídat\\
\\
Míra dostal vynadáno, neb dříve zjistí cenu a včas to nahlásí Aničce\\
Anička chce mít kompletní info o penězích vyplácených na plese\\
a to ještě před plesem\\
\\
V galerii přidáme dva stoly u stěny a
ubereme dva uprastřed, neb byly "nahňácaní".\\
příště přiřazovat kapelám atd. souvislé bloky stolů.
\\
14 dní před plesem se udělá inventura, aby se to mohlo prodávat dál.\\
\\
Udělat plánek Žofína, aby lidé věděli kde co je. (David)\\
\\
\subsection*{Stížnosti na Žofín (Míra)} 
\begin{itemize}
\item smrad v rytířáku
\item balení věcí dřív
\item sebrali krabice od tomboli
\end{itemize}

přes Petera získat lístky na FTVS\\
\\
dřív krátkodobou smlouvu na pronájem - Míra\\
\\
Míra musí vymyslet valnou hromadu a svolat ji měsíc dopředu (květen/červen)  + počítat s nějakým časem na vyvírání času a data\\
 \\
 sehnat projektor\\
 \\
 video z plesu máme, Míra to hodí na
 spolkový YouTube (ano to opravdu existuje) + fotky\\
 \\
 Míra zkontroluje úvodní vide (proslovy), především řeč děkana.\\
 \\
 Anička zjistí zda nejde sehnat kameramana levněji.\\
 \\
 Anička pošle video z udílení cen Kociánovi\\
 




\end{document}