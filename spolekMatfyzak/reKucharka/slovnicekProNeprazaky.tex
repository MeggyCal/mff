\subsection{Slovníček pro Nepražáky}
\begin{longtable}{l p{8.5cm}}
    \textit{Arnošt} & menza Arnošta z Pardubic \\
    \textit{Blanka} &  tunel se zběsile měnícími se semafory, který od září 2015 pravidelně ucpává prostor okolo Koleje 17. listopadu \\
    \textit{Hlavák} &  Hlavní nádraží (C); též Wilsoňák \\
    \textit{Karlák} &  Karlovo náměstí (B), !!!TOTO NENÍ KARLŮV MOST!!! \\
    \textit{Kulaťák} &  Vítězné náměstí, Dejvická (A) \\
    \textit{Lítačka} &  tramvajenka \\
    \textit{Masaryčka} &  Masarykovo nádraží; některými zvrhlíky zváno Masna, pamětníky zváno Střed \\
    \textit{Máj} &  OD MY (Tesco) na Národní třídě \\
    \textit{Mírák} &  Náměstí Míru (A) \\
    \textit{MS} &  Malá Strana \\
    \textit{Nádrhol} &  Nádraží Holešovice (C); někdy Holešárna, Holešky, nebo jen Holešovice \\
    \textit{Národka} &  Národní třída (B) \\
    \textit{Nároďák} &  Národní divadlo nebo národní tým \\
    \textit{Opletalka} &  menza Jednota \\
    \textit{Palačák} &  Palackého náměstí, vedou sem jihozápadní výstupy z metra Karlovo náměstí (B) \\
    \textit{Pavlák, Ípák, Slinták} & I. P. Pavlova (C), informatici občas vyslovují [aj pí] \\
    \textit{pod ocasem, pod koněm} & u sochy sv. Václava na Václaváku, klasicky používaná fráze: "Sejdeme se pod ocasem." \\
    \textit{Smícháč} &  Smíchovské nádraží (B) \\
    \textit{Staromák} &  Staroměstské náměstí - to s orlojem \\
    \textit{Šervůd} &  park před Hlavákem obydlený bezdomovci \\
    \textit{Štrosmajerák, Štros} & Strossmayerovo náměstí \\
    \textit{Václavák} &  Václavské náměstí - to s koněm
\end{longtable}