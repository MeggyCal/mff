\subsection{Zdroje informací}
Ať už hledáte příklady z Děmidoviče (Sbírka úloh a cvičení z matematické
analýzy), kupce na hromadu krabic od pizzy nebo příležitost na balení holek,
existuje mnoho úložišť a informačních kanálů, na které se obrátit. Ty z nich,
které jsou specializované na matfyzácké potřeby, jsou uvedené v následujícím
textu.


\subsubsection{Elektronické zdroje}
\subsubsubsection{Wiki}
Aby to nebylo jednoduché, existuje několik matfyzáckých wiki
\\\\
Na \url{kucharka.matfyzak.cz} právě jste.
\\\\
Na adrese \url{wiki.matfyz.cz} je sbírka nejrůznějších informací od matfyzáků a
pro matfyzáky týkajících se studia (zápisky, poznámky, zadání dřívějších
písemek, projekty, témata diplomek atd.), zkušeností s vyučujícími a předměty,
studia v zahraničí, způsobu trávení volného času v Praze a dalších věcí. Na této
wiki je spousta zastaralých stránek. Užitečné podstránky ale jsou Předměty se
všemi předměty a Vyučující s vyučujícími.
\\\\
Další, kdysi funkční, wiki je \url{mff.lokiware.info}. Teď už není aktulizovaná,
ale dají se tam najít informace užitečné především pro studenty informatiky.


\subsubsubsection{Fórum}
Na forum.matfyz.info je (neoficiální!) fórum fakulty. Neocenitelný zdroj
informací hlavně pro informatiky (ti ho založili), ale také pro matematiky i
fyziky. Dozvíte se tam hlavně užitečné informace ke zkouškám, můžete se poučit
ze zkušeností předchozích ročníků a hlavně sdílejte své zážitky pro budoucí
generace.
\\\\
Kromě informací o zkouškách (včetně zadání), bakalářkách a podobných studijních
otřesnostech tam najdete také nabídky práce, minifórum koleje, SKASu, studijního
oddělení a nebo veselé pozvánky na akce ve volném čase.


\subsubsubsection{SKAS}
Studentská komora Akademického senátu posílá informace o dění na fakultě a v
akademickém senátu formou nepravidelných zpráviček (SKASky), které rozesílá
e-mailem. Kromě toho má web \url{skas.mff.cuni.cz}, kde lze nalézt i aktuální
informace o změnách v předpisech, návody nebo výsledky studentské ankety za
poslední roky.


\subsubsubsection{Spolek Matfyzák}
Na webu \url{matfyzak.cz} se objevují informace o chystaných akcích nebo si tu
můžete objednat matfyzácké tričko, hrníček či placku a samozřejmě také přečíst a
editovat tuto kuchařku.


\subsubsubsection{Facebook}
Pokud nejste odpůrcem tohoto informačního kanálu, doporučujeme sledovat stránky
školy, fakulty, knihovny, dále \textit{Spolek Matfyzák} a \textit{SKAS MFF}.
Bydlíte-li na Koleji 17. listopadu, může pro vás být užitečná ještě skupina
\textit{Koleje 17.listopadu}, případně \textit{Koleje 17. listopadu –
Společenské Deskové Hry} a \textit{KolejBĚH (Kolej 17. listopadu)}.
\\\\
Nastupující prváci každého oboru mají zpravidla založenou skupinu, kde si
sdílejí informace o zkouškách, domácí úkoly a další užitečné věci.


\subsubsubsection{Propagační akce Matfyzu}
Informace o Matfyzem pořádaných akcích (obvykle pro středoškoláky a
základoškoláky, ale nějaké i pro širokou veřejnost), můžete najít na další wiki,
a to ovvp.mff.cuni.cz. Je tam i kontakt, pokud byste se chtěli zapojit do
propagačních akcí jako organizátoři. Propagační články se pak objevují na
\url{matfyz.cz}.


\subsubsubsection{Studnice vědomostí}
Pokud jste informatici a máte přístup k Linuxové laboratoři, na
/afs/ms/doc/vyuka je tzv. studnice vědomostí – různá PDF a další zdroje k různým
předmětům.


\subsubsubsection{Google}
A samozřejmě platí – pokud něco nevíte nebo neumíte, nemusíte se to učit, stačí
vám Google.


\subsubsubsection{Matfyzácké konference}
Jsou mailingovým listem, ve správě odpovídajícího oddělení. Stud-l:
zprostředkovává informace od studijního oddělení a poradenských pracoviště
směrem ke studentům. Budete tak dostávat mailem zprávy o vyhlášení různých
grantů, soutěží, úředních hodinách na SO, ale rovněž o plánovaných seminářích a
akcích pro studenty mimo rámec výuky. Registrovat se však musíte sami
prostřednictvím \url{lists.karlov.mff.cuni.cz/mailman/listinfo/stud-l} Podobný
mailing list má i oddělení zahraničních vztahů, které informuje jeho
prostřednictvím o stipendijních programech, cenách a konferencích. Případně i
knihovny apod. Informace o těchto rozesílacích seznamech naleznete na
\url{psik.mff.cuni.cz}.


\subsubsubsection{Matfyz FAQ}
Máte pocit, že podobný dotaz ze života na Matfyzu už třeba někdo položil? Pak
možná visí i s odpovědí na Matfyz FAQ!


\subsubsection{Nástěnky}
Důležitý zdroj informací pro každého studenta. Na nástěnkách SKASu (všude krom
koleje), kolejní rady (na koleji), oddělení pro vnější vztahy a propagaci
(hlavně nabídky prací a brigád), Spolku Matfyzák (na koleji) či studijního
oddělení (na Karlově) bývají důležité (a zřídka i naprosto nedůležité) informace
týkající se dané oblasti. Na nástěnkách jednotlivých kateder naleznete náměty na
bakalářské a magisterské práce či úmluvy na výběrové přednášky. Na ostatních
nástěnkách se dozvíte, kdo prodá skripta, kdo koho doučí, co promítají pražská
kina, jaká zajímavá zaměstnání pro studenty nabízí různé firmy nebo od koho
levně koupíte kolečkové brusle.


\subsubsection{Knihovna MFF}
Knihovna MFF je rozčleněna na 3 oddělení (matematické, fyzikální a
informatické). Oddělení fyzikální se nachází v budově Ke Karlovu 3 (v prvním
patře), oddělení matematické v budově Sokolovská 83. Nejmladší je oddělení
informatické – sídlí v budově na Malostranském náměstí 25. Zde také najdete fond
knihovny lingvistiky. Další, pro studenty nepostradatelnou knihovnou, je
Půjčovna skript a učebnic – tu naleznete na Troji, v přízemí budovy V
Holešovičkách 2.
\\\\
V knihovně si lze půjčit skripta a učebnice (na 150 dní, ale můžete si je jednou
prodloužit), knížky (tj. \textit{neskripta}, ty se půjčují na jeden měsíc s
možností si je dvakrát prodloužit), čtečky elektronických knih a také flash
disky a kalkulačky. Přehled vašich výpůjček najdete online v Centrálním katalogu
univerzity ckis.cuni.cz. Diplomové práce a časopisy se domů nepůjčují, ale
můžete si z nich okopírovat, co potřebujete. V knihovně v Karlíně a v Troji
funguje velká kopírka na kopírovací karty, které si můžete koupit v dané
knihovně.
\\\\
Knížky v regálech jsou označeny různými barevnými identifikačními štítky. Ty
bílé s písmeny, co jsou nalepeny na hřbetu knihy, jsou signatury (tj. adresy
knih) – podle těch knížky hledáme a v podstatě znamenají to, že si tyto knihy
lze vypůjčit na jeden měsíc. Zelené štítky značí, že si můžete knížku půjčit na
tři měsíce (resp. na celý semestr), a oranžové jsou na knihách, které si odnést
nemůžete a lze je prostudovat jen v prostorách knihovny.
\\\\
Výpůjčky si můžete samozřejmě prodlužovat a také je umožněno si rezervovat právě
nedostupné dokumenty. Upozornění o tom, že si už knížku můžete vyzvednout, chodí
na e-mailovou adresu (k tomu je pochopitelně nutné zadat do SISu správný
e-mail).
\\\\
Dalším důležitým zdrojem informací pro studium jsou nejrůznější databáze –
jejich přehled najdete na stránkách knihovny či na Portálu elektronických zdrojů
Univerzity Karlovy (PEZ). Matfyz na jejich předplácení vynakládá nemalé
prostředky (řádově v milionech korun), tak si jich patřičně važte a hojně je
využívejte. Další možností, kde hledat informace o dokumentech (a to jak
tištěných, tak elektronických), je celouniverzitní vyhledáváč UKAŽ.
\\\\
Na výše uvedené webové stránce najdete i elektronický katalog knihovny (přes
příslušnou ikonku můžete hledat také v katalozích ostatních fakult UK). Při
hledání dokumentu přímo v regále vám ráda pomůže služba u výpůjčky.
\\\\
Registrace do knihovny je zdarma a stačí vám pouze studentský průkaz. Jak se
jednou v některém oddělení Knihovny MFF zaregistrujete, můžete navštěvovat i
všechna ostatní oddělení a dílčí knihovny (meteorologie, astronomie, geofyziky
či Knihovnu dějin přírodních věd). Jako student UK můžete bezplatně využívat
služeb knihoven všech fakult UK.
\\\\
Pro úspěšné fungování v knihovně je nutné vědět, že při překročení stanovené
výpůjční lhůty vám knihovna může nasolit pěknou pokutu. Za každý 1 den a 1
dokument budete platit 3 Kč. Knihovna vás elektronicky upozorní, že se blíží
konec výpůjční doby i jak ji prodloužit. Pokud začnete včas řešit nemožnost
vrácení knížek v termínu, možná se i vyhnete pokutě. Vypůjčené knihy můžete
vracet na jakémkoliv oddělení či můžete k vrácení knih využít biblioboxu před
knihovnou.


\subsubsubsection{Další zdroje literatury}
Občas je výhodné obrátit se i na mimofakultní zdroje, třeba na \textit{Městskou
knihovnu} – ta je na Mariánském náměstí poblíž stanice metra Staroměstská a má
pobočky po celé Praze – nebo na \textit{Národní technickou knihovnu}
(\url{techlib.cz}), která se nachází v krásné budově v Dejvicích.
\textit{Národní knihovna} (\url{nkp.cz}) je umístěna v Klementinu, a pokud jste
vytrvalí, dá se tam sehnat téměř vše. Matematici mohou mít se speciálnějšími
požadavky úspěch v knihovně Matematického ústavu Akademie věd v Žitné.


\subsubsection{Poradenská pracoviště}
Na Matfyzu jsou k dispozici tři poradenské jednotky, určené studentům všech
oborů, ročníků a původu.


\subsubsubsection{Kariérní poradenské centrum (KPC MFF UK)}
Studentům nabízí možnost osobních konzultací, vyhledávání prostoru pro stáže a
internshipy, různé formy pracovních úvazků. Umožňuje najít studentům mentora
\url{ovvp.mff.cuni.cz/wiki/studenti/mentoring/start} Organizuje v průběhu
semestru pracovní veletrhy, semináře, odborné kurzy a exkurze. Aktuální i
uplynulý program naleznete na \url{ovvp.mff.cuni.cz/wiki/studenti/kurzy}.
Kontaktní osobou je dr. Karolina Houžvičková Šolcová. Informace můžete dostávat
rovnou do své schránky, pokud se registrujete na
\url{lists.karlov.mff.cuni.cz/mailman/listinfo/stud-l}. S KPC spolupracuje také
portál \url{careermarket.cz}, kde naleznete řadu dalších aktuálních zpráv a
materiálů vztahujících se k uplatnění studentů a absolventů přírodovědných
oborů.


\subsubsubsection{Program Erasmus}
Českým studentům, usilujícím o výjezd do zahraničí, je k dispozici dr. Ondřej
Pangrác (\url{www.mff.cuni.cz/fakulta/struktura/lide/2624.htm}). Spolupracuje s
Evropskou kanceláří pro program Erasmus na RUK, a proto první dotazy směřujte na
něj. Kontaktní osobou pro studenty přijíždějící ze zahraničí je dr. Kristýna
Kysilková (\url{www.mff.cuni.cz/fakulta/struktura/lide/9901.htm}).
\\\\
Rovněž můžete využívat celouniverzitní International CUNI Club
(\url{www.ic-cuni.cz/}), kde mohou studenti (a nejen) zájemci o Erasmus
procvičovat cizí jazyky zdarma v jazykovém tandemu, dělat Buddy či poznat
studenty z univerzity/města/země , kam se chystají vyjet, a dostat tak nejlepší
informace o místě a univerzitě.. A samozřejmě tam pak už budou mít kamarády, co
jim pak mohou pomoci v novém kulturním prostředí...A ukážou jim nejlevnější
puby, hospody, doporučí nejlepší pivo, atd..


\subsubsubsection{Poradenství pro studenty se speciálními potřebami}
Studenti se mohou obracet na dr. Lukáše Krumpa.
\url{www.mff.cuni.cz/studium/handicap/} Dále je studentům určena Kancelář pro
studenty se speciálními potřebami při UK \url{ipsc.cuni.cz/IPSC-138.html}, která
působí při Informačním, poradenském a sociálním centru UK.