\subsection{Kolej 17.~listopadu}
Budovy koleje a menzy 17. listopadu jsou jediné budovy z grandiózního
reálně-socialistického plánu na výstavbu univerzitního městečka, které byly
skutečně dostavěny (v roce 1988). Bývala to kolej matfyzácká (kromě pater A1 –
A6, které k všeobecné radosti obývaly studentky PedF), ale dnes už slouží celé
UK (a výjimečně i dalším VŠ). Pro dobrou dostupnost ji oceňují hlavně právníci a
medici.

Celý komplex je tvořen dvěma bloky koleje (budovy A a B) a budovou dnes již
bývalé menzy (budova C, nyní patří FHS a probíhá její přestavba). Pod všemi
budovami se nachází dvě patra rozsáhlého suterénu, kde je umístěna nová menza,
technické zázemí koleje, sklady prádla i jiného, tělocvična, hudební zkušebna,
kotelna, dílny a spousta dalších různě využitých či zcela nevyužitých prostor (i
proto sem byl roku 1999 z Malé Strany přestěhován soudní archiv, aby byl vzápětí
v roce 2002 zničen povodní, později byla do stejných prostor situována část
dnešní menzy, zřejmě v přesvědčení, že kastroly příští povodeň neodplaví tak
snadno jako spisy).

Budovy A a B jsou v suterénu spojeny (aktuálně ne)průchozí cestou, zpočátku hůře
nalezitelnou. Když tam tehdejší vedoucí Ing. Zmrzlík zabloudil už podruhé,
přikázal vyznačit optimální trasu názorným modrým pruhem - ten je dnes již
značně setřený a odplavený povodní, ale dá se nalézt. To je ve skutečnosti jedna
z místních urban-legend – pravda je taková, že se v předpovodňové době mezi
budovami chodilo skrz prostor současné menzy a soudobý, pruhem označený, průchod
patřil k nepřístupnému technickému zázemí.


\subsubsection{Budova A}
Budova A je větší z obou budov (20 pater + přízemí), blíže k cestě. Doprava je
zajišťována čtyřmi výtahy (většinou jezdí jenom tři, v době největší potřeby i
méně) a dvěma požárními schodišti potaženými linem. Dva malé výtahy mají
jednotný přivolávací systém a jejich provoz je optimalizován, čímž jsou výrazně
sníženy čekací doby. Na koleji existuje nepsané pravidlo (psané doporučení)
nejezdit výtahem z přízemí do 3. (a nižšího) patra, nepoužívat výtah na přejezd
do sousedních pater a analogicky nejezdit z 5. a nižšího patra do přízemí, také
jezdit do -1 mimo doby oběda není slušné. Kromě toho, že tyto cesty jsou po
schodech opravdu rychlejší, jejich dodržováním si ušetříte vražedné pohledy od
zbytku výtahu, který obvykle jede do dvouciferných pater. Samozřejmě se
připouští výjimky, ať už ze zdravotních důvodů, fyzického vyčerpání či podobně
opodstatněného důvodu.

Okna pokojů jsou orientována buď na východ, s malebným výhledem na magistrálu,
nebo na západ s výhledem na Céčko a tunel Blanka (resp. na kolonu před
portálem). Východní strana je pro velkou část obyvatel populárnější, jelikož v
letních měsících pálí do oken pouze ranní slunce.

Mafyzáci zabírají něco málo přes polovinu koleje, zbytek jsou studenti jiných
fakult UK a dnes už i jiných VŠ. Na dvacátém patře se tradičně, ač většinou bez
povolení, pořádají tak dvakrát do roka chodbovice – párty na chodbě. Aby byli
všichni ušetřeni starostí s oprávněnými stížnostmi, raději se tam vůbec
nestěhujte, pokud víte, že by vám to vadilo.

V přízemí za schodištěm do suterénu je lab, který nahradil původní studovnu.
Více informací o labu najdete na stránce o labech. Za výtahy se nachází
kanceláře.

Jedna smutnější informace zde musí také zaznít, ač se tónem nehodí do veselého
zbytku kuchařky. V roce 2008 ukončil svůj život student skokem z 19. patra
budovy A, obdobně skončil v roce 2016 i život studenta ze 16. patra. Nelze k
tomu dodat nic jiného než – nedělejte to.


\subsubsection{Budova B}
Budova B je ta menší z budov (16 pater + přízemí), kdysi bývala synonymem vyšší
životní úrovně, ale to už nějakou dobu neplatí.

V podzemí je místnost Kolejní rady, klavír, ekumenická místnost, posilovna a
cestou k Áčku hudební zkušebna. V přízemí pak sídlí kolejní obchůdek, studovna,
televizní místnost a jeden ze sedmi vedoucích správy KaM (podorgán ředitele
KaM).

Doprava je zajištěna dvěma malými výtahy a výtahem evakuačním. Relativně často
jeden z nich nejezdí do všech pater (hlavně 15. patro) anebo nejezdí vůbec.
Výtah číslo 5 (více vlevo z dvojice naproti vrátnici) je většinou matfyzáků
považován za neúnosně pomalý. Ještěže je tu požární schodiště, pro velký úspěch
opět potaženo linem. Oproti výtahům na budově A, kde se můžete setkat s cedulkou
„MIMO PROVOZ“, se zde můžete setkat s cedulkou „VÝTAH V PROVOZU“. Skrytý význam
této cedulky ponecháme pozornému čtenáři za domácí cvičení.

Pokoje mají výhled buď na východ, tedy na Miladu, nebo na západ do stráně.
Oproti západní straně budovy A jsou nižší patra západní strany kryta strání před
ostrým sluncem v průběhu večera a v zimním období z nich lze pozorovat rodinky
divočáků.


\subsubsection{Bývalá menza (budova C)}
Shora vypadá jako staveniště a do kategorie čirých pověstí patří, že jím
skutečně je. Aktuálně studentům neslouží, ale měla by se sem (po dokončení
přestavby) přesunout Fakulta humanitních studií (taktéž FHS neboli Fakulta
hledající smysl), která dosud vlastní budovu nemá. Vzhledem k tomu, že jsou
všechny budovy propojené, je hluk ze staveniště intenzivní i v pokojích, které
výhled na staveniště nemají.


\subsubsection{Milada}
U koleje býval squat Milada, vhodný pro svobodomyslnější matfyzáky. Stavba měla
po dobu výstavby koleje sloužit stavební firmě a poté měla být zbourána -- k
čemuž došlo právně, nikoliv však fakticky. Po nekonečných tahanicích je nyní v
majetku Ministerstva školství, mládeže a tělovýchovy. Poté, co policisté
vítězoslavně Miladu na konci června 2009 dobyli, přičemž vážně poškodili do té
doby celkem nepoškozenou střechu, ji na další roky nechali být a rozpadat se.
Stále ale můžete obdivovat umělecká díla na jejích zdech. Podle aktuálních
informací smlouvu o bezúplatném převodu do rukou univerzity, která chce budovu
přestavět tak, aby sloužila studentům, blokuje ministerstvo financí, neboť dle
jeho názoru není dostatečně doložen veřejný zájem.

Každoročně se squateři pokouší uskutečnit pochod k Miladě. Nicméně díky pečlivé
práci (a hlavně značné přesile) policie končí u sjezdu z magistrály. Krom
vystěhování z koleje a výpůjčky věcí je to jedna z mála příležitostí, kdy je
vhodné mít kolejenku u sebe.


\subsubsection{Magistrála}
Vede z Jižního Města na Prosek. Je opravdovým požehnáním naší koleje. Ničí naše
uši a otravuje naše plíce. Je použitelná na mnoho způsobů. V zimních měsících
nám poskytuje rozptýlení, když sledujeme boj silničářů o udržení sjízdnosti této
důležité komunikace. Ve zkouškovém období a při výuce cizích jazyků se bavíme
počítáním projíždějících aut. Navíc poskytuje blahodárný nízkofrekvenční zvuk
přispívající ke klidnému usínání. Její osvětlení v noci připomíná řecké písmeno
epsilon.

Skoro každé ráno na magistrále vzniká fronta spěchajících řidičů, pohybující se
rychlostí pomalého chodce. Na psychiku některých pak působí příznivě, mohou-li
zdeptané řidiče pěšky předcházet.

K magistrále se po nekonečných letech stavby a průtahů připojil i tunelový
komplex Blanka. V odpoledních hodinách před jeho vjezdem a za jeho výjezdem
obvykle vzniká kolona.


\subsubsection{Technické speciality}
Zásadní význam pro dnešní stav koleje mělo rozhodnutí nedokončit vnější obložení
budovy sklem (šlo o ekonomické důvody – buď mohla být skla až nahoru, nebo
telefonní rozvody až na pokoje; nakonec vyhrály telefony). Takto byly sice
zachráněny životy ubytovaných studentů, kteří by se při spolehlivosti ventilace
jednou určitě udusili, ale nesvědčilo to panelům bez vnější úpravy a silně to
prodražovalo provoz, protože v zimě budova bez tepelné izolace vytápěla i
slušnou část okolí. Tato vlastnost sice byla odstraněna rekonstrukcí před
několika lety, jiná specifika však zůstala.

Také původní okna nebyla ledajaká. Otevírat šlo pouze jedno (kdysi bylo možné
otevírat obě, ale bohužel rámy měly tendenci zcela svévolně vypadávat, čímž
ohrožovaly ubytované studenty jednak při procházce kolem budov, a pak i
nebezpečím vyloučení z koleje, kterým se trestá vyhazování věcí z oken). Okna
však nebylo třeba příliš otevírat, protože většinou větrala sama. Když do
několika pokojů zavřenými okny přes noc nasněžilo, bylo rozhodnuto o kompletní
výměně oken na obou budovách, a to okamžitě, tedy za prosincových třeskutých
mrazů. Pro ilustraci, práce probíhaly následovně: každé ráno dělníci vybourali
okna v jednom patře, celý den běhali po pokojích a kutali do toho, co zbylo ze
stěn, a večer vsadili okna nová. Patrně se jednalo o novou zdravotní akci mající
za účel donutit studenty k otužování. Někteří kolejní domorodci tvrdí, že
takovéto jevy jsou spíše pravidlem než výjimkou, a na jízlivou otázku, zda se
příští rok nebude celou zimu opravovat topení, hrozí pěstmi a vykřikují: „Radši
neprorokuj! Nebylo by to poprvé.“

Rozsáhlá rekonstrukce v létě roku 2005 měla zase za následek úplné vystěhování
všech ubytovaných, tedy jev, který někteří studenti ubytovaní na jistých patrech
budovy A dosud nikdy nezažili.

Zlí jazykové tvrdí, že otevře-li se daný nadkritický počet oken nad sebou, kolej
spadne. Zatím to však nebylo experimentálně ověřeno. Sama budova má podobné
vlastnosti jako komín. Zatímco v nejnižších patrech bývala zima, v osmnáctém
patře nebylo dokonce ani v neizolované budově za největších mrazů vůbec potřeba
otevírat topení (problémy byly pochopitelně opačné, topení obvykle nešlo
zavřít). Tento jev byl zachován i u koleje s vnější izolací, ale byl výrazně
oslaben. Další zlepšení v tomto ohledu přišlo v roce 2014, kdy byly instalovány
termoizolační fólie na okna budovy B a horní poloviny budovy A, které srazily
teplotu v letních slunečných dnech o dalších příjemných pár stupňů.

Meteorologové vám jistě rádi vysvětlí, proč celá kolej funguje jako větrná
hůrka. Zatímco v celé Praze fouká jen mírný větřík, kolem Áčka stojí matfyzáci,
kteří se snaží v silných poryvech větru alespoň udržet se na nohou. Ti
šťastnější občas udělají i malý krok dopředu. Celá cesta kolem (kratší!) stěny
Áčka trvá až několik minut, vydrží jen ti nejodolnější. Po několika větrných
dnech i nejtvrdohlavější matfyzák pochopí, proč není vždy dobré otevírat okna na
ventilaci bez ověření, že nebude třeba zapotřebí celé buňky k jeho zavření.

Zatím poslední rekonstrukce proběhla přes letní období roku 2009 a týkala se
vestibulu obou budov. Důvodem byly nevyhovující požární předpisy a možnost
unikání požáru z budovy. Prastaré památeční dřevěné (a krásné) vrátnice byly
odstraněny, zmizely i sedačky, stolky a květiny. S přestavbou vestibulu byla
spojena hlavně nutnost procházet podzemím do druhé budovy a teprve tamtudy ven
(a obráceně). Zvláště z vynášení odpadků se tehdy stalo dobrodružství. Budova A
získala nový vchod a ze starého se stal evakuační východ (přece jen už skrz něj
prošla řádka lidí, ať si chvíli odpočine). Na autobus je to teď opticky blíž.
Novinkou jsou automatické dveře, oceníte je hlavně při stěhování.

Stavební práce na koleji (či v její blízkosti) všeobecně začínají se zkouškovým
obdobím; těžko říci zda náhodou či úmyslem. Ať už to bylo budování
protipovodňových opatření v zimě 2009, zmíněná přestavba vestibulů v létě 2009,
pokládání základů pro budovu pod kolejí v létě 2014, stavba tunelu Blanka anebo
přestavba budovy C.


\subsubsection{Internet}
Díky neutuchající aktivitě někdejších členů SRK (Správní rada koleje, dnes už
jen KR, Kolejní rada) a za vydatné podpory SKASu byl realizován jeden matfyzácký
sen. Budeme-li to počítat podle data oficiálního slavnostního otevření, pak se
Matfyz stal 17. května 1996 třetí fakultou v republice, která pro své studenty
zajistila na kolejích připojení na Internet.

V současnosti je ubytovaným k dispozici symetrické připojení 100 (Mb) na 100
(Mb) za 100 (Kč) s veřejnou IP adresou. Celá kolej pak má desetigigabitové
připojení do PASNETu (agregace je tedy cca 1:10). Tomu celému se většinou říká
prostě \textit{KolejNet}.

\subsubsubsection{Co je třeba k připojení na Internet udělat?}
Jelikož síť provozuje MFF UK, je možnost připojení samozřejmá pro studenty této
fakulty. Pro studenty jiných složek UK je to otázkou výjimky, která je v tomto
případě spíše formalitou. V případě studentů jiných VŠ je to věc individuální
výjimky, která typicky udělena je, automatické to ale není.

Pro všechny kategorie je ale postup stejný. Zajdete do kanceláře na konci chodby
u ubytovaček, společně se svou občankou, studentskou průkazkou, emailem a MAC
adresou síťovky. Nahlásíte, kde bydlíte a která zásuvka bude vaše (je to
jednoduché - pokud při pohledu od dveří k oknu spíte na posteli nalevo, je vaše
zásuvka, při pohledu čelem k zásuvce, ta levá; pro pravou analogicky). Pokud
zásuvku opravdu netušíte, řekněte jakoukoliv (i žádnou), tento údaj si v IS
můžete následně změnit sami, a to nejen před objednáním, ale i kdykoli poté (i
po zaplacení).

Když dodáte, co máte, dostanete účet. Nevýhodou systému je, že se už nějak na
Internet musíte dostat (když bude nejhůř, je hned v přízemí počítačová
laboratoř), protože vám po něm přijde heslo a na \url{is.ms.mff.cuni.cz} si
generujete platby. Pak stačí nakonfigurovat počítač podle návodu na webu
KolejNetu \url{www.kolej.mff.cuni.cz} a počkat, až dorazí vaše platba.

Informační systém sítě vaši platbu zjistí v pracovní den následující po dni, kdy
dorazila na cílový bankovní účet. Kdy platba dorazí na cílový účet, vám řekne
vaše banka (ze které platbu posíláte). Pokud se při placení spletete (chybný
symbol, částka), nepokoušejte se chybu napravit svépomocí – většinou se vám to
nepovede a situaci jen dál zhoršíte. Kontaktujte správu sítě, nejlépe e-mailem,
a nezapomeňte uvést kompletní údaje (tedy název a číslo zdrojového účtu, datum
převodu, částku a všechny symboly) o tom, jak platba doopravdy proběhla, co bylo
špatně a jak to mělo být správně. Údaje o platbě přitom pokud možno čerpejte z
výpisu, nikoliv z příkazu k úhradě (ten mohla banka provést špatně/jinak) a
posílejte je jako text, nikoliv jako obrázek nebo (ještě hůř) přílohu v nějakém
proprietárním formátu.

\subsubsubsection{Pravidla pro používání}
Součástí pravidel KolejNetu je i (zkratkovitě řečeno) zákaz dělat server zbytku
Internetu. V překladu to znamená, že nesmíte používat P2P sítě, jako je
BitTorrent (tj. to, co činí Internet zábavným), anebo si dát pozor na nastavení
Skype, kde je nutné vypnout P2P (jinak si Skype z vás udělá server). Technicky
zdatnější si nesmí např. nechávat otevřené SSH pro vzdálenou práci na kolejním
počítači.

Dále platí 30 GB limit odchozích dat za plovoucí týden. Pokud vás zajímá, jak
velký děláte provoz vy, najděte na stránce KolejNetu vpravo nahoře odkaz Traffic
(a klikněte na něj). Počítá se pouze provoz mimo síť, tzn. v rámci koleje (mínus
lab) lze uploadovat „bez omezení“ (tj. v rámci pravidel).

Kromě toho platí jakási magická pravidla o odchozích SMTP serverech jako obrana
proti spamu. Nebude-li vám to jasné z pravidel, zeptejte se Dana Lukeše, on vám
je jistě ochotně vysvětlí.

Oficiálně je síť určena pro studijní účely a nesmíte na ní řešit osobní,
natožpak komerční korespondenci, ani navštěvovat webové servery nesouvisející se
vzděláváním. Porušení tohoto předpisu není aktivně vyhledáváno. Náhoda je ale
blbec, zejména pokud na sebe upozorníte nějakým (jiným) průšvihem, může to být
přitěžující okolnost.


\subsubsection{Opravy závad}
Opravy na naší koleji, kupodivu, probíhají docela rychle. Nejdůležitější je
zapsat poruchu na vrátnici na lísteček. Opravná akce probíhá asi takto: ráno
zapíšete do sešitu tekoucí sifon u umyvadla. Kolem poledne se na pokoji objeví
zámečník a ujistí vás, že na tohle je potřeba instalatér. Areál koleje je velký,
takže než je instalatér nalezen, má po směně. Druhý den ráno se dostaví
instalatér. Zkonstatuje, že záchod je v pořádku a zanechá vám o tom vzkaz.
Můžete tedy znovu připomenout na vrátnici, že stále protéká sifon. Takto lze
závadu obecně vyřešit, protože instalatérovi po nějaké době dojdou výmluvy a
sifon opraví. Obecně platí, že máte-li jakoukoli závadu (včetně maličkostí typu
ucpaný odpad, nefunkční klíč od skříně či zavzdušněné topení), napište to do
závad. Až na závady vyžadující výměnu – může se stát, že díly nejsou na skladě,
a tedy si budete muset počkat, dokud se neobjednají – je problém obvykle vyřešen
do druhého dne.

\textbf{Hlavně nic neopravujte sami!}


\subsubsection{Kolejní rada}
Kolejní radu, svůj samosprávný orgán, který zastupuje zájmy ubytovaných proti
zájmům vedení koleje, si volí na koleji ubytovaní studenti. Na jejích webových
stránkách najdete spoustu užitečných věcí – telefonní seznam, formuláře,
provozní řády, výtahovou etiketu, otevírací doby, vyhledávání ubytovaných a
simulátor rozestavování nábytku. Adresa: \url{listopad.koleje.cuni.cz}.


\subsubsection{Společné ubytování kluka a holky}
Je možné a dokonce legálně. Odpadají tím v minulosti běžné problémy se
zamlouváním pokojů, přestěhování „načerno“ a následný chaos v ubytovacích
štaflích (štafle jsou papíry, na kterých je napsáno, kde kdo bydlí). Po revoluci
o tuto výsadu studenti svedli několik bitev a vyhráli. Někteří mají ještě v živé
paměti, jak museli mít potvrzení od rodičů, že ve svých pětadvaceti letech mohou
bydlet se svou přítelkyní/přítelem (v extrémních případech manželkou či
manželem). Celé to spočívalo v jakési vyhlášce ministerstva zdravotnictví ze
sedmdesátých let, která společné ubytování považovala za odporující
socialistické morálce a doporučovala pro každé pohlaví zvláštní budovu; není-li
to z technických důvodů možné, tak alespoň zvláštní patro.


\subsubsection{Zamlouvání pokojů}
\sout{Celá akce probíhá přes Internet někdy v srpnu. Nejdříve dostanou šanci ti,
kteří
jsou spokojeni se svým příbytkem a nechtějí se v příštím roce stěhovat jinam.
Poté si ti, kteří se stěhovat chtějí, vyberou z toho, co zbylo, systémem, kdo
dřív přijde, ten si lépe vybere. Obzvláště pokoje na východní straně mizí s
překvapivou rychlostí a vězte, že v 5. minutě je již téměř každá buňka na této
straně obsazena alespoň jedním člověkem.}

\sout{Celé se to netýká studentů, kteří na koleji bydlí i přes prázdniny, neboť
ti si
mohou pokoj, na kterém právě bydlí, zamluvit již v červnu při podpisu smlouvy na
celý další rok.}

Z důvodu příchodu GDPR (a lenosti KaM něco programovat) jsou všichni studenti
ubytováni do stejných pokojů jako v předchozím roce. Noví obyvatelé jsou do
pokojů přiděleni náhodně.


\subsubsection{Co si lze půjčit}
U vrátného si můžete zapůjčit klíče od prádelny či sušárny (praček je řádově
méně než sušáren, navíc pračky se postupem času rozbíjejí), stejně jako vysavač,
žehličku a pumpičku na kolo. Na prádelny a sušárny bývá fronta, občas je lepší
předem na vrátnici zavolat.

Pro uschování jízdních kol slouží tzv. kolárny, které spravuje kolejní rada.
Jsou skoro na každém druhém patře, ale některé pořád plné. Kontakty na správce
najdete na stránkách kolejní rady.

Také máme na koleji knihovnu sci-fi a fantasy literatury. Nachází se ve dvacátém
patře Áčka a je otevřená každé pondělí a středu od 19:00 do 20:00. Více
informací najdete na jejích stránkách knihovna.krakonos.org, nebo na nástěnce
kolejní rady u požárního výtahu na Béčku.


\subsubsection{Svoboda na naší koleji}
Je značná, téměř už taková, jakou bychom chtěli. Smí se (přesněji, je
tolerováno) libovolně si přestavět pokoj a vyzdobit si ho dle libosti. Na
základě vyhlášky magistrátu o čistotě hlavního města Prahy je však zakázána
výzdoba oken. Rovněž na chodby a do výtahů je zakázáno cokoliv vylepovat. Taktéž
je zakázáno lepit cokoliv, nebo, nedej bože, zatloukat hřebíčky do stěn či
dveří; ale koho by ty holé zdi bavily...

Dřívější měsíční poplatek za elektrospotřebiče byl s velkou slávou zahrnut do
ceny kolejného, přirozeně s patřičnou přirážkou. S běžnými spotřebiči tedy
nebývá problém, jen na lednici potřebujete potvrzení lékaře.

Chov zvířat je střídavě zakazován a povolován hygieniky. Většinou je třeba
souhlas alespoň spolubydlícího, občas celé buňky. A pak samozřejmě vedení
koleje. Informujte se v ubytovací kanceláři.


\subsubsection{Sportování}
Z důvodu přerodu ve staveniště se v areálu koleje již nenachází hřiště na
volejbal nebo nohejbal, dva basketbalové koše mezi Béčkem a Céčkem ani jakési
pseudohřiště na fotbal. Zkusit ale můžete tenisové kurty u řeky, kde ale platíte
stejně jako ostatní (s kolejí mají společnou jenom polohu). Letité pokusy o
změnu situace příliš ovoce nepřinesly, což je způsobeno zejména tím, že v okolí
koleje není zcela jasné, co komu patří. Pokud pak hledáte nářadí, např. na
volejbal, pak na vrátnici budovy B najdete věci k zapůjčení.

Pravidelně nefungující a nespolehlivé výtahy inspirovaly studenty k uspořádání
několika forem běhu do schodů. Sportovněji zaměřený (a originálně pojmenovaný)
je „Běh do schodů“, při kterém se běží od menzy do 20. patra na čas (poslední
rekord z podzemí do 20. patra je minuta a 36 vteřin). Druhou takovouto akcí je
„KolejCup“, pravidelně pořádaný během zkouškového, který je vědomostní obdobou,
kdy jsou po koleji rozmístěny otázky a na základě (ne)znalosti se mezi nimi týmy
přesouvají. Obsahem otázek je především učivo z prvního ročníku napříč všemi
obory. Oblíbenou akcí je také „Výstup na K2“, kdy se lezci během
dvacetipatrového výstupu prohřívají v každém patře alkoholem.


\subsubsection{Kuchařka}
Kolej 17. listopadu má svoji vlastní podrobnou kuchařku v režii Kolejní rady.
Najdete ji na adrese \url{listopad.koleje.cuni.cz/kucharka}.