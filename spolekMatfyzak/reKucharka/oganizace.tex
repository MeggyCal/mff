\subsection{Organizace}
Fakulta je dost složitý organismus, takže její strukturu popíšeme pouze stručně.


\subsubsection{Univerzita Karlova}
I když si to mnoho lidí neuvědomuje, Matfyz patří pod Univerzitu Karlovu, o které
jste asi už někdy slyšeli. Kromě nás tam patří filosofové, právníci, biologové a
další nematfyzáci. Vedení univerzity se říká \textit{rektorát}, v jeho čele je
\textit{rektor} -- prof. Tomáš Zima.


\subsubsection{Matematicko-fyzikální fakulta}
V čele fakulty stojí \textit{děkan}, který se ji snaží ukočírovat. Samozřejmě to
nemůže zvládnout sám, a proto má k dispozici kolegium děkana -- do něj patří
například osm \textit{proděkanů}, každý má na starosti některou oblast života
fakulty. Od září 2012 do roku 2020 je děkanem prof. Jan
Kratochvíl.

\subsubsection{Senát}
Dalším klíčovým orgánem fakulty je \textit{Akademický senát MFF UK} (AS), jenž
má 25 členů -- z toho 16 členů tvoří \textit{zaměstnaneckou komoru} (ZKAS) a 9
členů studentskou komoru (SKAS). Senát má značný vliv na většinu podstatných
fakultních záležitostí, mimo jiné volí a odvolává děkana a je potřeba při sestavování fakultního rozpočtu. Na konci každého akademického roku
studenti volí do SKASu tři zástupce na tříleté funkční období. Jednání senátu i
zápisy z jednání jsou veřejně přístupné. Studenti Matfyzu dále každé tři roky
volí své dva zástupce do \textit{Akademického senátu Univerzity Karlovy} (AS
UK), což je něco jako SKAS, jen pro celou univerzitu.


\subsubsection{Sekce}
MFF UK se dělí na tři \textit{sekce}, a to na \textit{matematickou},
\textit{fyzikální} a \textit{informatickou}. Sekce se skládají z jednotlivých
\textit{kateder} nebo \textit{ústavů} a každá má vlastního proděkana. Mimo tyto
sekce stojí \textit{Katedra jazykové přípravy} a \textit{Katedra tělesné výchovy}. Dále jsou na
fakultě tzv. \textit{účelová zařízení} (např. \textit{knihovna fakulty}) a pochopitelně \textit{děkanát},
který se skládá ze spousty oddělení, o kterých běžný matfyzák vůbec neví a ani
vědět nepotřebuje; až na čestnou výjimkou, kterou je oddělení studijní, v jehož
čele stojí \textit{proděkan pro studijní záležitosti} (doc. František Chmelík),
na kterého se nebojte případně obrátit.
