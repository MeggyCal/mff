\subsection{Přidělování koleje}

Kvůli výraznému zvýšení ceny ubytování na kolejích se problém,
který univerzita marně řešila celá léta (tedy nedostatečný počet
míst na kolejích), jaksi vyřešil sám. Obecně již většinou není
problém sehnat ubytování na koleji a různá kritéria (včetně
absolutních) se stala pouhými formalitami, záleží ale také, jakou
kolej chcete. Ta naše má většinou stále mírný přetlak. Čím delší
ubytování máte, tím menší částku za den platíte.

Žádost o kolej podáváte od konce června do začátku července
prostřednictvím systému REHOS. Je-li ji vyhověno, potvrdíte svůj
zájem složením zálohy na účet KaM. Pokud ne, můžete zkusit
námitku. V~půli srpna pak přichází druhá šance na zbytek kapacity
pro ty, co zaspali/zapomněli/neposlali zálohu a pak malé množství
těch, co kolej nedostali z~kapacitních důvodů. To už probíhá
systémem kdo dřív přijde, ten má dříve nárok na ubytování.
Následují rezervace místa na pokoji a pak rezervace dne a hodiny
nástupu na kolej.

Zvýšení ceny kolejného mělo být kompenzováno takzvaným stipendiem
(pří\-spě\-vkem) na ubytování, které se rozděluje plošně mezi všechny
studenty, kteří splní určitá základní kritéria (např. prezenční
forma studia, první studijní program atd.). Vyšší částka, kterou
dostávají sociálně potřební studenti, je celkem rozumná (cca 1400
Kč/měsíc), částka pro ostatní zhruba poloviční (cca 650 Kč/měsíc),
záleží na výši prostředků pro daný rok.