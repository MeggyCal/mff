\subsection{Nefakultní budovy}
Kromě budov, ve kterých probíhá výuka těch podřadnějších předmětů, jako je
matematická analýza, navštívíte i budovy jiné. Nejdůležitější z nich je SCUK
Hostivař, kde probíhá většina tělocviků. Dalšími budovami jsou například
Karolinum – historické sídlo univerzity, které navštívíte při imatrikulaci – a
nebo loděnice v Malé Chuchli, kde probíhá výuka kanoistiky.


\subsubsection{SCUK (Sportovní centrum UK) Hostivař}
\budovaKratce{Bruslařská 10, 102 00 Praha 10 - Hostivař}
{Tělocvikáři}
V prvním patře (to je to, do kterého vejdete hlavním vchodem) této budovy sídlí
katedry tělesné výchovy (KTV) několika fakult. Zajišťují výuku tělocviku a
starají se o příležitosti ke sportování. Naše KTV je samozřejmě ta úplně vzadu.

okud hledáte šatny, u většiny sportů půjdete správně, když za vchodem sejdete
po schodech a dáte se do chodby vlevo (dámské jsou pak vpravo, pánské vlevo). Na
bazén si oproti ISICu vyzvedněte visací zámeček u okýnka pod schody a jděte do
šaten za okýnkem. Na ostatní sporty byste měli mít zámek vlastní.

Tato budova, nebo spíše její umístění, se stane důvodem, proč budou někteří z
vás nenávidět tělesnou výchovu. Budova se totiž nachází na opačném konci Prahy
než zbytek matfyzáckého světa.


\subsubsection{Karolinum (rektorát UK)}
\budovaKratce{Ovocný trh 5, 116 36 Praha 1}
{Byrokraté}
Rektorát fakultě pochopitelně nepatří. Je to něco jako náš děkanát, ale „o patro
výš“. Ve staroslavném \textit{Karolinu} se konají dva obřady ohraničující
studium: \textit{imatrikulace} (tu může zažít každý) a \textit{promoce} (tu
jenom ti vytrvalí).

Imatrikulace je slavnostní přijímání do stavu studentského (neméně důležité je
však přijímání do stavu matfyzáckého, které vás čeká v rámci Beánie).
Neodmyslitelnou součástí imatrikulace je dobloudění do Karolina na vlastní pěst,
hrstka lidí pravidelně dorazí pozdě. Vyrazte z domova (či z koleje) raději o
několik hodin dříve, a pokud už opravdu nevíte jak vchod do Karolina najít,
zkuste chodit dokola kolem Stavovského divadla. Jistě vám přijde něco nápadné.


\subsubsection{Loděnice Malá Chuchle}
\budovaKratce{Strakonická Praha 5 - Malá Chuchle}
{Tělocvikáři}
Na ty, kteří si zapíší kanoistiku, čeká první měsíc zimního semestru a poslední
měsíc(e) semestru letního dojíždění do Malé Chuchle (v zimě se s kajaky nacpete
mezi plavce do bazénu v Hostivaři). I když se to tak nezdá, Malá Chuchle je
ještě v pásmu MHD, na které vám stačí lítačka. Nejlepší (a snad jediné rozumné)
spojení je autobusem ze Smíchovského nádraží. Po vystoupení z autobusu na
zastávce Malá Chuchle je (při příjezdu od Smíchovského nádraží) nutné podejít
podchodem silnici, pokračovat rovně až k řece a vejít do budovy s modrým plotem,
která se nachází na pravé straně. Cyklisté můžou ze Smíchovského nádraží využít
též cyklostezku, která vede proti proudu Vltavy až k loděnici. Při první cestě
je ale vhodné mít s sebou někoho, kdo ví, jak loděnice vypadá.

V loděnici jsou uskladněny kanoe, rychlostní kajaky a veslice, vše je možné
půjčit a vyzkoušet po domluvě s tělocvikářem.