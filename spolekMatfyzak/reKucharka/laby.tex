\subsection{Laby}
Počítačovým laboratořím říká každý správný matfyzák pouze \textit{lab} a na
Matfyzu jich je naštěstí relativně dost. V každé budově je nejméně jeden takový,
do něhož mají přístup normální uživatelé (studenti Matfyzu).
Kromě toho existuje ještě spousta katedrových počítačů, u kterých sedávají
studenti vyšších ročníků.

K přístupu do labu potřebujete zpravidla průkaz studenta a uživatelské konto.
Potřebujete-li založit konto, procedura bývá taková, že si člověk zažádá u
služby v daném labu, žádosti je zpravidla vyhověno a často okamžitě je konto
založeno. V Troji a na Karlově má přístup k počítačům automaticky každý student
univerzity. Přihlásíte se tam pomocí stejných údajů jako do SISu. Někdy je pro
založení účtu potřeba odchytit správce a absolvovat školení o specifikách daného
labu.

Obsazenost labů závisí na denní době, fázi semestru a poloze v Praze. Nejplnější
bývají laby v zimním semestru v období psaní zápočtových programů. Nejvíce místa
je obvykle na Karlově, jelikož zdejší dav není lehké najít. Nezapomeňte se řídit řádem daného labu a
směrnicí děkana č. 4/2008, ze které např. vyplývá, že se máte k počítačům chovat
slušně a že nesmíte používat nelegální software ani psát vulgární e-maily.

V každém labu je tiskárna, tisk stojí jednotně jedna koruna za černobílou stranu
A4 a 5 až 15 korun za barevnou. Ne všude je barevná tiskárna. Ve většině labů
najdete i scanner.

Každý lab má svoji webovou stránku, rozcestník najdete na: \url{https://www.mff.cuni.cz/cs/vnitrni-zalezitosti/it-a-sluzby/pocitacove-laboratore}.


\subsubsection{Malá Strana}
Malostranský lab se nachází v prostorách zvaných Rotunda (z hlavního vchodu
pořád rovně, než se po levé straně objeví mohutné chromované dveře vedoucí do
velké místnosti). Jde o prostor, který byl v roce 1927 přistavěn jako
reprezentativní dvorana Ústřední státní pokladny. Lab je doslova obklopen
knihovnou a je opravdu nádherný. Navíc je tam ve všech ročních obdobích příjemná
teplota.

V pravé půlce labu jsou vám k dispozici počítače s Windows 7, v levé převážně
Gentoo Linux (oba systémy sdílejí váš domovský adresář, takže můžete libovolně
přecházet). Vzdáleně se můžete přihlásit na Solaris. 

Účty se tu zakládají centrálně v pevné dny a časy. Konzultujte se službou nebo
nástěnkou před labem.

\subsubsection{Karlov}
Na Karlově je problémem vůbec lab najít. Vězte tedy, že se nalézá v budově Ke
Karlovu 3 v téměř nejzapadlejším koutě podzemí za dveřmi číslo M -145. K labu vedou
šipky, které se vyplatí následovat, nechcete-li v karlovském podzemí zůstat.

Na každém počítači si můžete vybrat mezi systémem Windows 7 a Ubuntu.


\subsubsection{Karlín}
Lab (K10) se nachází v přízemí budovy, hned vpravo od vchodových dveří. V labu
je možno získat účet na linuxovém serveru Artax. Tento účet vám umožní vzdálenou
práci přes SSH, práci na lokálních stanicích a také přihlašování k Windows. V
labu existuje též server Atrey, kde se účty až na výjimky nezakládají. Na všech
počítačích je možné pracovat pod MS Windows i pod Linuxem. Softwarové vybavení
laboratoře pokrývá potřeby matematiků.

Kromě tohot labu jsou na Karlíně ještě další místnosti s počítači, kde probíhá výuka (K4 a K11).


\subsubsection{Troja}
LabTF se od ostatních labů na Matfyzu liší tím, že je ve dvou od sebe vzdálených
místnostech. V přízemí v místnosti T007 sedí služba.

Druhá místnost je v 1. patře. Na rozdíl od spodní místnosti tady probíhá jenom
výuka. Na všech počítačích je možné pracovat pod MS Windows 7. Na některých PC
je taky možné pracovat i pod Linuxem (Ubuntu). 

\subsubsection{Kolej 17. listopadu}
Na koleji býval velmi oblíbený lab otevřený nonstop, ve kterém někteří studenti dokonce nocovali. Bohužel v rámci rekonstrukce koleje a jejího okolí tento lab přechází pod správu KAMu, takže v něm drasticky ubylo počítačů a hrozí jeho zrušení.


\subsubsection{Alternativy}
Situace, kdy má počítačová laboratoř zavřeno a spolubydlící si zabral buňku pro
radostné trávení času s přítelkyní, jsou časté a může se stát, že budou
kolidovat s chvílemi, kdy opravdu potřebujete pracovat. Máte-li notebook, můžete
využít noční studovnu v Národní technické knihovně v Dejvicích. Je otevřená
kdykoliv, kdy není otevřená vlastní knihovna, je v ní k dispozici WiFi,
elektrický proud, záchod a automaty s pitím a bagetami. Registrovat se ale je
potřeba někdy v běžných otevíracích dobách – stojí to padesát korun na rok.
