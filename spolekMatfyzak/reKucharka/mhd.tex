\subsection{MHD}
Jednotlivé budovy MFF jsou rozeseté po celé Praze, a tak jednou z běžných
činností matfyzáka je přejezd z jedné budovy do druhé. Týdně ujetou průměrnou
vzdálenost navíc zvyšuje dojíždění na tělocvik do SCUK Hostivař, které je
umístěno tak, aby to k němu měli všichni studenti UK stejně daleko (tzn. co
nejdále). Jelikož pražská MHD patří k nejlepším městským hromadným dopravám na
světě, je velmi výhodné ji využívat - dobrá znalost MHD umožňuje ušetřit až
hodinu času denně. Pokud budete jezdit na kole, leckdy ušetříte hodiny dvě.
Přestože k dokonalosti a maximální časové efektivitě se dá dopracovat až praxí,
nabízíme zde několik tipů, jak se zpočátku neztratit.
\\\\
Kompletní jízdní řády a další informace o MHD jsou na webu Dopravního podniku
\url{www.dpp.cz/}
\\\\
Výhodné je používání aplikace IDOS, ke stažení ve verzi pro Android i iOS.
Rozšířené nastavení IDOSu navíc umožňuje změnu délky přestupu - ta se u studentů
MFF běžně snižuje až na 0 minut.


\subsubsection{Jízdné}
Protože matfyzáci při cestách do školy (a jinam) obvykle několikrát denně
křižují Prahu, je téměř nezbytnou pomůckou nějaký jízdní kupón, který je
výhodnější než jednorázové jízdenky. Měsíční studentský kupón se vyplatí již při
týdenním používání. Kupón můžete mít buďto v papírové, nebo v elektronické
formě.
\\\\
Studentský kupón na měsíc od října 2018 stojí 130 Kč, čtvrtletní je za 360 Kč,
roční se prodává za 1280 Kč. Prodávají se v centrále DPP Na Bojišti (po cestě z
Karlova, přes poledne tu nebývají fronty), v informačních centrech DPP ve
stanicích metra Muzeum, Anděl, Můstek a Nádraží Holešovice. Další prodejní místa
v metru lze nalézt na stránkách
\url{www.dpp.cz/jizdenky-prodejni-mista-v-metru/}.
\\\\
Dříve byly měsíční a čtvrtletní kupóny vázány ke kalendářním měsícům. Od 13.
června 2010 záleží jen na datu pořízení kupónu, jsou tedy flexibilnější.
\\\\
Nárok koupit si MHD kupón se studentskou slevou máte celý rok, tzn. i o
prázdninách. Ale pozor, pokud si kupujete kupón přesahující do dalšího
akademického roku, tj. od října dále, musíte mít nové potvrzení o studiu.

\subsubsubsection{Papírový kupón}
K papírovému kupónu je nezbytně nutný studentský průkaz (ISIC s platnou
revalidační známkou pro nový rok nebo školou vyplněný speciální formulář
dopravního podniku), který opravňuje lítačku koupit a bez kterého je lítačka
neplatná. Při kontrole pak revizoři vždy vyžadují mít studentský průkaz s sebou.

\subsubsubsection{Elektronický kupón}
Kromě průkazky v MHD slouží Lítačka také jako průkazka do Městské knihovny v
Praze. Studenti si však Lítačku pořizovat nemusí, mohou využít papírové varianty
kupónu MHD, ale možné jsou obě varianty. Stejně jako u papírového kupónu by měl
mít student vždy při sobě potvrzení o studiu, ale v praxi se nekontroluje. Od
září 2018 je možné elektronický kupón na MHD dobít nejen na Lítačku, ale i na In
Kartu (ČD), nebo na bezkontaktní bankovní kartu.
\\
\textbf{Dobíjení}
\\
První nabití kupónu je nutné provést na kontaktním místě DPP s doložením
studentského průkazu. Další nabíjení je možné jak na kontaktních místech, tak i
prostřednictvím internetu na stránkách e-shopu DPP. Každý srpen až říjen je
potřeba znovu ukázat potvrzení o studiu na libovolném kontaktním místě, do té
doby nejde koupit zlevněné jízdné přes e-shop.

\subsubsubsection{Dopis o přijetí pomůže se studentským jízdným před zápisem}
Platnost potvrzení o studiu ze SŠ končí dle DPP již k poslednímu dni června nebo
srpna. Pokud si budete chtít koupit kupón, který toto datum přesahuje, je nutné
se osobně dostavit k pokladně nejlépe s papírovým potvrzením o přijetí. Toto se
může hodit i v létě mezi bakalářem a magistrem.

\subsubsubsection{OpenCard}
Slavná OpenCard měla od 31. prosince 2011 úplně nahradit obyčejné kupóny a stát
se kartičkou využívanou nejen na MHD, ale Praha už ji zrušila; nahrazena byla
Lítačkou.

\subsubsubsection{Obyčejné (jednorázové jízdenky) jízdenky}
Někdy se kupovat celý kupon nevyplatí, zapomene se po Vánocích koupit nový a
nebo prostě jen přijedou příbuzní z Moravy a chtějí poradit, jaké lístky si mají
koupit. Klíčové jsou čtyři základní možnosti (pro dospělé osoby).
\begin{itemize}
\item Základní jízdenka (cena 32 Kč) --- platí na všechny dopravní prostředky
MHD po dobu 90 minut.
\item Krátkodobá jízdenka (cena 24 Kč) --- přestupní, platí jen 30 minut.
\item Celodenní jízdenka (cena 110 Kč) --- platí 24 hodin, označuje se jen
jednou.
\item Třídenní jízdenka (cena 310 Kč) --- platí 72 hodin, označuje se jen
jednou.
\end{itemize}
Všechny čtyři typy jízdenek si lze koupit nejen v trafice či automatu, ale také
s pomocí mobilního telefonu – po zaslání klíčového slova \textbf{DPT} a ceny
jízdenky (tedy \textbf{DPT32}, \textbf{DPT24}, \textbf{DPT110} nebo
\textbf{DPT310}) na číslo \textbf{90206} vám během chvilky v SMS přijde kód,
který v případě kontroly ukážete revizorovi. SMS jízdenky platí jen v centru
města a neplatí ve vlacích.

\subsubsubsection{Revizoři}
Pokud vás revizor zastaví, nepokoušejte se utéct, většinou jich je víc a
spolupracují s městskou policií. Když vás chytí revizor bez lístku, je pokuta
1500 Kč; pokud zaplatíte na místě nebo ve lhůtě stanovené ve smluvních
podmínkách DPP („nejpozději 15. kalendářní den ode dne kontroly“), platí se
jenom 800 Kč. Pokud jste si lítačku jenom zapomněli a přivezete ji ukázat
(příští pracovní den od 12:30 až 15. kalendářní den ode dne kontroly) na
centrálu Na Bojišti, zaplatíte jen 50 korun.


\subsubsection{Části MHD}
V Praze jezdíme metrem, tramvají, autobusem, lanovkou, přívozem, vlakem, na kole
a na koloběžce. S platnou lítačkou většinou ani neplatíme pokuty.

\subsubsubsection{Metro}
Metro je páteří celého systému, jezdí od pěti ráno do půlnoci, respektive trochu
déle, protože kolem půlnoci vyjíždějí poslední spoje z konečných. Spousta času
se ztrácí na jezdicích schodech, kde také platí jistá forma cestovatelské
etikety – v pravé části schodů se stojí, v levé chodí (resp. běží). Až budete v
Praze déle, všimnete si, že cestování metrem se dá optimalizovat – v nástupní
stanici, když se na vlak čeká, se můžete připravit na místo, kde bude v cílové
stanici výstup z metra. Někteří matfyzáci systém ladí k dokonalosti, kdy se
přesouvají v rámci soupravy i na mezistanicích.
\\\\
Metro je ideální pro přepravu na delší vzdálenosti. Pokud jedete jenom kousek a
ve stejném směru jede i tramvaj, je lepší jet tramvají, pokud vám zrovna jede. V
pražském metru je příjemně i při největších parnech a mrazech.
\\\\
Více informací o budování metra, předrevolučních názvech stanic, futuristických
vizích pražského metra za 100 let a mnoho dalšího najdete na zajímavém webu
\url{www.metroweb.cz}.

\subsubsubsection{Tramvaj}
Tramvaje byly páteří systému před vybudováním metra, které mělo původně fungovat
jako podzemní tramvaje. Jsou ideální pro poskakování po městě. Rychlost závisí
především na dopravní situaci. Ve špičce nemůže tramvaj v některých úsecích
kvůli autům projet (zejména Malostranská - Újezd a obráceně). Každá linka má
svou stálou trasu, ale skoro pořád je někde výluka, takže je třeba dávat pozor a
sledovat vývěsky, kde je téměř vždy napsáno, k jakým čachrům zase došlo. Výluky
najdete na informačních tabulích na nástupištích metra, na tramvajových a
autobusových zastávkách ve formě žlutých tabulek nad jízdními řády nebo přímo ve
vozidlech nad okny.
\\\\
Čísla denních tramvají jsou od jedničky do dvaceti šesti, ne všechna se ale
používají.

\subsubsubsection{Autobusy}
Autobusy jezdí v místech, kde nejezdí ani tramvaj, ani metro, většinou
paprskovitě od stanic metra (hlavně konečných). V autobusech se také vyskytují
návaly. Autobusy mají čísla 100 až 670. Autobusové linky číselných řad 1xx a 2xx
patří pod MHD a není na nich nutné lítačku komukoli ukazovat, ale i v nich se
čas od času vyskytne revizor. Příměstské autobusy (linky číselných řad 3xx, 4xx
a 6xx) mají zabudovány v pokladně u řidiče čtečky karet. Při nástupu směrem z
Prahy je nutné lítačku přiložit na určené místo na boku pokladny a řidič si tak
zkontroluje platnost nahraných kuponů. Příměstské autobusy jedoucí směrem do
Prahy se na území Prahy chovají většinou normálně – jízdenky se neukazují,
nastupuje se všemi dveřmi. Od září 2012 jsou některé linky s krátkými intervaly
(včetně linky 112) označené jako metrobusy a považovány za páteřní linky.

\subsubsubsection{Vlaky}
Součástí Pražské integrované dopravy jsou i vlaky (tzn. platí v nich lítačka),
ale ne všechny – pouze linky S a R – což je většina osobáků a spěšných vlaků,
které po Praze dneska jezdí, a některé vybrané rychlíky. V současnosti jsou
spoje obsluhovány novými a pěknými soupravami City Elefant.
\\\\
Pro matfyzáky je zajímavá trať z Hlavního nádraží do Benešova (v katalozích ČD
má číslo 221). Lze ji použít při cestování do Hostivaře. Výhodou je, že zatímco
metrem a autobusy se k hostivařskému nádraží mlátíte hodinu, vlak tuto trať
urazí za třináct minut. Nevýhodou je, že takovéto monstrum vyjíždí z Hlavního
nádraží obvykle jednou za půl hodiny (intervaly se mění podle denní doby). Je
třeba myslet na to, že na Hlaváku odjíždějí ze zadních nástupišť a čas na
přeběhnutí z metra se blíží dvěma minutám.
\\\\
Ve vlacích neplatí ustanovení pražských dopravních podniků o pokutách, jinými
slovy sleva pro sklerotiky se neposkytuje - zapomenete-li si lítačku, tak pokud
to zpozorujete včas a přiznáte se průvodčímu, zaplatíte jízdné Českých drah s
příplatkem 40 korun.

\subsubsubsection{Lodní doprava}
Kromě nákladních lodí a výletních parníků pro turisty brázdí vody Vltavy i
několik přívozů spadajících pod PID, např. z Podhoří do Podbaby, který lze
výhodně použít k dopravě na Suchdol, kde má sídlo hnojárna (Česká zemědělská
univerzita). Přívozy nejsou v provozu přes zimu a nevyplují ani v případě
zvýšené povodňové aktivity. Pro zajímavost, když byla z důvodů rekonstrukce
Vyšehradského tunelu přerušena tramvajová doprava po břehu řeky, byla zavedena
náhradní kyvadlová lodní doprava.

\subsubsubsection{Noční tramvaje a autobusy}
Noční tramvaje jezdí přibližně od 23:45 do 5:00 a to ve stejných kolejích, ale
po jiných trasách než normální tramvaje. Všechny se sjíždějí na zastávce
Lazarská (u Spálené ulice kousek od Národní třídy), kde je vždy čas na přestup.
Přestupních bodů je však více, vyplatí se věřit IDOSu. Interval mezi tramvajemi
je 30 minut, o víkendu 20, takže je třeba dávat pozor, kde vás vyklopí. Mnohým
spáčům se stává, že dojedou na úplně opačný konec Prahy, kde je ještě spánkem
zpitomělé řidič vyhodí z tramvaje a nechá mrznout. Síť nočních tramvají je
podporovaná sítí nočních autobusů, pro něž platí trošku jiná pravidla (např. se
nepotkávají a některé z nich jezdí v hodinových intervalech). Vzhledem k časové
návaznosti a dobré noční propustnosti ulic lze tomuto způsobu dopravy přijít na
mnoha trasách na chuť. Například noční transfer z Jižáku na Kolej 17. listopadu
je v noci mnohem rychlejší než přes den. Vhodnou stanicí, jak se dostat v noci z
centra, je I. P. Pavlova, kde staví noční autobusy jedoucí jak na Kuchyňku, tak
na Volhu.
\\\\
Čísla nočních tramvají jsou od 91 do 99, pražské noční autobusy mají čísla 901
až 915 a příměstské pak 951 až 960.

\subsubsubsection{Lanovky}
V Praze existují tři lanovky – na Petřín, v zoo a u hotelu NH Praha (dříve hotel
Mövenpick). Na první jmenovanou vám stačí lítačka. Lanovka má celkem tři stanice
– Újezd, Nebozízek a Petřín. Kromě jarní, letní, podzimní a zimní přestávky
jezdí během roku neustále. Většinou je však naplněna turisty a dětmi. Lanovka v
zoo nepatří do PID, je tedy potřeba si pro jízdu lanovkou pořídit lístek u
vchodu do lanovky. Lanovka u hotelu je soukromá, ale je možno ji využívat.

\subsubsubsection{Náhradní doprava}
Velmi často se stává, že je v Praze nějaká výluka. V tom případě se objeví
informace na zastávkách povrchové dopravy a ve stanicích metra a zároveň se
vyrojí autobusy a tramvaje s divným označením. Dle zvyklostí se tramvaje, pokud
zajišťují náhradní dopravu, označují čísly 31-49, autobusy potom písmenem X a
číslem (většinou logicky, např. je-li třeba zajistit náhradní dopravu za tramvaj
č. 3, jezdí autobus X3). Pokud je výlukou postiženo metro, bývá náhradní doprava
označována X{A,B,C}. Většinou jsou na náhradní trasy nasazovány autobusy, pokud
je však výluka na lince A, je náhradní doprava zajišťována zpravidla tramvajemi.

\subsubsubsection{Kolo}
Výhodou kola je, že na většině tras je rychlejší než autobus či tramvaj. Pokud
jedete na delší vzdálenost, můžete se i s kolem svézt metrem, vlakem, přívozem,
lanovkou a ve vybraných úsecích i tramvajemi. Za přepravu kola platit nemusíte,
stačí, že máte jízdenku pro sebe. V některých stanicích metra můžete pro kolo
využít výtah, ale většinou ho potáhnete po eskalátorech. V metru se kolo
přepravuje na zadní plošině každého vagónu a kromě prvního vagónu i na první
plošině, nejlepší je poslední vagón.

\subsubsubsection{Koloběžka}
Skládací koloběžka je super na kratší přesuny (třeba z koleje na metro), zadarmo
ji můžete přepravovat ve všech prostředcích MHD, i když se na vás občas řidiči v
autobusech tváří kysele. Velká koloběžka je, z hlediska pravidel MHD, kolem.


\subsubsection{Specifika pro Kolej 17. listopadu}
\subsubsubsection{Přístupové cesty na kolej}
Nejdůležitějším bodem, odkud se cestuje na kolej, je stanice metra Nádraží
Holešovice. Odtud se na kolej dá jít pěšky přes most nebo se jede autobusem.
Cesta pěšky trvá patnáct minut, ale pokud \textit{opravdu} spěcháte, můžete to
uběhnout i za sedm minut. Autobusem se na kolej jezdí buď linkou 112 na zastávku
Pelc Tyrolka, která je koleji nejblíže, nebo na Kuchyňku autobusem číslo 201. Až
se budete rozhodovat, jestli jet na Kuchyňku, nebo na Pelc Tyrolku, vězte, že
průměrnému matfyzákovi trvá cesta z Kuchyňky o 3 – 4 minuty déle než cesta z
Pelc Tyrolky. V opačném směru, tj. z koleje pryč, autobus 112 nejezdí.
\\\\
Při cestách od Výstaviště (tedy např. ze Strossmayeráku nebo Právnické fakulty)
se s výhodou používá tramvaj 17 do zastávky Trojská (z koleje: po silnici k zoo,
až narazíte na tramvajové koleje). Cesta na tramvaj trvá asi devět minut nebo se
dá využívat 112 ve směru k zoo na stanici Povltavská.
\\\\
Každý matfyzák dříve nebo později objeví, že tramvajové zastávky s názvem
"Nádraží Holešovice" jsou dvě, a to tehdy, když zrovna vystoupí na té špatné a
zmateně se rozhlíží, kde to je. Do odlehlejší zastávky jezdí tramvaj 17.

\subsubsubsection {Dvěstějednička}
Po zrušení zastávky Pelc Tyrolka ve směru Nádraží Holešovice se stala aspirantem
na kolejní autobus. Nicméně v tomto směru jí šance kazí několik faktorů –
navzdory optimistickému jízdnímu řádu je pravděpodobnost příjezdu autobusu
rovnoměrně a spojitě rozdělena v průběhu celého dne. Ze strany dopravního
podniku byla snaha situaci vylepšit a bylo provedeno metodicky bezchybné měření
v pátek během letního zkouškového období, kdy se zjistilo, že většina lidí
potřebuje autobus mezi 8-9 ranní. Byly tedy zkráceny intervaly v tuto dobu,
bohužel s malým úspěchem. Na druhou stranu to matfyzákům poskytuje prostor pro
filozofické debaty, jestli ujel jejich autobus, který má jet za minutu, nebo
autobus, který měl odjet před devíti minutami. Nejen proto se objevují neověřené
zvěsti, že to vše je v rámci moderní inscenace známého díla Čekání na Godota.

\subsubsubsection {Stodvanáctka}
Stodvanáctka jezdí do zoo. Původně to byl náš kolejní autobus, dokud kolem
koleje jezdil ještě oběma směry. Je několik bezúspěšných snah, aby 112 opět
stavěla u koleje i ve druhém směru. Většinou na ní nejsou návaly, ale u
některých spojů se z ní stává induktivní autobus (když už se tam vešlo n lidí,
vejde se jich tam i n+1), nevejdou se pouze ti, kteří vyměknou. Bohužel indukce
se nevztahuje na kočárky. Ačkoliv neexistuje v autobuse místo, kam by se kočárek
nedokázal procpat, je mnohdy problémem se z autobusu přes vzájemně zaklíněné
kočárky dostat. Velice poučné je v období školních výletů sledovat marné snahy
učitelek narvat do stodvanáctky celou třídu najednou.
\\\\
\textbf{Hra}
\\
Hru hraje každý cestující autobusem 112. Každý, kdo v autobuse zmáčkne tlačítko
STOP, prohrál. Pokud na zastávce nestojí lidi, hraje se tzv. hardcore mode.
Nejbližší zastávka bez znamení je zoo, odtud je možno jet autobusem 112 zpět na
Holešovice, nicméně ne kolem koleje. Legenda praví, že se celé kolo dá objet i
dvakrát, než autobus zastaví někdo zvenčí.


\subsubsection{Ideální algoritmy MHD}
Níže naleznete několik doporučených tras, které se vám pravděpodobně budou
hodit. Pokud máte času nazbyt a pro strach uděláno, nebojte se experimentovat a
nalézt si vaši vlastní optimální trasu.
\\\\
V~následujících doporučených cestách pomocí MHD je použito toto
značení:
\\\\
\begin{tabularx}{\textwidth}{ |l|X| }
\hline
M C (X \ra Y) & metro, trasa C, z~X do Y \\
\hline
T 8, 24 (X \ra Y)/2 & tramvaje číslo 8 a 24 z~X do Y a jsou to dvě
zastávky \\
\hline
B 201 (X \ra Y)/2 & autobus číslo 201 (zbytek jako u~tramvaje)\\
\hline
V~221 (X \ra Y) & vlak na trati 221 z~X do Y \\
\hline
\end{tabularx}

\subsubsubsection{Cesty od koleje 17. listopadu k~budovám fakulty}
Vaše cesty z koleje zpravidla povedou přes Nádraží Holešovice, kam se pomocí MHD
lze dostat dvěma způsoby, a tedy můžete použít následující substituci: B 201
(Kuchyňka \ra Nádrhol)/2 = B 112 (Pelc Tyrolka \ra Povltavská)/1, T 17 (Trojská
\ra Nádrhol (sever))/1 = B 112 (Pelc Tyrolka \ra Povltavská)/1, B 112 (Trojská
\ra Nádrhol (jih))/1. Nověji je téměř vždy výhodnější jít pěšky po mostě a na
první křižovatce doprava až k Nádrholu. Krom toho, že časový rozptyl je
minimální a spolehlivost maximální, tak na metro přijdete ze strany, která bude
poté na Pavláku/Florenci nejblíže výstupu. Pokud nepotřebujete na metro, ale
směrem na Štrosmajerák, vyplatí se jít pěšky na Trojskou (do 10 min).
\\\\
\textbf{Troja (fyzika - učebny T)}\\
Pěšky po chodníku směrem k mostu, projít pod mostem do nízké modrošedé budovy
spojené s věžákem stejné barvy.
\\\\
\textbf{Troja (angličtina - učebny V)}\\
Pěšky po chodníku směrem k mostu, před ním zleva obejít parkoviště-vrakoviště,
projít pod mostem a rovně do nízké budovy nespojené s věžákem stejné barvy.
\\\\
\textbf{Karlín}\\
B 201 (Kuchyňka \ra Nádrhol)/2, M C (Nádrhol \ra Florenc), T 3, 8, 24 (Florenc
\ra Křižíkova)/2, přejít ulici, budova naproti, nebo i pěšky z Florence. Tato
volba je průměrně nejlepší, nicméně s relativně velkým rozptylem.
\\\\
Alternativně lze zvolit cestu pouze s jedním přestupem přes Bulovku, tj. B 201
(Kuchyňka \ra Bulovka)/3, T 3, 24 (Bulovka \ra Křižíkova)/8. Budova je pár
desítek metrů za zastávkou po pravé straně.
\\\\
Pěkná a rychlá je i cesta na kole podél řeky: po cyklotrase A2 od kolejí směrem
na Palmovku přes Thomayerovy sady a Rohanský ostrov do Karlína. Ve dvoře budovy
MFF je použitelný stojan. Přístup do dvora je druhými vraty, zvoňte na vrátnici.
\\\\
\textbf{Karlov} \\
B 201 (Kuchyňka \ra Nádrhol)/2, M C (Nádrhol \ra Pavlák), vylézt směr ulice Na
Bojišti, přejít po semaforech magistrálu, projít kousek po její pravé straně
nahoru směrem k Vyšehradu, zahnout doprava (ulice Na Bojišti), na konci doleva,
pořád rovně, poslední dvě budovy na pravé straně jsou budovy Ke Karlovu 3 a 5 (v
opačném pořadí). Z Pavláku lze též jet jednu zastávku autobusem 148 do zastávky
Dětská nemocnice Karlov a tím se přiblížit ke škole.
\\\\
\textbf{Malá Strana}\\
Existuje několik možností, jak se tam pomocí autobusu, metra a tramvaje dostat.
Asi nejrychlejší z nich je dostat se pomocí autobusu 112 nebo pěšky na zastávku
Trojská a odtud T 17 (Trojská \ra Čechův most)/6, T 15 (Čechův most \ra
Malostranské náměstí)/2.
\\\\
V závislosti na denní době a ročním období časově ekvivalentní, či trochu
pomalejší spojení je pomocí metra, a to B 201 (Kuchyňka \ra Nádrhol)/2 (nebo
pěšky), M C (Nádrhol \ra Muzeum), M A (Muzeum \ra Malostranská), T 12, 15, 20,
22, 23 (Malostranská \ra Malostranské náměstí)/1.
\\\\
Pokud neradi přestupujete a nikam nespěcháte, můžete se svézt z Holešovic až na
Malou Stranu pomocí tramvaje číslo 12. Doporučujeme přibalit svačinu a
dostatečně tlustou knížku.
\\\\
Mnohdy je ale doprava kolem Malostranské zkolabovaná, tudíž je občas lepší jít z
Malostranské (častěji na Malostranskou) pěšky, můžete přitom projít
Valdštejnskou zahradou.
\\\\
Taky je možné využít kolo: pokud nechcete kopírovat trasu tramvají 12 nebo 17,
můžete se vydat po cyklostezce směrem k zoo, Trojskou lávkou (resp. přívozem na
jejím místě) projdete do Stromovky a nahoru do Letenského parku. Nakonec sjedete
Chotkovy sady a cesta vpravo před Malostranskou vede až na náměstí před školou.
Kola se parkují v chodbě vlevo (u automatů) nebo ve dvoře.
\\\\
\textbf{Hostivař}\\
B 201 (Kuchyňka \ra Nádrhol)/2, M C (Nádrhol \ra Hlavák), V 221 (Hlavák \ra
Hostivař) - vlak jezdí pod krycím názvem S9, pěšky po silnici podél trati směrem
na Benešov, dokud zpoza paneláků nevykoukne SCUK, nebo kolem rybníka dolů a
sídlištěm k SCUKu. Alternativně pěšky přes silnici a dolů na konečnou tram a B
125 (Nádr. Hostivař \ra Gercenova)/1.
\\\\
Nebo: B 201 (Kuchyňka \ra Nádrhol)/2, M C (Nádrhol \ra Muzeum), M A (Muzeum \ra
Skalka), B 125 (Skalka \ra Gercenova)/7, příčnou ulicí (to je Gercenova), vlevo
kolem nákupního střediska.
\\\\
Nebo: B 201 (Kuchyňka \ra Střížkov)/7, B 183 (Střížkov \ra Gercenova)/18. Tato
varianta je spolehlivá především mimo špičku.

\subsubsubsection{Cesty mezi jednotlivými budovami fakulty}
\textbf{Karlov --- Karlín}\\
Pěšky na Pavlák, M C (Pavlák \ra Florenc), T 3, 8, 24 (Florenc \ra Křižíkova)/2,
přejít ulici a vejít dovnitř.
\\\\
\textbf{Troja --- kamkoliv}\\
jako z koleje
\\\\
\textbf{Karlov --- Malá Strana}\\
Pěšky celou ulicí Ke Karlovu, až dojdete k tramvajovým kolejím, pak vlevo a za
další křižovatkou na pravé straně je zastávka Štěpánská, T 22, 23 (Štěpánská \ra
Malostranské náměstí)/6. Cestu si můžete zkrátit přes Kateřinskou zahradu.
Bojíte-li se jít na Štěpánskou, tak pěšky na Pavlák a T 22, 23 (Pavlák \ra
Malostranské náměstí)/7. Hlavně na Pavláku nepodlehněte nacvičené cestě do
metra.
\\\\
\textbf{Karlín --- Malá Strana}\\
M B (Křižíkova \ra Národní třída), T 22, 23 (Národní třída \ra Malostranské
náměstí)/4.
\\\\
Nebo: T 8 (Křižíkova \ra Náměstí Republiky)/4, T 15 (Náměstí Republiky \ra
Malostranské náměstí)/4.
\\\\
Nebo: M B (Křižíkova \ra Můstek), M A (Můstek \ra Malostranská), T 12, 15, 20,
22, 23 (Malostranská \ra Malostranské náměstí)/1. Často rychlejší než výše
uvedené pokud nejede správná tramvaj a opravdu spěcháte (seběhnete schody,
proběhnete přestup na metro A, vyběhnete schody). Takto se dá nejlépe jezdit,
pokud přebíháte mezi oběma budovani o 10 min přestávce, na další hodinu přijdete
většinou o cca 10 min později.