\subsection{Spolek Matfyzák}

\textit{Spolek Matfyzák} je studentská organizace, která pro matfyzáky pořádá
různé společenské, kulturní a další akce – krom jiného může i za spáchání této
kuchařky. Jejím členem může být každý (i bývalý) student MFF, či osoba jinak s
Matfyzem spřízněná.
V čele spolku stojí Náčelník, který je obklopen Náčelnictvem. Na to, aby se ve
spolku neděly nekalosti, dohlíží Rada starších. Web spolku je na matfyzak.cz.
Kontaktovat nás můžete na adrese spolek@atfyzak.cz nebo také přes Facebook.
Spolek není personálně stálá organizace; mění se, jak studenti přicházejí a
odcházejí. Staňte se aktivním členem Spolku Matfyzák! Pokud rádi organizujete,
máte podnětné nápady nebo chcete jen vytáhnout nějaká moudra ze starších
spolužáků, Spolek je k tomu ideální. Stačí se ozvat. Naučíte se přitom spoustu
užitečných dovedností a získáte přehled a další kamarády.


\subsubsection{Beánie}

Beánie je mnohovrstevnatá studentská slavnost konající se tradičně v období
imatrikulací. Vrstvou nejvážnější jest přijímání prváků do stavu matfyzáckého
podle prastarého Jarníkova rituálu. V dalších vrstvách se obvykle promítá
kultovní film Matfyzák a jiné matfyzácké filmy, pořádají se koncerty
matfyzáckých i nematfyzáckých hudebních skupin, bývají různé humorné soutěže a
scénky, či slavná výstava Matfyzák včera, dnes a zítra. Ve vrstvě komerční si
můžete koupit matfyzácké tričko nebo jiné artefakty s logem i bez něj.
Dříve se dařilo všechny tyto vrstvy vměstnat do některého z pražských hudebních
klubů, nicméně poslední roky je radši rozprostíráme po větších prostorách
trojského areálu Matfyzu.

\subsubsection{Matfyzácký ples}

Matfyzácký ples v krásných prostorách paláce Žofín je méně exotickým způsobem
strávení večera. Každý matfyzák se tam snaží blýsknout se před svými
přednášejícími perfektně vykrouženou spinovou otáčkou ve waltzu (protože
zkouškové se blíží a nikdy nevíte, co vám může pomoci), případně alespoň něco
vyhrát v tombole. Vstupenky seženete několik týdnů před plesem v knihovnách, na
koleji nebo přes internet. Rezervujte si je včas, bývají brzo vyprodané.


\subsubsection{Další činnost Spolku}

Krom těchto akcí pořádáme každoročně několik dalších. Některé z nich se konávají
na Koleji 17. listopadu - patří mezi ně například vědomostní běhací hra KolejCup
a Běh do schodů, jehož průběh si lze domyslet (současný rekord je 1 minuta a 36
vteřin). Mezi populární akce patří Tour de Pub - pravidelně nepravidelná
návštěva nějaké hospody s přátelským popovídáním nad půllitrem piva nebo nepiva.
Stačí sledovat webové stránky nebo nástěnky ve fakultních budovách a o blížících
se akcích se včas dozvíte.
Dále Spolek spolupracuje s jinými organizátory, divadly (sledujte naše stránky
pro slevy do několika z nich), prodává trička s logem Matfyzu a jiné předměty
(e-shop je na webu) a spravuje tuto kuchařku.

\subsubsubsection{Celouniverzitní akce}
Mimo matfyzácké akce se Spolek Matfyzák účastní i akcí celouniverzitních, jako
je třeba akce pro studenty - Studentský jarmark. Dozvíte se na něm, co všechno
můžete na univerzitě podniknout, poslechnete si muziku, otestujete menu
podzemního klubu a představí se vám jednotlivé spolky.
Kromě toho Spolek zveřejňuje pozvánky na zajímavé univerzitní akce nebo
přednášky.


\subsubsection{Matfyz Boost}

Pokud máte nějaký nápad na matfyzáckou akci, který chcete realizovat, můžete se
nám ozvat a rádi vám s ní pomůžeme. Nebo taky můžete využít soutěž Matfyz Boost
a získat tak až pět tisíc na svůj projekt. Více informací je na stránkách
boost.matfyzak.cz.


\subsubsection{Celouniverzitní spolky}

Na univerzitě pak existuje spousta dalších spolků, nejaktivnější v tom jsou
filozofové a právníci. Trochu jiné postavení má Centrum spolků, studentů a
absolventů UK (cssauk.cz), které tyto spolky v rámci univerzity sdružuje; je
tedy takovým spolkem spolků. Mezi současné celouniverzitní aktivity patří
například Studentský jarmark či Studentský majáles, Centrum organizuje
studentské oslavy 17. listopadu na Albertově a besedy s významnými osobami nebo
zprostředkovává spolkům kontakt s vedením univerzity.