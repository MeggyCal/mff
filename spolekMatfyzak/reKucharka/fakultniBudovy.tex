\subsection{Fakultní budovy}
Naše fakulta sestává z mnoha budov rozesetých po celé Praze. Nejčastěji budete
navštěvovat následující čtyři lokality: Karlov, kde sídlí děkanát a fyzici,
Malou Stranu, která je sídlem informatiků, Karlín, jenž je domovem pro většinu
matematiků, a Troju, kde mají laboratoře a posluchárny fyzici a jsou zde také
učebny pro výuku cizích, neprogramátorských, jazyků.


\subsubsection{Karlov}
V oblasti Karlova a sousedícího Albertova (pod Karlovem) zabírá naše univerzita
většinu budov; kromě nás jsou tam ještě přírodovědci a lékaři. Matfyzu patří dvě
budovy na konci ulice Ke Karlovu.


\budovaInfo{Ke Karlovu 3, Praha 2}
{V~budově se místnosti označují písmenem \uv{M} (M1, M2, M3, M5 a M6)}
{Matematici, fyzici, děkanát, studijní.}
{Stojan ve dvoře}

\noindent Budova Ke Karlovu 3 je centrální nervovou soustavou fakulty. Sídlí zde
děkan, schází se tu vedení, akademický senát i vědecká rada. Cesty studentů
(pokud právě neběží na přednášku nebo cvičení) končí často hned v přízemí na
studijním oddělení. Na přednášky sem chodí především matematici v prvním ročníku
– M1 je jedinou matematikům dostupnou místností, kam je možné vtěsnat celou
paralelku pro prváky. Občas sem zavítají i fyzici, jelikož tu také probíhají
základní fyzikální praktika. V suterénu se vyskytoval dobře zásobený (i když
trochu drahý) bufet, nicméně ten byl zrušen. Na chodbě v prvním patře také
naleznete Galerii vědeckého obrazu.


\budovaInfo{Ke Karlovu 5, Praha 2}
{Budova se honosí písmenem \uv{F} (F1, F2 a KFK)}
{Fyzici}
{Stojan ve dvoře sousední budovy}

\noindent V budově Ke Karlovu 5 sídlí Fyzikální ústav UK. V posluchárně F1
probíhá v zimním semestru „exhibiční přednáška“ Fyzika v experimentech. Vřele
doporučujeme navštívit (a zatleskejte jim, budou mít radost). Samotná F1 je
krásná posluchárna – je plná nejrůznějších fyzikálních hraček. Budete-li v ní
mít přednášku, jistě ji oceníte.

\subsubsection{Malá Strana}

\budovaInfo {Malostranské náměstí 25, Praha 1}
{Označují se písmenem \uv{S} (S1, S3, \dots, S11), čtyři počítačové učebny (SW1,
SW2, SU1, SU2), rotunda, spousta seminárních místností. Celá budova je často
označována jako MS.}
{Informatici}
{Stojany v chodbě vlevo (před SW2) a ve dvoře}

\noindent Budova informatiků je přilepená k chrámu sv. Mikuláše, je krásná
a poměrně nově zrekonstruovaná. Došlo i k restauraci některých maleb na
zdech, najdete-li si chvíli, doporučujeme exkurzi na zadní schodiště. Krásu
budovy potvrzuje i to, že se v místní aule konají bakalářské promoce celé
univerzity. Kvůli své poloze v historickém centru je budova
uzamčena a dveře se odemykají přes zaměstnanecké a studentské průkazy.

Při zmíněné rekonstrukci se našla Rotunda sv. Václava z jedenáctého století,
ježbyla ztracena téměř čtyři sta let, od roku 2016 je přístupná veřejnosti.

Běžným aprílovým vtipem starších matfyzáků je poslat
studentydo místnosti S2, která už nějakou dobu neexistuje.

Místem, kde můžete přečkat přestávku mezi přednáškami, je velký počítačový
labumístěný v přízemí. Za pěkného počasí bývá též otevřeno posezení na střeše
labu.Vchod naleznete na vrcholu schodiště, kterým se sestupuje do
knihovny.Sejdete-li po tomto schodišti až dolů, dostanete se k trezorům, a pokud
vám topořád nestačí, na konci chodby vlevo jsou schody, kterými se dá sejít
ještěhlouběji. Tam se nachází robotická dílna.

Přepadne-li vás během dne na Malé Straně hlad, jistě oceníte stravovací
zařízení\textit{Profesní dům} v suterénu. Vaří dobře a matfyzáci tam mají od
nedávna dojednanou slevu. Lepší cenou se pyšní bagetomat v rotundě. V
okolí můžete zajít do ulice ke Karlovu mostu, kde je vpravo McDonald's, dále
vlevo čínské bistro nebo skoro u mostu vpravo Biomarket Vacek. Asi 100 metrů po
tramvajových kolejích ve směru k Újezdu (nikoliv tedy tunýlkem), je čínské bistro
a hned vedle i bagetérie Subway. Pokud by to pořád bylo málo, na Dražického
náměstí najdete školní jídelnu.


\subsubsection{Karlín}

\budovaInfo{Sokolovská 83, Praha 8}
{\uv{K} (K1--K12) + seminární místnosti}
{Matematici}
{Stojan ve dvoře (zvoňte vrátnici)}

\noindent Výsadní právo na tuto čtyřpatrovou budovu mají matematici, sídlí
tu matematické katedry. V přízemí je lab a prodejna MatfyzPressu. Naproti
přes ulici je prodejna potravin a pekárna. Napravo od něj je také čínské bistro,
kde je po domluvě možné si jednu přestávku jídlo objednat a další vyzvednout
(hodí se, když máte vyučování bez pauzy na oběd).

Nejužitečnější místností na Karlíně je bezesporu \textit{Respirium}. Vlastně je
to hned po malostranském labu druhá nejlepší místnost Matfyzu. Najít ho nelze
úplněsnadno - je umístěno v podzemí. Poté, co vejdete do budovy, tak u
vrátnice zabočte vlevo, sejděte pár schodů dolů, projděte přes skleněné dveře,
dolů po schodech, chodbou doleva a zhruba uprostřed chodby jsou po pravé straně
dveře s nápisem Respirium. Na chodbě často nesvítí světla, takže k nalezení
správných dveří to chce buď trochu odvahy a štěstí, nebo baterku.

Pokud se do Respiria dostanete, můžete využít veškerý jeho komfort:
mikrovlnku,varnou konvici, tabuli pro hromadné řešení úkolů se spolužáky,
knihovničku a pohodlná křesílka na odpočinek.


\subsubsection{Troja}

\budovaInfo{V Holešovičkách 2, 180 00 Praha 8}
{Označeny písmeny \uv{T}, \uv{A}, \uv{V}, \uv{C} a \uv{L} (T1, T2, T5--T10,
VxxxJAZ) + seminární místnosti na
jednotlivých katedrách}
{Fyzici}
{Stojany před budovami a v chodbě před vrátnicí L}

\noindent Budov v Troji je pět – jedna vysoká a čtyři placaté.
Hlavní vchod vede do hlavní budovy. Zde jsou velké posluchárny T1 a T2 a v prvním patře malé T5 až T10.
V přízemí je bufet, který si lidé pochvalují, skriptárna a neopomenutelný lab.

Průchodem se dostanete do vysoké budovy. Ta patří výlučně fyzikům (a nejen našim,
sídlí zde i FJFI ČVUT).
Za hlavní budovou jsou těžké laboratoře, v nichž se nachází, mimo jiné,
\textit{Van der Waalsův urychlovač} a jaderňácký reaktor VR1, krycím jménem
\textit{Vrabec}.

V budově položené nejblíže magistrále je katedra jazykové přípravy,
fyzici nízkých teplot, astronomové. Celý komplex lze projít „suchou
nohou“ díky důmyslnému systému katakomb, který se pod ním nachází, a lávce mezi
hlavní budovou a těžkými laboratořemi.

Mezi jazyky a lesem na Bílé skále leží kryopavilon. Zde se zkapalňuje helium a nacházejí se
laboratoře katedry fyziky nízkých teplot a Fyzikálního ústavu AV ČR. V prvním
patře je seminární místnost C126. Tato budova není přístupná chodbami
z ostatních objektů.

Poslední budovou, která se ještě staví, je nový informatický pavilon, který má být dostavěn do konce roku 2019.