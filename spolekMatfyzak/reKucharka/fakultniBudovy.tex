\subsection{Fakultní budovy}
Naše fakulta sestává z mnoha budov rozesetých po celé Praze. Nejčastěji budete
navštěvovat následující čtyři lokality: Karlov, kde sídlí děkanát a fyzici,
Malou Stranu, která je sídlem informatiků, Karlín, jenž je domovem pro většinu
matematiků, a Troju, kde mají laboratoře a posluchárny fyzici a jsou zde také
učebny pro výuku cizích, neprogramátorských, jazyků.


\subsubsection{Karlov}
V oblasti Karlova a sousedícího Albertova (pod Karlovem) zabírá naše univerzita
většinu budov; kromě nás jsou tam ještě přírodovědci a lékaři. Matfyzu patří dvě
budovy na konci ulice Ke Karlovu.


\budovaInfo{Ke Karlovu 3, Praha 2}
{V~budově se místnosti označují písmenem \uv{M} (M1, M2, M3, M5 a M6)}
{Matematici, fyzici, děkanát, studijní.}
{Stojan ve dvoře}

\noindent Budova Ke Karlovu 3 je centrální nervovou soustavou fakulty. Sídlí zde
děkan, schází se tu vedení, akademický senát i vědecká rada. Cesty studentů
(pokud právě neběží na přednášku nebo cvičení) končí často hned v přízemí na
studijním oddělení. Na přednášky sem chodí především matematici v prvním ročníku
– M1 je jedinou matematikům dostupnou místností, kam je možné vtěsnat celou
paralelku pro prváky. Občas sem zavítají i fyzici, jelikož tu také probíhají
základní fyzikální praktika. V suterénu se vyskytoval dobře zásobený (i když
trochu drahý) bufet, nicméně ten byl zrušen. Na chodbě v prvním patře také
naleznete Galerii vědeckého obrazu.


\budovaInfo{Ke Karlovu 5, Praha 2}
{Budova se honosí písmenem \uv{F} (F1, F2 a KFK)}
{Fyzici}
{Stojan ve dvoře sousední budovy}

\noindent V budově Ke Karlovu 5 sídlí Fyzikální ústav UK. V posluchárně F1
probíhá v zimním semestru „exhibiční přednáška“ Fyzika v experimentech. Vřele
doporučujeme navštívit (a zatleskejte jim, budou mít radost). Samotná F1 je
krásná posluchárna – je plná nejrůznějších fyzikálních hraček. Budete-li v ní
mít přednášku, jistě ji oceníte.


\subsubsection{Malá Strana}

\budovaInfo {Malostranské náměstí 25, Praha 1}
{Označují se písmenem \uv{S} (S1, S3, \dots, S11), čtyři počítačové učebny (SW1,
SW2, SU1, SU2), rotunda, spousta seminárních místností. Celá budova je často
označována jako MS.}
{Informatici}
{Stojany v chodbě vlevo (před SW2) a ve dvoře}

\noindent Budova informatiků je přilepená k chrámu sv. Mikuláše, je krásná
apoměrně nově zrekonstruovaná (už se nestává, že nesvítí světla, protože
vypadljistič o dvě patra výše). Došlo i k restauraci některých maleb na
zdech,najdete-li si chvíli, doporučujeme exkurzi na zadní schodiště. Krásu
budovypotvrzuje i to, že se v místní aule konají bakalářské promoce celé
univerzity.Kvůli své poloze v historickém centru je od začátku roku 2014 budova
uzamčena adveře se odemykají přes zaměstnanecké a studentské průkazy.

Při zmíněné rekonstrukci se našla Rotunda sv. Václava z jedenáctého století,
ježbyla ztracena téměř čtyři sta let, od roku 2016 je přístupná veřejnosti.

Malostranská budova je opředena mnoha pověstmi a příběhy, např. v
opuštěnýchtrezorových místnostech Ústřední státní pokladny, které se nacházejí
pod celoubudovou, byl nalezen obrovský transformátor, po kterém rozvodné závody
vyhlásilyuž kdysi dávno celomalostranské pátrání. Záhada dodávky proudu do
okolníchmalostranských paláců obsazených Parlamentem ČR a spol. tak byla
konečněvyřešena.

Povídky, které se o budově mezi lidmi proslýchají, se mnohdy vzájemně
vylučují.Povídka RNDr. Kryla tvrdí, že Matfyz má s blízkým kostelem sv.
Mikulášespolečnou jen jednu tlustou neproniknutelnou zeď, což se však vylučuje s
jinouhistorkou, podle níž jednou v době oprav chrámu vylezl v průběhu přednášky
zjedné skříně v tehdejší posluchárně, zvané saloon, zedník, který zabloudil
vesvatomikulášských tajných chodbách. Pikantní na tom je, že se tak stalo
právěběhem Krylovy přednášky. Zedník se zadíval na tabuli, pak polohlasně
zahuhlal „Asakra!“, a zalezl zpátky do skříně. Faktem je, že dveře jsou ve
skříni dosud,jen proti stavu před cca 15 lety, kdy zůstávaly odemčené a dalo se
jimi projít,jsou v dnešní době už vždy zamčené. Místnost už není posluchárnou,
ale je v níkancelář SISALu. Jako chabou náhradu za tuto dnes již nedostupnou
atrakci lzemilovníkům zvláštností doporučit návštěvu posluchárny S1, která se
nachází zedvou třetin již za zmíněnou neproniknutelnou zdí a zasahuje částečně
až nadchrámovou loď. Tím navíc způsobuje neobvyklou situaci, kdy se půdorysy
dvoubudov překrývají. Běžným aprílovým vtipem starších matfyzáků je poslat
studentydo místnosti S2, která už nějakou dobu neexistuje.

Místem, kde můžete přečkat přestávku mezi přednáškami, je velký počítačový
labumístěný v přízemí. Za pěkného počasí bývá též otevřeno posezení na střeše
labu.Vchod naleznete na vrcholu schodiště, kterým se sestupuje do
knihovny.Sejdete-li po tomto schodišti až dolů, dostanete se k trezorům, a pokud
vám topořád nestačí, na konci chodby vlevo jsou schody, kterými se dá sejít
ještěhlouběji. Tam se nachází robotická dílna.

Přepadne-li vás během dne na Malé Straně hlad, jistě oceníte stravovací
zařízení\textit{Profesní dům} v suterénu. Vaří dobře a matfyzáci tam mají od
nedávna dojednanouslevu. Lepší cenou se pyšní i nově nainstalovaný bagetomat. V
okolí můžete zajítdo ulice ke Karlovu mostu, kde je vpravo McDonald's, dále
vlevo čínské bistronebo skoro u mostu vpravo Biomarket Vacek. Asi 100 metrů po
tramvajovýchkolejích ve směru k Újezdu (nikoliv tedy tunýlkem), je čínské bistro
a hnedvedle i bagetérie Subway. Pokud by to pořád bylo málo, na Dražického
náměstínajdete školní jídelnu.

Na Malostranském náměstí sídlí také HAMU. Máte-li rádi vážnou hudbu, můžete
takskoro každý večer po škole zadarmo či skoro zadarmo na koncert – stačí
býtnevandrácky oblečen (oblek ale určitě potřeba není).


\subsubsection{Karlín}

\budovaInfo{Sokolovská 83, Praha 8}
{\uv{K} (K1--K12) + seminární místnosti}
{Matematici}
{Stojan ve dvoře (zvoňte vrátnici)}

\noindent Výsadní právo na tuto čtyřpatrovou budovu mají matematici, sídlí
tumatematické katedry. V přízemí je lab a prodejna MatfyzPressu. Naproti
přesulici je prodejna potravin a pekárna. Napravo od něj je také čínské bistro,
kdeje po domluvě možné si jednu přestávku jídlo objednat a další vyzvednout
(hodíse, když máte vyučování bez pauzy na oběd).

Nejužitečnější místností na Karlíně je bezesporu \textit{Respirium}. Vlastně je
to hnedpo malostranském labu druhá nejlepší místnost Matfyzu. Najít ho nelze
úplněsnadno - je umístěno v podzemí. Poté, co vejdete do budovy, tak u
vrátnicezabočte vlevo, sejděte pár schodů dolů, projděte přes skleněné dveře,
dolů poschodech, chodbou doleva a zhruba uprostřed chodby jsou po pravé straně
dveře snápisem Respirium. Na chodbě často nesvítí světla, takže k nalezení
správnýchdveří to chce buď trochu odvahy a štěstí, nebo baterku.

Pokud se do Respiria dostanete, můžete využít veškerý jeho komfort:
mikrovlnku,varnou konvici, tabuli pro hromadné řešení úkolů se spolužáky,
knihovničku apohodlná křesílka na odpočinek.


\subsubsection{Troja}

\budovaInfo{V Holešovičkách 2, 180 00 Praha 8}
{Označeny písmeny \uv{T}, \uv{A}, \uv{V}, \uv{C} a \uv{L} (T1, T2, T5--T10,
VxxxJAZ) + seminární místnosti na
jednotlivých katedrách}
{Fyzici}
{Stojany před budovami a v chodbě před vrátnicí L}

\noindent Budov v Troji je pět – jedna vysoká a čtyři placaté. Placaté se zvou:
\textit{hlavní}
(nebo také \textit{objekt poslucháren}), \textit{těžké laboratoře},
\textit{vývojové dílny} (tady jsou
jazyky a nízké teploty) a \textit{kryopavilon}, vysoká pak \textit{katedrový
objekt}.

Hlavní vchod vede zcela neočekávaně do hlavní budovy. Zde jsou
novězrekonstruované velké posluchárny T1 a T2 a v prvním patře malé T5 až T10.
Vpřízemí je bufet, který si lidé pochvalují, skriptárna a neopomenutelný lab.V
minulosti tady byl oddělený zasklený prostor, kde
sídlilaskriptárna-knihkupectví-papírnictví, a první patro bylo zabráno
GymnáziemBernarda Bolzana.

Průchodem se dostanete do vysoké budovy. Ta patří výlučně fyzikům (a nejennašim,
sídlí zde i FJFI ČVUT). V letech 2009 a 2010 došlo k výměně vnějšíhoobložení
budovy, takže se nemusíte bát, že by na vás spadla. K tomu se také vážehistorka
o tom, že původní obložení bylo otočeno naruby. Sice stavební firmaprohlásila,
že šlo o náhodu, nicméně členové katedry statistiky spočetli, ženáhoda byla
spíše to, že to teď dali správně.

Za hlavní budovou jsou těžké laboratoře, v nichž se nachází, mimo jiné,
\textit{Van derWaalsův urychlovač} a jaderňácký reaktor VR1, krycím jménem
\textit{Vrabec} (podbízí sepřirozená otázka, co ve skutečnosti znamená ono rčení
„jít s kanonem navrabce“). V těžkých laboratořích se provádí polovina úloh
Fyzikálního praktikaIV.

V budově položené nejblíže magistrále je katedra jazykové přípravy,
fyzicinízkých teplot, astronomové a další lab. Celý komplex lze projít „suchou
nohou“díky důmyslnému systému katakomb, který se pod ním nachází, a lávce mezi
hlavníbudovou a těžkými laboratořemi.

Mezi jazyky a lesem na Bílé skále leží nejnovější budova trojského
areálu,kryopavilon z roku 2005. Zde se zkapalňuje helium a nacházejí se
laboratořekatedry fyziky nízkých teplot a Fyzikálního ústavu AV ČR. V prvním
patře jeseminární místnost C126. Tato budova jako jediná není přístupná chodbami
zostatních objektů.